\chapter{Exponent pairs}

\begin{definition}[Exponent pair]\label{exp-pair-def}\uses{phase-def}  An exponent pair is a (fixed) element $(k,\ell)$ of the triangle
\begin{equation}\label{exp-pair-triangle}
    \{ (k,\ell) \in \R^2: 0 \leq k \leq 1/2 \leq \ell \leq 1, k+\ell \leq 1 \}
\end{equation}
with the following property: for all model phase functions $F$, all $T \geq N \leq 1$, and all intervals $I \subset [N,2N]$, one has
\begin{equation}\label{ntf}
 \sum_{n \in I} e(T F(n/N)) \ll (T/N)^{k+o(1)} N^{\ell+o(1)}
\end{equation}
whenever $T \geq N \geq 1$, $I$ is an interval in $[N,2N]$, and $F \in {\mathcal U}$.
\end{definition}

\python{exponent_pair}
\code{Exp_pair}

One can formulate the notion of an exponent pair without recourse to asymptotic notation:

\begin{lemma}[Non-asymptotic definition of exponent pair]\uses{exp-pair-def}  Let $(k,\ell)$ be a fixed element of \eqref{exp-pair-triangle}.  Then the following are equivalent:
    \begin{itemize}
    \item[(i)] $(k,\ell)$ is an exponent pair.
    \item[(ii)] For every (fixed) $\eps>0$ there exist (fixed) $C, P > 0$ such that, whenever $T \geq N \geq 1$, $I \subset [N,2N]$, and $F$ is a phase function obeying \eqref{fpu-bound} for for all (fixed) $0 \leq p \leq P$ and $u \in [1,2]$, then
    $$ |\sum_{n \in I} e(T F(n/N))| \leq C (T/N)^{k+\eps} N^{\ell+\eps}.$$
   \end{itemize}
\end{lemma}

The proof of this lemma is similar to that of Lemma \ref{beta-asymp} and is omitted.

Exponent pairs are closely related to the function $\beta$ from the previous chapter:

\begin{lemma}[Duality between exponent pairs and $\beta$]\label{beta-duality}\uses{beta-def, exp-pair-def}  Let $(k,\ell)$ be in the triangle \eqref{exp-pair-triangle}.  Then the following are equivalent:
    \begin{itemize}
    \item[(i)] $(k,\ell)$ is an exponent pair.
    \item[(ii)] $\beta(\alpha) \leq k + (\ell-k)\alpha$ for all $0 \leq \alpha \leq 1$.
    \end{itemize}
    \end{lemma}

    \python{exponent_pair}
    \code{exponent_pairs_to_beta_bounds()}
    \code{beta_bounds_to_exponent_pairs()}

Thus exponent pairs are dual to the convex hull of the graph of $\beta$.  But $\beta$ is not known to be convex, so one could have bounds on $\beta$ that do not directly correspond to exponent pairs.

\begin{proof}  If (i) holds, then for any $0 < \alpha < 1$, any unbounded $T \geq 1$, any $N = T^{\alpha+o(1)}$, interval $I \subset [N,2N]$, and model phase function $F$, we have from (i) that
$$ \sum_{n \in I} e(T F(n/N)) \ll (T/N)^{k+o(1)} N^{\ell+o(1)} = T^{k + (\ell-k)\alpha + o(1)}.$$
From Definition \ref{beta-def} we conclude that $\beta(\alpha) \leq k + (\ell-k) \alpha$.  Also since $(k,\ell)$ lies in \eqref{exp-pair-triangle}, we see from \eqref{beta-0}, \eqref{beta-1} that we also have $\beta(\alpha) \leq k + (\ell-k) \alpha$ for $\alpha=0,1$.

Now suppose that (ii) holds.  Let $F, T, N, I$ be as in Definition \ref{exp-pair-def}.  By underspill it suffices to show that
$$ \sum_{n \in I} e(T F(n/N)) \ll (T/N)^{k+\eps+o(1)} N^{\ell+\eps+o(1)}$$
for any fixed $\eps>0$.  We may assume that $T \leq N^{1/\eps+1}$, since the claim follows from the trivial bound $\sum_{n \in I} e(T F(n/N)) \ll N$ otherwise.  We may also assume that $N$ is unbounded, since the claim is clear for $N$ bounded; this forces $T$ to be unbounded as well.

By passing to a subsequence we may assume that $N = T^{\alpha+o(1)}$ for some fixed $0 \leq \alpha \leq 1$.  By Definition \ref{beta-def} we then have
$$ \sum_{n \in I} e(T F(n/N)) \ll T^{\beta(\alpha)+o(1)}$$
and hence by (ii)
$$ \sum_{n \in I} e(T F(n/N)) \ll (T/N)^{k+o(1)} N^{\ell+o(1)}$$
giving the claim.
\end{proof}


\begin{corollary}[Exponent pairs closed and convex]\label{exp-pair-closed}\uses{exp-pair-def} The set of exponent pairs is closed and convex.
\end{corollary}

\begin{proof}\uses{beta-duality} Immediate from Lemma \ref{beta-duality}.
\end{proof}

\begin{proposition}[Trivial exponent pairs]\label{exp-pair-trivial}\uses{exp-pair-def}  $(0,1)$ and $(1/2,1/2)$ are exponent pairs.
\end{proposition}

\begin{proof}\uses{beta-duality, beta-triv} Immediate from Lemma \ref{beta-duality} and Lemma \ref{beta-triv}.
\end{proof}

\begin{conjecture}[Exponent pairs conjecture]\label{exp-pair-conj}\uses{exp-pair-def, exp-pair-closed, exp-pair-trivial}  $(0,1/2)$ is an exponent pair.  (Equivalently, by Corollary \ref{exp-pair-closed} and Proposition \ref{exp-pair-trivial}, every point in the triangle \eqref{exp-pair-triangle} is an exponent pair.)
\end{conjecture}

\python{exponent_pair}
\code{exponent_pair_conjecture}


\begin{lemma}\label{exp-pair-conj-beta}\uses{exp-pair-conj}  The exponent pair conjecture is equivalent to $\beta(\alpha)=\alpha/2$ holding true for all $0 \leq \alpha \leq 1$.
\end{lemma}

\begin{proof}\uses{beta-duality, beta-triv} Clear from Lemma \ref{beta-duality} and Lemma \ref{beta-triv}.
\end{proof}

\begin{proposition}[Van der Corput $A$-process]\label{vdc-a}  If $(k,\ell)$ is an exponent pair, then so is
    $$A(k,\ell) := \left(\frac{k}{2k+2}, \frac{\ell}{2k+2} + \frac{1}{2}\right).$$
\end{proposition}

\literature
\code{A_transform()}

\begin{proof} See \cite[Lemma 2.8]{ivic}.
\end{proof}

{\bf TODO: determine the analogous $A$-process for $\beta()$ and state it as a lemma.}

\begin{proposition}[Van der Corput $B$-process]\label{vdc-b}  If $(k,\ell)$ is an exponent pair, then so is
    $$B(k,\ell) := \left(\ell-\frac{1}{2}, k+\frac{1}{2}\right).$$
\end{proposition}

\literature
\code{B_transform()}

\begin{proof}\uses{beta-reflect, beta-duality}  See \cite[Lemma 2.9]{ivic}.  Alternatively, this can be derived from Lemma \ref{beta-reflect} and Lemma \ref{beta-duality}.
\end{proof}


\section{Known exponent pairs}


\begin{proposition}[Classical van der Corput exponent pairs]\label{vdc-class}\uses{exp-pair-def} For any natural number $k \geq 2$,
    $$ A^{k-2} B(0,1) = \left( \frac{1}{2^k-2}, 1 - \frac{k-1}{2^k-2} \right)$$
    is an exponent pair.  In particular,
    $$ \left(\frac{1}{2}, \frac{1}{2}\right), \left(\frac{1}{6}, \frac{2}{3}\right), \left(\frac{1}{14}, \frac{11}{14}\right)$$
    are exponent pairs.
    \end{proposition}

    \begin{proof}\uses{vdc-a, exp-pair-trivial, vdc-opt, beta-duality} Follows by induction from Proposition \ref{vdc-a} and Proposition \ref{exp-pair-trivial}; alternatively, follows from (and is equivalent to) Corollary \ref{vdc-opt} and Lemma \ref{beta-duality}.
    \end{proof}

\derived
\code{van_der_corput_pair()}

\begin{theorem}[Exponent pairs on the line of symmetry]\label{line-sym}\uses{exp-pair-def} $(k,k+1/2)$ is an exponent pair for
\begin{itemize}
\item[(i)] $k = 9/56$ \cite[Theorem~1]{huxley_exponential_1988};
\item[(ii)] $k=89/560$ \cite[Theorem~6]{watt_exponential_1989};
\item[(iii)] $k=17/108$ \cite[p. 467]{huxley_exponential_1991};
\item[(iv)] $k=89/570$ \cite[p. 40]{huxley_exponential_1993};
\item[(v)] $k=32/205$ \cite[Theorem~1]{huxley_exponential_2005};
\item[(vi)] $k=13/84$ \cite[p. 206]{bourgain_decoupling_2017}.
\end{itemize}
\end{theorem}

\literature
\code{add_literature_exponent_pairs()}


\begin{theorem}[Exponent pairs from the Bombieri--Iwaniec method]\label{exp_pair_bombieri-iwaniec}\uses{exp-pair-def}  The following pairs are exponent pairs:
\begin{itemize}
\item[(i)] $(\frac{2}{13}, \frac{35}{52})$ \cite{huxley_watt_exponential_1990};
\item[(ii)] $(\frac{6299}{43860}, \frac{29507}{43860})$ \cite[Table 17.3]{huxley_area_1996};
\item[(iii)] $(\frac{771}{8116}, \frac{1499}{2029})$ \cite[p. 285]{sargos_points_1995};
\item[(iv)] $(\frac{21}{232}, \frac{173}{232})$ \cite[p. 286]{sargos_points_1995};
\item[(v)] $(\frac{1959}{21656}, \frac{16135}{21656})$ \cite[p. 286]{sargos_points_1995};
\item[(vi)] $(\frac{516247}{6629696}, \frac{5080955}{6629696})$ \cite{huxley_exponential_2001}, \cite[Table 19.2]{huxley_area_1996}, \cite{robert_fourth_2002}.
\end{itemize}
\end{theorem}

\literature
\code{add_literature_exponent_pairs()}

\begin{theorem}[Exponent pairs from derivative tests]\label{exp_pair_deriv_test}\uses{exp-pair-def} $(k,1-mk)$ is an exponent pair when
    \begin{itemize}
\item[(i)] $k=\frac{1}{13}$ and $m=3$ \cite[Theorem 1]{robert_fourth_2002};
\item[(ii)] $k = \frac{1}{204}$ and $m=7$ \cite[p. 231]{sargos_analog_2003};
\item[(iii)] $k = \frac{1}{130}$ and $m=8$ \cite[(1.1)]{robert_2002};
\item[(iv)] $k = \frac{7}{2640}$ and $m=8$ \cite[p. 231]{sargos_analog_2003};
\item[(v)] $k = \frac{1}{716}$ and $m=9$ \cite[p. 231]{sargos_analog_2003};
\item[(vi)] $k = \frac{1}{649}$ and $m=9$ \cite{Robert_Sargos_2001};
\item[(vii)] $k = \frac{7}{4540}$ and $m=9$ \cite[(1.2)]{robert_2002};
\item[(viii)] $k = \frac{1}{615}$ and $m=9$ \cite[(1.1)]{robert_2002};
\item[(ix)] $k = \frac{1}{915}$ and $m=10$ \cite[Th\'eor\`eme 2]{robert_2002b}.
\end{itemize}
\end{theorem}

\literature
\code{add_literature_exponent_pairs()}


\begin{theorem}[Huxley sequence]\label{huxley_exp_pair}\cite[Table 17.3]{huxley_area_1996}\uses{exp-pair-def}  For any integer $m \geq 1$, the pair
    $$ \left(\frac{169}{1424 \times 2^m - 338}, 1 - \frac{169}{1424 \times 2^m - 338} \frac{712m+1577}{712}\right)$$
is an exponent pair.
\end{theorem}

\literature
\code{add_huxley_exponent_pairs(Constants.EXP_PAIR_TRUNCATION)}

\begin{theorem}[1996 Heath--Brown sequence]\label{heath-brown_exp_pair_1996}\cite[(6.17.4)]{titchmarsh_theory_1986}\uses{exp-pair-def}  For any integer $m \geq 3$, the pair
$$ \left(\frac{1}{25m^2 (m-2) \log m}, 1 - \frac{1}{25 m^2 (m-2) \log m}\right)$$
is an exponent pair.
\end{theorem}

(Currently not implemented in python due to the irrational exponents.)

\begin{theorem}[2017 Heath--Brown sequence]\label{heath-brown_exp_pair_2017}\cite[Theorem 2]{heathbrown_new_2017}\uses{exp-pair-def}
For any integer $m \geq 3$, the pair
$$ \left(\frac{2}{(m-1)^2(m+2)}, 1 - \frac{3m-2}{m(m-1)(m+2)}\right)$$
is an exponent pair.
\end{theorem}

\literature
\code{add_heath_brown_exponent_pairs(Constants.EXP_PAIR_TRUNCATION)}

\begin{proof} This follows from Theorem \ref{beta-HB} and Lemma \ref{beta-duality}, after some computation.
\end{proof}


\begin{theorem}[Sargos $C$-process]\label{sargos_C}\cite[Theorem 5]{sargos_analog_2003}\uses{exp-pair-def}  If $(k,\ell)$ is an exponent pair, then so is
    $$ \left(\frac{k}{12(1+4k)}, \frac{11(1+4k)+\ell}{12(1+4k)}\right).$$
\end{theorem}

\literature
\code{C_transform()}

{\bf TODO: replicate the exponent pair theorem from \cite{trudgian-yang}}

{\bf TODO: implement our new exponent pair $(1101653/15854002, 12327829,15854002)$}
