\chapter{The generalized Dirichlet divisor problem}

\begin{definition}[Divisor sum exponents]\label{divisor-def} Let $k \geq 1$ be a fixed integer. Then, $\alpha_k$ is the best (fixed) exponent for which one has the asymptotic
$$ \sum_{n \leq x} d_k(n) = x P_{k-1}(\log x) + O(x^{\alpha_k+o(1)})$$
for unbounded $x > 0$, where $P$ is an explicit polynomial of degree $k-1$ and $d_k(n) := \sum_{n_1 \dots n_k=n} 1$ is the $k^{\mathrm{th}}$ divisor function. The implied constant may depend on $k$.
\end{definition}


In terms of ``epsilons and deltas'', $\alpha_k$ is the least exponent such that for all $\eps > 0$ there exists $C > 0$ such that
$$ |\sum_{n \leq x} d_k(n) - x P_{k-1}(\log x)| \leq C x^{\alpha_k + \eps}$$
for all $x \geq C$.

In the case $k = 1$, the problem is trivial. In particular:

\begin{lemma}[$d_1$ exponent]\label{divisor-1}\uses{divisor-def} One has $\alpha_1=0$.
\end{lemma}
\begin{proof}
Follows from $\sum_{n \le x}1 = x + O(1)$.
\end{proof}

On the other hand, the value of $\alpha_k$ is an open problem for all $k \ge 2$. Unconditionally, the following lower-bound on $\alpha_k$ is known to hold.

\begin{lemma}[Lower bound on $\alpha_k$]\label{divisor-lower}\uses{divisor-def}
For all $k \geq 1$, one has
\[
\alpha_k \geq \frac{1}{2} - \frac{1}{2k}.
\]
\end{lemma}
\begin{proof} See Hardy \cite{hardy_divisor_1916}.
\end{proof}

Multiple authors {\bf [give cite]} have refined Hardy's result but only in the $o(1)$ term in the exponent. It is conjectured that this lower bound on $\alpha_k$ is in fact an equality \cite[p.\ 320]{titchmarsh_theory_1986}. Amongst other consequences, this conjecture implies the Lindel\"of hypothesis \cite[Chapter XII]{titchmarsh_theory_1986}.
\begin{conjecture}[Generalised Dirichlet divisor problem conjecture]
For all $k \geq 1$, one has
\[
\alpha_k = \frac{1}{2} - \frac{1}{2k}.
\]
\end{conjecture}

The remainder of this chapter focuses on upper bounds on $\alpha_k$. The following series of lemmas record the sharpest known estimates.
\begin{theorem}\label{divisor-2-bound}\cite[Theorem 1.2]{li_yang_gauss_2024}\uses{divisor-def} One has $\alpha_2 \leq \alpha^* = 0.314483\ldots$, where $\alpha^*$ is the solution to the equation
\[
\frac{8}{25}\alpha - \frac{(\sqrt{2(1+14\alpha)} - 5\sqrt{-1+8\alpha})^2}{200} + \frac{51}{200} = \alpha
\]
on the interval $\alpha \in [0.3, 0.35]$.
\end{theorem}

\begin{theorem}\label{divisor-kolesnik}\cite{kolesnik}\uses{divisor-def} One has $\alpha_3 \leq 43/96$.
\end{theorem}

For larger $k$, estimates typically make use of the following relationship with zeta-moments.

\begin{lemma}\label{mas}\uses{divisor-def,zeta-moment-def}  Let $k \geq 2$ be an integer. If $M(\sigma,k) = 1$ then $\alpha_k \leq \sigma$.
\end{lemma}

\begin{proof}  See \cite[\S 13.3]{ivic}.
\end{proof}

Using Lemma \ref{mas}, the following bounds were obtained:

\begin{theorem}\cite{heathbrown_mean_1981} For $4 \le k \le 8$, one has
\[
\alpha_k \leq \frac{3k-4}{4k}.
\]
\end{theorem}

\begin{theorem}\cite{ivic_ouellet_1989} One has
\begin{align*}
\alpha_{10} \le 27/40,\qquad \alpha_{11} &\le 0.6957,\qquad \alpha_{12} \le 0.7130,\qquad \alpha_{13} \le 0.7306,\\
\alpha_{14} \le 0.7461,\qquad \alpha_{15} &\le 0.75851,\qquad \alpha_{16} \le 0.7691,\qquad \alpha_{17} \le 0.7785,\\
\alpha_{18} \le 0.7868,\qquad \alpha_{19} &\le 0.7942,\qquad \alpha_{20} \le 0.8009.
\end{align*}
\end{theorem}
\begin{theorem}\cite[Theorem 13.12]{ivic}\uses{mas}  One can bound $\alpha_k$ by
    \begin{align*}
        (3k-4)/4k & \hbox{ for } 4 \leq k \leq 8 \\
        35/54 & \hbox{ for } k = 9 \\
        41/60 & \hbox{ for } k = 10 \\
        7/10 & \hbox{ for } k = 11 \\
        (k-2)/(k+2) & \hbox{ for } 12 \leq k \leq 25 \\
        (k-1)/(k+4) & \hbox{ for } 26 \leq k \leq 50 \\
        (31k-98)/32k & \hbox{ for } 51 \leq k \leq 57 \\
        (7k-34)/7k & \hbox{ for } k \geq 58.
    \end{align*}
\end{theorem}

\begin{theorem}[\cite{bellotti_generalised_2023}]For integer $k \ge 30$, one has
\[
\alpha_k \leq 1 - 1.421(k - 1.18)^{-2/3}.
\]
Moreover, $\alpha_k \leq 1 - 1.889k^{-2/3}$ for sufficiently large $k$.
\end{theorem}
For completeness we record the historical progression in bounds for $\alpha_k$.
\begin{lemma}[Piltz bound]For $k \ge 1$, one has
\[
\alpha_k \le 1 - \frac{1}{k}.
\]
\end{lemma}
\begin{lemma}[Voronoi, Landau bound]
For $k \ge 2$, one has
\[
\alpha_k \leq 1 - \frac{2}{k + 1}.
\]
\end{lemma}
\begin{proof}
See Voronoi \cite{voronoi_sur_1903} for $k = 2$ and Landau \cite{landau_uber_1912} for $k \ge 3$.
\end{proof}

\begin{lemma}[Hardy--Littlewood bound]
For $k \ge 4$, one has
\[
\alpha_k \leq 1 - \frac{3}{k + 2}.
\]
\end{lemma}
\begin{proof}
See \cite{hardy_littlewood_approximate_1923}. The original proof relied on the assumption that $\mu(1/2) \le 1/6$ which was published later.
\end{proof}

\begin{lemma}[van der Corput first bound on $\alpha_2$]
One has $\alpha_2 \le 33/100$.
\end{lemma}
\begin{proof}
See van der Corput \cite{van_der_corput_verscharfung_1922}.
\end{proof}


\begin{lemma}[van der Corput second bound on $\alpha_2$]
One has $\alpha_2 \le 27/82$.
\end{lemma}
\begin{proof}
See van der Corput \cite{van_der_corput_zum_1928}.
\end{proof}

\begin{lemma}[Walfisz bound on $\alpha_3$]
One has $\alpha_3 \le 43/87$.
\end{lemma}
\begin{proof}
See Walfisz \cite{walfisz_uber_1926}.
\end{proof}

\begin{lemma}[Atkinson bound on $\alpha_3$]
One has $\alpha_3 \le 37/75$.
\end{lemma}
\begin{proof}
See Atkinson \cite{atkinson_divisor_1941}.
\end{proof}


\begin{lemma}[Richert, Tsung bound]
One has $\alpha_2 \le 15/46$.
\end{lemma}
\begin{proof}
Proved independently by Chih \cite{chih_on_1950} and Richert \cite{richert_verschrfung_1953}.
\end{proof}

\begin{lemma}[Y\"{u}h bound on $\alpha_3$]
One has $\alpha_3 \le 14/29$.
\end{lemma}
\begin{proof}
See Y\"{u}h \cite{yuh_divisor_1958}.
\end{proof}

\begin{lemma}[Y\"{u}h--Wu bound on $\alpha_3$]
One has $\alpha_3 \le 8/17$.
\end{lemma}
\begin{proof}
See Y\"{u}h--Wu \cite{yuh_wu_divisor_1962}.
\end{proof}

\begin{lemma}[Chen bound on $\alpha_3$]
One has $\alpha_3 \le 5/11$.
\end{lemma}
\begin{proof}
See Chen \cite{chen_divisor_1965}.
\end{proof}

\begin{lemma}[Kolesnik first bound on $\alpha_2$]
One has $\alpha_2 \le 12/37$.
\end{lemma}
\begin{proof}
See Kolesnik \cite{kolesnik_improvement_1969}.
\end{proof}

\begin{lemma}[Kolesnik second bound on $\alpha_2$]
One has $\alpha_2 \le 346/1067$.
\end{lemma}
\begin{proof}
See Kolesnik \cite{kolesnik_1973}.
\end{proof}

\begin{lemma}[Kolesnik first bound on $\alpha_3$]
One has $\alpha_3 \le 43/96$.
\end{lemma}
\begin{proof}
See Kolesnik \cite{kolesnik}.
\end{proof}


\begin{lemma}[Kolesnik third bound on $\alpha_2$]
One has $\alpha_2 \le 35/108$.
\end{lemma}
\begin{proof}
See Kolesnik \cite{kolesnik_order_1982}.
\end{proof}


\begin{lemma}[Huxley bound]
One has $\alpha_2 \le 131/416$.
\end{lemma}
\begin{proof}
See Huxley \cite{huxley_exponential_2003}
{\bf TODO: link this up with exponential sum bounds?}
\end{proof}

\begin{lemma}[Heath-Brown bound for large $k$]
For any $k \ge 1$, one has
\[
\alpha_k \le 1 - 0.849k^{-2/3}.
\]
\end{lemma}
\begin{proof}
See Heath-Brown \cite{heathbrown_new_2017}.
\end{proof}
{\bf TODO: list known bounds on $\alpha_k$}
