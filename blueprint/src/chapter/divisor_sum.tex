\chapter{The generalized Dirichlet divisor problem}

\begin{definition}[Divisor sum exponents]\label{divisor-def} Let $k \geq 1$ be an integer. Then, $\alpha_k$ is the best exponent for which one has the asymptotic
$$ \sum_{n \leq x} d_k(n) = x P_{k-1}(\log x) + O(x^{\alpha_k+o(1)})$$
as $x \to \infty$, where $P$ is an explicit polynomial of degree $k-1$ and $d_k(n) := \sum_{n_1 \dots n_k=n} 1$ is the $k^{\mathrm{th}}$ divisor function. The implied constant may depend on $k$. 
\end{definition}

In the case $k = 1$, the problem is trivial. In particular:

\begin{lemma}[$d_1$ exponent]\label{divisor-1}\uses{divisor-def} One has $\alpha_1=0$.
\end{lemma}
\begin{proof}
Follows from $\sum_{n \le x}1 = x + O(1)$. 
\end{proof}

On the other hand, the value of $\alpha_k$ is an open problem for all $k \ge 2$. Unconditionally, the following lower-bound on $\alpha_k$ is known to hold. 

\begin{lemma}[Lower bound on $\alpha_k$]\label{divisor-lower}\uses{divisor-def}
For all $k \geq 1$, one has
\[
\alpha_k \geq \frac{1}{2} - \frac{1}{2k}.
\]
\end{lemma}
\begin{proof} See Hardy \cite{hardy_divisor_1916}.
\end{proof}

Multiple authors {\bf [give cite]} have refined Hardy's result but only in the $o(1)$ term in the exponent. It is conjectured that this lower bound on $\alpha_k$ is in fact an equality \cite[p.\ 320]{titchmarsh_theory_1986}. Amongst other consequences, this conjecture implies the Lindel\"of hypothesis \cite[Chapter XII]{titchmarsh_theory_1986}.  
\begin{conjecture}[Generalised Dirichlet divisor problem conjecture]
For all $k \geq 1$, one has
\[
\alpha_k = \frac{1}{2} - \frac{1}{2k}.
\]
\end{conjecture}

The remainder of this chapter focuses on upper bounds on $\alpha_k$. 

\section{Known bounds on $\alpha_2$}

Currently the sharpest known upper bound on $\alpha_2$ is:

\begin{theorem}\label{divisor-2-bound}\cite[Theorem 1.2]{li_yang_gauss_2024}\uses{divisor-def} One has $\alpha_2 \leq \alpha^* = 0.314483\ldots$, where $\alpha^*$ is the solution to the equation
\[
\frac{8}{25}\alpha - \frac{(\sqrt{2(1+14\alpha)} - 5\sqrt{-1+8\alpha})^2}{200} + \frac{51}{200} = \alpha
\]
on the interval $\alpha \in [0.3, 0.35]$.
\end{theorem}

\Cref{div-alpha2-table} records the historical progression of upper bounds on $\alpha_2$.

\begin{table}[ht]
    \def\arraystretch{1.2}
    \centering
    \caption{Historical bounds on $\alpha_2$}
    \begin{tabular}{|c|c|}
    \hline
    Reference & Upper bound on $\alpha_2$\\
    \hline
    Piltz & $1/2 = 0.5$\\
    \hline
    Voronoi (1903) \cite{voronoi_sur_1903} & $1/3 = 0.3333\ldots$\\
    \hline
    van der Corput (1922) \cite{van_der_corput_verscharfung_1922} & $33/100 = 0.33$\\
    \hline
    van der Corput (1928) \cite{van_der_corput_zum_1928} & $27/82 = 0.3292\ldots$\\
    \hline 
    Chih (1950) \cite{chih_on_1950}, Richert (1953) \cite{richert_verschrfung_1953} & $15/46 = 0.3260\ldots$\\
    \hline 
    Kolesnik (1969) \cite{kolesnik_improvement_1969} & $12/37 = 0.3243\ldots$\\
    \hline 
    Kolesnik (1973) \cite{kolesnik_1973} & $346/1067 = 0.3242\ldots$\\
    \hline 
    Kolesnik (1982) \cite{kolesnik_order_1982} & $35/108 = 0.3240\ldots$\\
    \hline 
    Huxley (2003) \cite{huxley_exponential_2003} & $131/416 = 0.3149\ldots$\\
    \hline 
    Li--Yang (2024) \cite{li_yang_gauss_2024} & $0.314483\ldots$\\
    \hline 
    \end{tabular}
\label{div-alpha2-table}
\end{table}


\section{Known bounds on $\alpha_3$}

Currently, the sharpest known bound on $\alpha_3$ is:

\begin{theorem}\label{divisor-kolesnik}\cite{kolesnik}\uses{divisor-def} One has $\alpha_3 \leq 43/96$.
\end{theorem}

\Cref{div-alpha3-table} records the historical progression of upper bounds on $\alpha_3$.

\begin{table}[ht]
    \def\arraystretch{1.2}
    \centering
    \caption{Historical bounds on $\alpha_3$}
    \begin{tabular}{|c|c|}
    \hline
    Reference & Upper bound on $\alpha_3$\\
    \hline
    Walfisz (1926) \cite{walfisz_uber_1926} & $43/87 = 0.4942\ldots$\\
    \hline
    Atkinson (1941) \cite{atkinson_divisor_1941} & $37/75 = 0.4933\ldots$\\
    \hline
    Rankin (1955) \cite{rankin_van_1955} & $0.4931466\ldots$\\
    \hline
    Y\"{u}h (1958) \cite{yuh_divisor_1958} & $14/29 = 0.4827\ldots$\\
    \hline 
    Yin (1959) \cite{} & $25/52 = 0.4807\ldots$\\
    \hline 
    Yin (1959) \cite{} & $10/21 = 0.4761\ldots$\\
    \hline
    Y\"{u}h--Wu (1962) \cite{yuh_wu_divisor_1962} & $8/17 = 0.4705\ldots$\\
    \hline 
    Yin (1964) \cite{} & $34/75 = 0.4533\ldots$\\
    \hline
    Chen (1965) \cite{chen_divisor_1965} & $5/11 = 0.4545\ldots$\\
    \hline
    Yin--Li (1981) \cite{}, Zheng (1988) \cite{} & $127/282 = 0.4503\ldots$\\
    \hline
    Kolesnik (1981) \cite{kolesnik} & $43/96 = 0.4479\ldots$\\
    \hline
    \end{tabular}
\label{div-alpha3-table}
\end{table}






\section{Known bounds on $\alpha_k$ for large $k$}
For larger $k$, estimates typically make use of the following relationship with zeta-moments. 

\begin{lemma}\label{mas}\uses{divisor-def,zeta-moment-def}  Let $k \geq 2$ be an integer. If $M(\sigma,k) = 1$ then $\alpha_k \leq \sigma$.
\end{lemma}

\begin{proof}  See \cite[\S 13.3]{ivic}.
\end{proof}

For completeness we record the historical progression in bounds for $\alpha_k$.  
\begin{lemma}[Piltz bound]For $k \ge 2$, one has
\[
\alpha_k \le 1 - \frac{1}{k}.
\]
\end{lemma}
\begin{lemma}[Voronoi, Landau bound]
For $k \ge 2$, one has
\[
\alpha_k \leq 1 - \frac{2}{k + 1}.
\]
\end{lemma}
\begin{proof}
See Voronoi \cite{voronoi_sur_1903} for $k = 2$ and Landau \cite{landau_uber_1912} for $k \ge 3$. 
\end{proof}

\begin{lemma}[Hardy--Littlewood bound for $k \ge 4$]
For $k \ge 4$, one has
\[
\alpha_k \leq 1 - \frac{3}{k + 2}.
\]
\end{lemma}
\begin{proof}
See \cite{hardy_littlewood_approximate_1923}. The original proof relied on the assumption that $\mu(1/2) \le 1/6$ which was published later. 
\end{proof}

\begin{lemma}[Tong bound for $4 \le k \le 11$]
One has
\begin{alignat*}{4}
\alpha_4 &\le 1/2,\qquad &&\alpha_5 \le 4/7,\qquad &&\alpha_6 \le 5/8,\qquad &&\alpha_7 \le 71/107\\
\alpha_8 &\le 41/59,\qquad &&\alpha_9 \le 31/43,\qquad &&\alpha_{10} \le 26/35,\qquad &&\alpha_{11}\le 19/25
\end{alignat*}
\end{lemma}
\begin{proof}
See Tong \cite{}.
\end{proof}

\begin{theorem}\cite{heathbrown_mean_1981} For $4 \le k \le 8$, one has 
\[
\alpha_k \leq \frac{3k-4}{4k}.
\]
\end{theorem}

\begin{theorem}[Ivi\'c--Ouellet bound for large $k$]\cite{ivic_ouellet_1989} One has
\begin{align*}
\alpha_{10} \le 27/40,\qquad \alpha_{11} &\le 0.6957,\qquad \alpha_{12} \le 0.7130,\qquad \alpha_{13} \le 0.7306,\\
\alpha_{14} \le 0.7461,\qquad \alpha_{15} &\le 0.75851,\qquad \alpha_{16} \le 0.7691,\qquad \alpha_{17} \le 0.7785,\\
\alpha_{18} \le 0.7868,\qquad \alpha_{19} &\le 0.7942,\qquad \alpha_{20} \le 0.8009.
\end{align*}
\end{theorem}
\begin{theorem}\cite[Theorem 13.12]{ivic}\uses{mas}  One can bound $\alpha_k$ by
    \begin{align*}
        (3k-4)/4k & \hbox{ for } 4 \leq k \leq 8 \\
        35/54 & \hbox{ for } k = 9 \\
        41/60 & \hbox{ for } k = 10 \\
        7/10 & \hbox{ for } k = 11 \\
        (k-2)/(k+2) & \hbox{ for } 12 \leq k \leq 25 \\
        (k-1)/(k+4) & \hbox{ for } 26 \leq k \leq 50 \\
        (31k-98)/32k & \hbox{ for } 51 \leq k \leq 57 \\
        (7k-34)/7k & \hbox{ for } k \geq 58.
    \end{align*}
\end{theorem}

\begin{lemma}[Heath-Brown bound for large $k$]
For any $k \ge 2$, one has
\[
\alpha_k \le 1 - 0.849k^{-2/3}.
\]
\end{lemma}
\begin{proof}
See Heath-Brown \cite{heathbrown_new_2017}.
\end{proof}

\begin{theorem}[\cite{bellotti_generalised_2023}]For integer $k \ge 30$, one has
\[
\alpha_k \leq 1 - 1.421(k - 1.18)^{-2/3}.
\]
Moreover, $\alpha_k \leq 1 - 1.889k^{-2/3}$ for sufficiently large $k$.
\end{theorem}