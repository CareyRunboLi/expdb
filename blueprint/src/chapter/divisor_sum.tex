\chapter{The generalized Dirichlet divisor problem}

\begin{definition}[Divisor sum exponents]\label{divisor-def} Let $k \geq 1$. Then $\alpha_k$ is the best exponent for which one has the asymptotic
$$ \sum_{n \leq x} d_k(n) = x P_{k-1}(\log x) + O(x^{\alpha_k+o(1)})$$
as $x \to \infty$, where $P$ is an explicit polynomial of degree $k-1$ and $d_k(n) := \sum_{n_1 \dots n_k=n} 1$ is the $k^{\mathrm{th}}$ divisor function.
\end{definition}

\begin{lemma}[$d_1$ exponent]\label{divisor-1}\uses{divisor-def} One has $\alpha_1=0$.
\end{lemma}

\begin{lemma}\label{divisor-2-bound}\cite[Theorem 13.1]{ivic}\uses{divisor-def} One has $\alpha_2 \leq 35/108$.
\end{lemma}

\begin{lemma}\label{divisor-kolesnik}\cite{kolesnik}\uses{divisor-def} One has $\alpha_3 \leq 43/96$.
\end{lemma}


\begin{lemma}[Lower bound]\label{divisor-lower}\uses{divisor-def}One has $\alpha_k \geq \frac{1}{2} - \frac{1}{2k}$ for all $k \geq 1$. 
\end{lemma}

\begin{proof} See Hardy \cite{hardy_divisor_1916}.
\end{proof}

It is conjectured that this lower bound is in fact an equality.

\begin{lemma}\label{mas}\uses{divisor-def,zeta-moment-def}  Let $k \geq 2$ be an integer. If $M(\sigma,k) = 1$ then $\alpha_k \leq \sigma$.
\end{lemma}

\begin{proof}  See \cite[\S 13.3]{ivic}.
\end{proof}

Using Lemma \ref{mas}, the following bounds were obtained:

\begin{theorem}\cite[Theorem 13.12]{ivic}\uses{mas}  One can bound $\alpha_k$ by
    \begin{align*}
        (3k-4)/4k & \hbox{ for } 4 \leq k \leq 8 \\
        35/54 & \hbox{ for } k = 9 \\
        41/60 & \hbox{ for } k = 10 \\
        7/10 & \hbox{ for } k = 11 \\
        (k-2)/(k+2) & \hbox{ for } 12 \leq k \leq 25 \\
        (k-1)/(k+4) & \hbox{ for } 26 \leq k \leq 50 \\
        (31k-98)/32k & \hbox{ for } 51 \leq k \leq 57 \\
        (7k-34)/7k & \hbox{ for } k \geq 58.
    \end{align*}
\end{theorem}

For completeness we record the historical progression in bounds for $\alpha_k$.  
\begin{lemma}[Piltz bound]For $k \ge 1$, one has
\[
\alpha_k \le 1 - \frac{1}{k}.
\]
\end{lemma}
\begin{lemma}[Voronoi, Landau bound]
For $k \ge 2$, one has
\[
\alpha_k \leq 1 - \frac{2}{k + 1}.
\]
\end{lemma}
\begin{proof}
See Voronoi \cite{voronoi_sur_1903} for $k = 2$ and Landau \cite{landau_uber_1912} for $k \ge 3$. 
\end{proof}

\begin{lemma}[Hardy--Littlewood bound]
For $k \ge 4$, one has
\[
\alpha_k \leq 1 - \frac{3}{k + 2}.
\]
\end{lemma}
\begin{proof}
See \cite{hardy_littlewood_approximate_1923}. The original proof relied on the assumption that $\mu(1/2) \le 1/6$ which was published later. 
\end{proof}

\begin{lemma}[van der Corput bound 1]
One has $\alpha_2 \le 33/100$. 
\end{lemma}
\begin{proof}
See van der Corput \cite{van_der_corput_verscharfung_1922}.
\end{proof}


\begin{lemma}[van der Corput bound 2]
One has $\alpha_2 \le 27/82$. 
\end{lemma}
\begin{proof}
See van der Corput \cite{van_der_corput_zum_1928}.
\end{proof}

\begin{lemma}[Walfisz bound]
One has $\alpha_3 \le 43/87$.
\end{lemma}
\begin{proof}
See Walfisz \cite{walfisz_uber_1926}.
\end{proof}

\begin{lemma}[Atkinson bound]
One has $\alpha_3 \le 37/75$.
\end{lemma}
\begin{proof}
See Atkinson \cite{atkinson_divisor_1941}.
\end{proof}


\begin{lemma}[Richert, Tsung bound]
One has $\alpha_2 \le 15/46$. 
\end{lemma}
\begin{proof}
Proved independently by Chih \cite{chih_on_1950} and Richert \cite{richert_verschrfung_1953}.
\end{proof}

\begin{lemma}[Y\"{u}h bound on $\alpha_3$]
One has $\alpha_3 \le 14/29$. 
\end{lemma}
\begin{proof}
See Y\"{u}h \cite{yuh_divisor_1958}.
\end{proof}

\begin{lemma}[Kolesnik bound 1]
One has $\alpha_2 \le 12/37$. 
\end{lemma}
\begin{proof}
See Kolesnik \cite{kolesnik_improvement_1969}.
\end{proof}

\begin{lemma}[Kolesnik bound 2]
One has $\alpha_2 \le 346/1067$. 
\end{lemma}
\begin{proof}
See Kolesnik \cite{kolesnik_1973}.
\end{proof}

\begin{lemma}[Kolesnik bound 3]
One has $\alpha_2 \le 35/108$. 
\end{lemma}
\begin{proof}
See Kolesnik \cite{kolesnik_order_1982}.
\end{proof}


\begin{lemma}[Huxley bound]
On has $\alpha_2 \le 131/416$.
\end{lemma}
\begin{proof}
See Huxley \cite{huxley_exponential_2003}
{\bf TODO: link this up with exponential sum bounds?}
\end{proof}
{\bf TODO: list known bounds on $\alpha_k$}
