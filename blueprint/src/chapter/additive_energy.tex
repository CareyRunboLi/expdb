\chapter{Large value additive energy}

\section{Additive energy}

\begin{definition}[Additive energy]\label{energy-def}  Let $W$ be a finite set of real numbers.  The \emph{additive energy} $E_1(W)$ of such a set is defined to be the number of quadruples $(t_1,t_2,t_3,t_4) \in W$ such that
$$
|t_1 + t_2 - t_3 - t_4| \leq 1.$$
\end{definition}

We remark that in additive combinatorics, the variant $E_0(W)$ of the additive energy is often studied, in which $t_1+t_2-t_3-t_4$ is not merely required to be $1$-bounded, but in fact vanish exactly.  However, this version of additive energy is less relevant for analytic number theory applications.

\begin{lemma}[Basic properties of additive energy]\label{add-energy}\uses{energy-def}
\begin{itemize}
\item[(i)] If $W$ is a finite set of reals, then
$$ E_1(W) \asymp \int_\R |\# \{ (t_1,t_2) \in W: |t_1+t_2 - x| \leq 1\} |^2\ dx.$$
More generally, for any $r>0$ we have
$$ E_1(W) \asymp r^{O(1)} \int_\R |\# \{ (t_1,t_2) \in W: |t_1+t_2 - x| \leq r\} |^2\ dx.$$
\item[(ii)] If $W$ is a finite set of reals, then
$$ E_1(W) \asymp \int_{-1}^1 |\sum_{t \in W} e(t\theta)|^4\ d\theta.$$
\item[(iii)] If $W_1,\dots,W_k$ are finite sets of reals, then
$$ E_1(W_1 \cup \dots \cup W_k)^{1/4} \ll E_1(W_1)^{1/4} + \dots + E_1(W_k)^{1/4}.$$
\item[(iv)]  If $W$ is $1$-separated and contained in an interval of length $T \geq 1$, then
$$ (\# W)^2, (\# W)^4/T \ll E_1(W) \ll (\# W)^3.$$
\item[(v)]  If $W$ is contained in an interval $I$, which is in turn split into $K$ equally sized subintervals $J_1,\dots,J_K$, then
$$ E_1(W)^{1/3} \ll \sum_{k=1}^K E_1(W \cap J_k)^{1/3}.$$
\end{itemize}
\end{lemma}

Note that the lower bound of $(\# W)^4 / T$ would be expected to be attained if the set $W$ is distributed ``randomly'' and is reasonably large (of size $\gg \sqrt{T}$).  So  getting upper bounds of the additive energy of similar strength to this lower bound can be viewed as a statement of ``pseudorandomness'' (or ``Gowers uniformity'') of this set.

\begin{proof} For (i), we just prove the first estimate, as the second follows from the first by several applications of the triangle inequality.  The right-hand side can be expanded as
    $$ \sum_{t_1,t_2,t_3,t_4 \in W} |\{ x: |t_1+t_2-x|, |t_3+t_4-x| \leq 1 \}|.$$
Every quadruple contributing to $E_1(W)$ then contributes $\gg 1$ to the right-hand side, giving the upper bound.  To get the matching lower bound, note that
$$ \sum_{t_1,t_2,t_3,t_4 \in W} |\{ x: |t_1+t_2-x|, |t_3+t_4-x| \leq 1/2 \} \leq E_1(W)$$
and hence
$$ E_1(W) \gg \int_\R |\# \{ (t_1,t_2) \in W: |t_1+t_2 - x| \leq 1/2\} |^2\ dx.$$
The upper bound then follows from the triangle inequality.

For (ii), we can upper bound the indicator function of $[-1,1]$ by the Fourier transform of a non-negative bump function $\varphi$, so that the right-hand side is bounded by
$$ \sum_{t_1,t_2,t_3,t_4 \in W} \varphi(t_1+t_2-t_3-t_4)$$
which is then bounded by $O(E_1(W))$ by choosing the support of $\varphi$ appropriately.  The lower bound is established similarly (using the arguments in (i) to adjust the error tolerance $1$ in the constraint $ |t_1+t_2 - x| \leq 1$ as necessary.)

For (iii), first observe we may remove duplicates and assume that the $W_i$ are disjoint, then we can use (ii) and the triangle inequality.

For (iv), the first lower bound comes from considering the diagonal case $t_1=t_3, t_2 = t_4$ and the upper bound comes from observing that once $t_1,t_2,t_3$ are fixed, there are only $O(1)$ choices for $t_4$ thanks to the $1$-separated hypothesis.  Finally, observe that
$$ \int_\R |\# \{ (t_1,t_2) \in W: |t_1+t_2 - x| \leq 1\} |\ dx = (2 \# W)^2$$
hence by Cauchy--Schwarz
$$ \int_\R |\# \{ (t_1,t_2) \in W: |t_1+t_2 - x| \leq 1\} |^2\ dx \gg (\# W)^2/T$$
and the claim follows from (i).

For (v), write $a_k \coloneqq E_1(W \cap J_k)^{1/4}$.  Each tuple $(t_1,t_2,t_3,t_4)$ that contributes to $E_1(W)$ is associated to a tuple $J_{k_1}, J_{k_2}, J_{k_3}, J_{k_4}$ of intervals with $k_1+k_2-k_3-k_4=O(1)$.  By modifying the proof of (ii), the total contribution of such a tuple of intervals is
$$ \ll \int_\R \prod_{j=1}^4 |\sum_{t \in W \cap J_{k_j}} e(t\theta)|\ d\theta$$
which by Cauchy--Schwarz is bounded by
$$ \ll a_{k_1} a_{k_2} a_{k_3} a_{k_4}.$$
Thus we see that
$$ E_1(W) \ll \sum_{m=O(1)} a * a * \tilde a * \tilde a(m)$$
where $\tilde a_k \coloneqq a_{-k}$ and $*$ denotes convolution on the integers.  By Young's inequality we then have
$$ E_1(W) \ll \|a\|_{\ell^{4/3}}^4$$
and the claim follows.

We remark that (v) can also be proven using \cite[Lemma 4.8, (4.2)]{cladek-tao}.
\end{proof}

We will also study the following related quantity.  Given a set $W$ and a scale $N>1$, let $S(N,W)$ denote the \emph{double zeta sum}
$$ S(N,W) \coloneqq \sum_{t,t' \in W} |\sum_{n \in [N,2N]} n^{-i(t-t')}|^2.$$
We caution that this normalization differs from the one in \cite{ivic}, where $n^{-1/2-i(t-t')}$ is used in place of $n^{-i(t-t')}$.  This sum may also be rearranged as
$$ S(N,W) = \sum_{n,m \in [N,2N]} |R_W(n/m)|^2$$
where $R_W$ is the exponential sum
$$ R_W(x) \coloneqq \sum_{t \in W} x^{it}.$$
From the first formula it is clear that $S(N,W)$ is monotone non-decreasing in $W$, and from the second formula one has the triangle inequality
$$ S(N, \bigcup_{i=1}^k W_i)^{1/2} \leq \sum_{i=1}^k S(N,W_i)^{1/2}$$
when the $W_i$ are disjoint, and hence also when they are not assumed to be disjoint, thanks to the monotonicity.

To relate $S(N,W)$ to $E_1(W)$, we first observe the following lemma, implicit in \cite{heath_brown_consecutive_II} and made more explicit in \cite[Lemma 11.4]{guth-maynard}.

\begin{lemma}[Energy controlled by third moment]\label{energy-third}\uses{energy-def} Let $T \geq 1$. If $a_n$ is a $1$-bounded sequence on $[N,2N]$ for some $1 \leq N \ll T^{O(1)}$, $W$ is $1$-separated in $[-T,T]$, and
$$|\sum_{n \in [N,2N]} a_n n^{-it}| \geq V$$
for all $t \in W$ and some $V>0$, then
$$ V^2 E_1(W) \ll T^{o(1)} \sum_{n,m \in [N,2N]} |R_W(n/m)|^3 + T^{-50}.$$
\end{lemma}

\begin{proof} By hypothesis, we have
$$ V^2 E_1(W) \leq \sum_{t_1,t_2,t_3,t_4 \in W: |t_1+t_2-t_3-t_4| \leq 1} |\sum_{n \in [N,2N]} a_n n^{-it_4}|^2.$$
By standard Fourier arguments (see \cite[Lemma 11.3]{guth-maynard}), we can bound
$$ |\sum_{n \in [N,2N]} a_n n^{-it_4}| \ll T^{o(1)} \int_{t: |t-t_4| \leq T^{o(1)}} |\sum_{n \in [N,2N]} a_n n^{-it}|\ dt + T^{-100}.$$
Since each $t_1,t_2,t_3$ generates at most $O(1)$ choices for $t_4$, we conclude that
$$ V^2 E_1(W) \ll T^{o(1)} \sum_{t_1,t_2,t_3 \in W} \int_{s: |s| \leq T^{o(1)}}  |\sum_{n \in [N,2N]} a_n n^{-i(t_1+t_2-t_3+s)}|^2\ ds + T^{-50},$$
The right-hand side can be rewritten as
$$ T^{o(1)} \sum_{n,m \in [N,2N]} a_n \overline{a_m} (n/m)^{-is} \overline{R_W}(n/m)^2 R_W(n/m) + T^{-50},$$
and the claim then follows from the triangle inequality.
\end{proof}

Thus, $S(N,W)$ involves a second moment of $R_W$, while the energy $E_1(W)$ is related to the third moment.  Using the trivial bound $|R_W(x)| \leq |W|$ we can then obtain the trivial bound
\begin{equation}\label{energy-triv}
V^2 E_1(W) \ll T^{o(1)} |W| S(N,W) + T^{-50}
\end{equation}
It is then natural to introduce the fourth moment
$$ S_4(N,W) \coloneqq \sum_{n,m \in [N,2N]} |R_W(n/m)|^4$$
since from H\"older's inequality one now has
\begin{equation}\label{v1w}
    V^2 E_1(W) \ll T^{o(1)} S(N,W)^{1/2} S_4(N,W)^{1/2} + T^{-50}
\end{equation}
(cf. \cite[Lemma 3]{heath_brown_consecutive_II}).  The quantity $S_4(N,W)$ can also be expressed as
$$ S_4(N,W) = \sum_{t_1,t_2,t_3,t_4 \in W} |\sum_{n \in [N,2N]} n^{-i(t_1+t_2-t_3-t_4)}|^2.$$

One can bound this quantity by an $S(N,W)$ type expression:

\begin{lemma}\label{wtu}\uses{energy-def} If $W \subset [-T,T]$ is $1$-separated and $1 \leq N \ll T^{O(1)}$, then one has
$$ S_4(N,W) \ll T^{o(1)} u^2 S(N,U) + T^{-100}$$
for some $1 \leq u \ll |W|$ and $1$-separated subset $U$ of $[-2T,2T]$ with
\begin{equation}\label{v1}
 u |U| \ll |W|^2
\end{equation}
and
\begin{equation}\label{v2}
     u^2 |U| \ll E_1(W).
\end{equation}
\end{lemma}

This result appears implicitly in \cite[p. 229]{heath_brown_consecutive_II}, and is made more explicit in the proof of \cite[Lemma 11.6]{guth-maynard}.

\begin{proof} One can bound
    $$ S_4(N,W) \ll T^{o(1)} \sum_{t_1,t_2,t_3,t_4 \in W} \int_{t = t_1+t_2-t_3-t_4+O(T^{o(1)})} |\sum_{n \in [N,2N]} n^{-it}|^2\ dt + T^{-100},$$
    and hence
    $$ S_4(N,W) \ll T^{o(1)} \sum_{t_1,t_2 \in [-2N,2N] \cap \Z} f(t_1) f(t_2) \int_{t = t_1-t_2+O(T^{o(1)})} |\sum_{n \in [N,2N]} n^{-it}|^2\ dt + T^{-100}$$
where $f$ is the counting function
$$ f(t) \coloneqq |\{ (t_1,t_2) \in W: |t-t_1-t_2| \leq 1 \}|.$$
Note that $f$ is integer valued and bounded above by $|W|$. By dyadic decomposition, one can then find $1 \leq u \ll |W|$ and a subset $U$ of $[-2N,2N] \cap \Z$ such that $f(t) \asymp u$ for $t \in U$, and
$$ S_4(N,W) \ll T^{o(1)} \sum_{t_1,t_2 \in U} u^2 \int_{t = t_1-t_2+O(T^{o(1)})} |\sum_{n \in [N,2N]} n^{-it}|^2\ dt + T^{-100}$$
which we can rearrange as
$$ S_4(N,W) \ll T^{o(1)} u^2 \int_{s = O(T^{o(1)})} \sum_{n,m \in [N,2M]} (n/m)^{is} |R_U(n/m)|^2\ ds + T^{-100}$$
and hence by the triangle inequality
$$ S_4(N,W) \ll T^{o(1)} v^2 S(N,V) + T^{-100}.$$
Also, by double counting one easily verifies the claims \eqref{v1}, \eqref{v2}.  The claim follows.
\end{proof}


\section{Large value additive energy region}

Because the cardinality $|W|$ and additive energy $E_1(W)$ of a set $W$ are correlated with each other, as well as with the double zeta sum $S(N,W)$, we will not be able to consider them separately, and instead we will need to consider the possible joint exponents for these two quantities.  We formalize this via the following set:

\begin{definition}[Large value energy region]\label{lv-edef} The \emph{large value energy region} $\Energy \subset \R^5$ is defined to be the set of all fixed tuples $(\sigma,\tau,\rho,\rho^*,s)$ with $1/2 \leq \sigma \leq 1$, $\tau, \rho, \rho' \geq 0$, such that there exists an unbounded $N > 1$, $T = N^{\tau+o(1)}$, $V = N^{\sigma+o(1)}$, a $1$-bounded sequence  $a_n$ on $[N,2N]$, and a  $1$-separated subset $W$ of cardinality $N^{\rho+o(1)}$ in an interval $J$ of length $T$ such that
    \begin{equation}\label{sig-large} \left|\sum_{n \in [N,2N]} a_n n^{-it} \right| \geq V
\end{equation}
for all $t \in W$, and such that $E_1(W) = N^{\rho^*+o(1)}$ and $S(N,W) = N^{s+o(1)}$.

We define the \emph{large value energy region for zeta} $\Energy_\zeta \subset \R^5$ similarly, but now the interval $J$ is required to be of the form $[T,2T]$, and the sequence $a_n$ is required to be of the form $1_I(n)$ for some interval $I \subset [N,2N]$.  Thus, in order for $(\sigma,\tau,\rho,\rho^*,s)$ to lie in $\Energy_\zeta$, there must exist an unbounded $N > 1$, $T = N^{\tau+o(1)}$, $V = N^{\sigma+o(1)}$, an interval $I$ in $[N,2N]$, and  $W = W$ is a $1$-separated subset of cardinality $N^{\rho+o(1)}$ in $[T,2T]$ such that
\begin{equation}\label{sig-large-zeta} \left|\sum_{n \in I} n^{-it} \right| \geq V
\end{equation}
for all $t \in W$, and such that $E_1(W) = N^{\rho^*+o(1)}$ and $S(N,W) = N^{s+o(1)}$.
\end{definition}

Clearly we have

\begin{lemma}[Trivial containment]\label{triv-contain}\uses{lv-edef} We have $\Energy_\zeta \subset \Energy$.
\end{lemma}

These region is related to $\LV(\sigma,\tau)$ and as follows:

\begin{lemma}\label{energy-region-lv}\uses{lv-def, lvz-def, energy-def} For any fixed $1/2 \leq \sigma \leq 1, \tau \geq 0$, we have
$$ \LV(\sigma,\tau) = \sup \{ \rho: (\sigma,\tau,\rho,\rho^*,s) \in \Energy\}$$
and
$$ \LV_\zeta(\sigma,\tau) = \sup \{ \rho: (\sigma,\tau,\rho,\rho^*,s) \in \Energy_\zeta\}$$
In particular, we have $\rho \leq \LV(\sigma,\tau)$ for all $(\sigma,\tau,\rho,\rho^*,s) \in \Energy$, and $\rho \leq \LV_\zeta(\sigma,\tau)$ for all $(\sigma,\tau,\rho,\rho^*,s) \in \Energy_\zeta$.
\end{lemma}

\begin{proof} Clear from definition.
\end{proof}

Inspired by this, we can define

\begin{definition}\label{lvze-def}\uses{energy-def}  For any fixed $1/2 \leq \sigma \leq 1, \tau \geq 0$, we define
$$ \LV^*(\sigma,\tau) \coloneqq \sup \{ \rho^*: (\sigma,\tau,\rho,\rho^*,s) \in \Energy\}$$
and
$$ \LV^*_\zeta(\sigma,\tau) \coloneqq \sup \{ \rho^*: (\sigma,\tau,\rho,\rho^*,s) \in \Energy_\zeta\}.$$
\end{definition}

Thus these exponents are upper bounds for the additive energy of large values of Dirichlet polynomials which may or may not be of zeta function type.

As usual, we have an equivalent non-asymptotic definition of the large value energy region:

\begin{lemma}[Non-asymptotic form of large value energy region]\label{lve-asymp}\uses{energy-def} Let $1/2 \leq \sigma \leq 1$, $\tau \geq 0$,  $\rho, \rho^* \geq 0$, and $s \in \R$ be fixed.  Then the following are equivalent:
    \begin{itemize}
    \item[(i)] $(\sigma,\tau,\rho,\rho^*) \in \Energy$.
    \item[(ii)] For every $\eps>0$ and $C > 0$, there exists $N \geq C$, $N^{\tau-\delta} \leq T \leq N^{\tau+\delta}$, $N^{\sigma-\delta} \leq V \leq N^{\sigma+\delta}$, a $1$-bounded $a_n$ for each $n \in [N,2N]$, and a $1$-separated subset $W$  of an interval $J$ of length $T$ such that \eqref{sig-large}
holds for all $t \in W$, with
    $$ N^{\rho-\eps} \leq |W| \leq N^{\rho+\eps},$$
    $$ N^{\rho^*-\eps} \leq E_1(W) \leq N^{\rho^*+\eps}$$
    $$ N^{s-\eps} \leq S(N, W) \leq N^{s+\eps}.$$
    \end{itemize}
    Similarly with $\Energy$ replaced by $\Energy_\zeta$, $J$ replaced by $[N,2N]$, $a_n$ replaced by an interval $I$ in $[N,2N]$, and \eqref{sig-large} replaced by \eqref{sig-large-zeta}.
\end{lemma}

This lemma is proven by a routine expansion of the definitions, and is omitted.

\begin{lemma}[Basic properties]\label{lve-basic}\
    \begin{itemize}
        \item[(i)] (Monotonicity in $\sigma$) If $(\sigma,\tau,\rho,\rho^*,s) \in \Energy$, then
        $(\sigma',\tau',\rho,\rho^*,s) \in \Energy$ for all $1/2 \leq \sigma' \leq \sigma$ and $\tau' \geq \tau$.
        \item[(ii)] (Subdivision) If $(\sigma,\tau,\rho,\rho^*,s) \in \Energy$ and $0 \leq \tau' \leq \tau$, then there exists $(\sigma,\tau', \rho', (\rho')^*,s') \in \Energy$ such that
        $$\rho' \leq \rho \leq \rho' + \tau - \tau'$$
        and
        $$(\rho')^* + (\rho-\rho') \leq \rho^* \leq \rho' + 3(\rho-\rho').$$
        and
        $$s' \leq s \leq s' + 2(\rho-\rho').$$
        \item[(iii)]  (Trivial bounds) If $(\sigma,\tau,\rho,\rho^*,s) \in \Energy$, one has
        $$ 2\rho, 4\rho-\tau \leq \rho^* \leq 3 \rho.$$
    \end{itemize}
\end{lemma}

\begin{proof}  ...
\end{proof}


\begin{lemma}[Raising to a power]\label{power-energy}  If $(\sigma,\tau,\rho,\rho^*,s) \in \Energy$, then $(\sigma,\tau/k, \rho/k, (\rho^*)/k,s/k) \in \Energy$ for any integer $k \geq 1$.
\end{lemma}

\begin{proof}
    ...
\end{proof}

\section{Known relations for the large value energy region}

\begin{theorem}[Reflection principle]\label{reflect}\cite[\S 11.5]{ivic}\uses{lv-edef} If $(\sigma,\tau,\rho,\rho^*,s) \in \Energy$ with $\sigma \geq 3/4$ and $\tau>1$, then for any integer $k \geq 1$, either $\rho \leq 2-2\sigma$, or there exists $0 < \alpha \leq k(\tau-1)$ and $(\sigma, \tau/\alpha, \rho/\alpha, \rho^*/\alpha, s'/\alpha) \in \Energy$ such that
$$ \rho \leq \min( 2-2\sigma, k(3-4\sigma)/2 + s' - 1).$$
\end{theorem}

\begin{proof} By definition, there exists an unbounded $N>1$, $T = N^{\tau+o(1)}$, a $1$-bounded sequence $a_n$ on $[N,2N]$, and a $1$-separated subset $W$ of an interval of length $T$ (which we can normalize without loss of generality to be $[0,T]$) such that
    $$ |\sum_{n \in [N,2N]} a_n n^{-it}| \geq N^{\sigma+o(1)}$$
for all $t \in W$, with $|W| = N^{\rho+o(1)}$, $E_1(W) = N^{\rho^*+o(1)}$ and $S(N,W) = N^{s+o(1)}$.  By \cite[(11.58)]{ivic}, one has
$$ |W|^2 \ll T^{o(1)} ( |W| N^{2-2\sigma} + N^{1-2\sigma} |W|^2 + N^{(3-4\sigma)/2} \int_{v = O(T^{o(1)})} \sum_{t,t' \in W} |\sum_{n \leq 4T/N} n^{-1/2+it-it'+iv}|\ dv ).$$
Since $\sigma>1/2$, the $N^{1-2\sigma} |W|^2$ term can be dropped.  Applying H\"older's inequality and dyadic pigeonholing as in \cite[(11.59)]{ivic}, we conclude that
$$ |W| \ll T^{o(1)} (N^{2-2\sigma} + N^{k(3-4\sigma)/2}  (\sum_{t,t' \in W} |\sum_{n \in [N',2N']} b_n n^{-1/2+it-it'+iv}|^2)^{1/2}$$
for some $v = O(T^{o(1)})$ and coefficients $b_n = O(T^{o(1)})$, and some $N' \ll (4T/N)^k$.  After passing to a subsequence if necessary, we may assume that $N' = N^{\alpha+o(1)}$ for some $0 \leq \alpha \leq k(\tau-1)$.  If $\alpha=0$ then the second term here is negligible compared to the first and we obtain $\rho \leq 2-2\sigma$, so suppose that $\alpha > 0$.  Using \cite[Lemma 11.1]{ivic} to eliminate the $b_n n^{-1/2+iv}$ coefficients, we conclude that
$$ |W| \ll T^{o(1)} (N^{2-2\sigma} + N^{k(3-4\sigma)/2-1} S(N', W).$$
By construction, we have $S(N',W) = (N')^{s'/\alpha+o(1)} = N^{s'+o(1)}$ for some tuple $(\sigma, \tau/\alpha, \rho/\alpha, \rho^*/\alpha, s'/\alpha) \in \Energy$.  The claim follows.
\end{proof}

Heuristically one expects $s \leq \max( \rho+1, 2\rho)+1$ (see \cite[(11.63)]{ivic}).  There is one easy case in which this is true:

\begin{lemma}\label{easy-double-zeta-bound}\uses{lv-edef}  If $(\sigma,\tau,\rho,\rho^*,s) \in \Energy$ with $\tau < 1$, then $s \leq \max(\rho+1, 2\rho)+1$.
\end{lemma}

\begin{proof} See the discussion after \cite[(11.63)]{ivic}.
\end{proof}

Another bound is

\begin{lemma}\label{double-zeta_from_exp_pair}\cite[Lemma 11.2]{ivic}\uses{lv-edef}  If $(k,\ell)$ is an exponent pair with $k>0$, and $(\sigma,\tau,\rho,\rho^*,s) \in \Energy$, then
$$ s \leq \max( \rho+1, 5\rho/3 + \tau/3, \frac{2+3k+4\ell}{1+2k+2\ell} \rho + \frac{k+\ell}{1+2k+2\ll} \tau) + 1.$$
\end{lemma}

Finally, we have the useful

\begin{lemma}[Heath-Brown bound on double sums]\label{hb-double}\uses{lv-edef}  If $(\sigma,\tau,\rho,\rho^*,s) \in \Energy$, then
    $$ s \leq \max( \rho+1, 2\rho, 5\rho/4+\tau/2) + 1.$$
\end{lemma}

Note that if $\tau \leq 3/2$, the $5\rho/4+\tau/2$ term is bounded by the convex combination $(3/4)(\rho+1)+(1/4)(2\rho)$ and may therefore be omitted.

\begin{proof} See \cite[Theorem 1]{heathbrown_large_1979} or \cite[Lemma 11.5]{ivic}.
\end{proof}

Lemma \ref{wtu} can be formulated in terms of the large value energy region as follows.

\begin{lemma}\label{wtu-alt}\uses{lv-edef}  If $(\sigma,\tau,\rho,\rho^*,s) \in \Energy$, then there exists $(\sigma,\tau,\rho',(\rho')^*,s') \in \Energy$ and $0 \leq \kappa \leq \rho$ such that
$$ \kappa + \rho' \leq 2 \rho$$
$$ 2\kappa + \rho' \leq \rho^*$$
and
$$ \rho^* + 2\sigma \leq \kappa + (s+s')/2.$$
\end{lemma}


\begin{proof}\uses{wtu}  By definition, there exists an unbounded $N \geq 1$, $T = N^{\tau+o(1)}$, $V = N^{\sigma+o(1)}$, $1$-bounded coefficients $a_n$ on $[N,2N]$, and a $1$-separated subset $W$ of an interval of length $T$ such that
$$ |\sum_{n \in [N,2N]} a_n n^{-it}| \geq V$$
for all $t \in W$, with $|W| = N^{\rho+o(1)}$, $|W^*| = N^{\rho^*+o(1)}$, and $S(N,W) = N^{s+o(1)}$, then from \eqref{v1w} we have
$$
    V^2 E_1(W) \ll T^{o(1)} S(N,W)^{1/2} S_4(N,W)^{1/2} + T^{-50}.
$$
By Lemma \ref{wtu}, there exists $1 \leq u \ll |W|$ and a $1$-separated subset $U$ of $[-2T,2T]$ such that
 such that
$$
    V^2 E_1(W) \ll T^{o(1)} u S(N,W)^{1/2} S(N,U)^{1/2} + T^{-50}
$$
with \eqref{v1}, \eqref{v2} holding.  Since $W$ is non-empty, $E_1(W) \geq 1$ and $V \geq N^{1/2} \geq 1$, so the $T^{-50}$ error here may be discarded.  Passing to a subsequence, we may assume that $u = N^{\kappa+o(1)}$ for some $0 \leq \kappa \leq \rho$, and that $|U| = N^{\rho'+o(1)}$ for some $\rho' \geq 0$.  Then we have $S_2(N,U) = s'$ for some
$(\sigma,\tau,\rho',(\rho')^*,s') \in \Energy$, and the claim follows.
\end{proof}

These bounds on the double zeta sums can be used to control additive energies:

\begin{theorem}[Heath-Brown relation]\label{hbt}\cite[(33)]{heathbrown_zero_1979}\uses{lv-edef} If $(\sigma,\tau,\rho,\rho^*,s) \in \Energy$, then one has
$$ \rho^* \leq 1-2\sigma + \max(\rho+1, 2\rho, 5\rho/4+\tau/2)/2 + \max(\rho^*+1, 4\rho, 3\rho^*/4+\rho+\tau/2)/2.$$
\end{theorem}

\begin{proof}\uses{wtu-alt} By Lemma \ref{wtu-alt} followed by  we have
$$ \rho^* + 2\sigma \leq \kappa + ( \max( \rho+1, 2\rho, 5\rho/4+\tau/2)+ \max( \rho'+1, 2\rho', 5\rho'/4+\tau/2) )/2 + 1$$
for some $0 \leq \kappa \leq \rho$ with
$$ \kappa + \rho' \leq 2\rho$$
$$ 2\kappa + \rho' \leq \rho^*$$
In particular,
$$ 2\kappa + 5\rho'/4 \leq 3\rho^*/4 + \rho$$
and the claim follows after moving the $\kappa$ inside the second maximum and performing some algebra.
\end{proof}

\begin{corollary}[Simplified Heath-Brown relation]\label{hb-energy-simp}  If $(\sigma,\tau,\rho,\rho^*,s) \in \Energy$ and $\tau \leq 3/2$, then
$$ \rho^* \leq \max(3 \rho + 1-2\sigma, \rho +4-4\sigma, 5\rho/2 + (3-4\sigma)/2).$$
\end{corollary}

This result essentially appears as \cite[Lemma 3]{heathbrown_zero_1979}.

\begin{proof}\uses{hbt} Apply the previous result.  For $\tau \leq 3/2$ we observe that $5\rho/4+\tau/2$ is less than $5\rho/4 + 3/4$, which is a convex combination of $\rho+1$ and $2\rho$.  Similarly $3\rho^*/4+\rho+\tau/2$ is less than $3\rho^*/4+\rho+3/4$, which is a convex combination of $\rho^*+1$ and $4\rho$. We conclude that
$$ \rho^* \leq  1-2\sigma + \max(\rho+1, 2\rho)/2 + \max(\rho^*+1, 4\rho)/2.$$
Thus $\rho^*$ is less than one of
$$ 1-2\sigma + (\rho+\rho^*+2)/2, 1-2\sigma + (5\rho+1)/2, 1-2\sigma + (2\rho+\rho^*+1)/2, 1-2\sigma + (6\rho)/2;$$
solving for $\rho^*$, we conclude
$$ \rho^* \leq \max( 4-4\sigma + \rho, (3-4\sigma)/2 + 5\rho/2, 3-4\sigma + 2\rho, 1-2\sigma + 3\rho).$$
But since $\sigma \geq 1/2$, $3-4\sigma + 2\rho$ is less than $5/2-3\sigma + 2\rho$, which is the mean of $4-4\sigma+\rho$ and $1-2\sigma+3\rho$. Thus
$$ \rho^* \leq \max( 4-4\sigma + \rho, (3-4\sigma)/2 + 5\rho/2, 1-2\sigma + 3\rho),$$
which gives the claim.
\end{proof}


\begin{lemma}[Second Heath-Brown relation]\label{hbt-2}\uses{lv-edef} If If $(\sigma,\tau,\rho,\rho^*,s) \in \Energy$ then
$$ \rho \leq \max( 2-2\sigma, \rho^*/4 + \max(\tau/4 + k(3-4\sigma)/4, k/4 + k(1-2\sigma)/2))$$
for any positive integer $k$.
\end{lemma}

\begin{proof} See \cite[Lemma 4]{heathbrown_zero_1979}.
\end{proof}

\begin{lemma}[Guth-Maynard relation]\label{gm-1}\uses{lv-edef}  If $(\sigma,\tau,\rho,\rho^*,s) \in \Energy$ then
$$ \rho \leq \max(2-2\sigma, 1-2\sigma + \max(S_1, S_2, S_3)/3)$$
where $S_1, S_2, S_3$ are real numbers with
$$ S_1 \leq -10,$$
$$ S_2 \leq \max(2+2\rho, \tau+1+(2-1/k) \rho, 2 + 2\rho + (\tau/2 - 3\rho/4)/k )$$
for any positive integer $k$ and
$$ S_3 \leq 2\tau + \rho/2 + \rho^*/2$$
and also
$$ S_3 \leq \max( 2\tau + 3\rho/2, \tau+1+\rho/2+\rho^*/2).$$
\end{lemma}

\begin{proof} This follows from \cite[Propositions 4.6, 5.1, 6.1, 8.1, 10.1, (5.5)]{guth-maynard}.
\end{proof}

\begin{lemma}[Second Guth-Maynard relation]\cite[Lemma 1.7]{guth-maynard}  If $(\sigma,\tau,\rho,\rho^*,s) \in \Energy$ then
$$ \rho^* \leq \rho + s - 2\sigma.$$
In particular, from Lemma \ref{hb-double} we see for $\tau \leq 3/2$ that
$$ \rho^* \leq \max(3\rho+1-2\sigma, 2\rho+2-2\sigma).$$
\end{lemma}

\begin{proof} By definition, we can find an unbounded $N>1$, $T = N^{\tau+o(1)}$, $V = N^{\sigma+o(1)}$, a $1$-bounded sequence $a_n$ on $[N,2N]$, and a $1$-separated subset $W$ of an interval of length $T$ such that
    $$ |\sum_{n \in [N,2N]} a_n n^{-it}| \geq V$$
    for all $t \in W$, with $|W| = N^{\rho+o(1)}$, $E_1(W) = N^{\rho^*+o(1)}$, and $S(N,W) = N^{s+o(1)}$. From \eqref{energy-triv} one has
$$    V^2 E_1(W) \ll T^{o(1)} |W| S(N,W) + T^{-50}.$$
Since $W$ is non-empty, $E_1(W) \geq 1$, and $V \gg 1$, so the $T^{-50}$ error can be discarded.  The claim then follows.
\end{proof}


\begin{lemma}[Third Guth-Maynard relation]\label{gm-3}\uses{lv-edef}  If $(\sigma,\tau,\rho,\rho^*) \in \Energy$ and $1 \leq \tau \leq 4/3$, then
    $$ \rho^* \leq \max(\rho+4-4\sigma, 21\rho/8+\tau/4+1-2\sigma, 3\rho+1-2\sigma).$$
\end{lemma}

\begin{proof} See \cite[Proposition 11.1]{guth-maynard}.
\end{proof}

We can put this all together to prove the Guth--Maynard large values theorem (Theorem \ref{guth-maynard-lvt}):


\begin{theorem}[Guth--Maynard large values theorem, again]\label{guth-maynard-lvt-again} One has
    $$ \LV(\sigma,\tau) \leq \max(2-2\sigma, 18/5 - 4 \sigma, \tau + 12/5 - 4\sigma).$$
\end{theorem}

\begin{proof}\uses{l2-mvt,huxley-lv, gm-1, gm-3, montgomery-subdivide} For $\sigma \leq 7/10$ this follows from Lemma \ref{l2-mvt}, and for $\sigma \geq 8/10$ it follows from Lemma \ref{huxley-lv}.  Thus we may assume that $7/10 \leq \sigma \leq 8/10$.  By subdivision (Lemma \ref{montgomery-subdivide}) it then suffices to treat the case $\tau = 6/5$, that is to say to show that
    $$ \rho \leq 18/5-4\sigma$$
whenever $(\sigma,\tau,\rho,\rho^*,s) \in \Energy$ with $\tau=6/5$ and $7/10 \leq \sigma \leq 8/10$.

Applying Lemma \ref{gm-1} and discarding the very negative $S_1$ term, we have
$$ \rho \leq \max(2-2\sigma, 1-2\sigma + \max(S_2, S_3)/3)$$
where $S_2, S_3$ are real numbers with
$$ S_2 \leq \max(2+2\rho, \tau+1+(2-1/k) \rho, 2 + 2\rho + (\tau/2 - 3\rho/4)/k )$$
for any positive integer $k$ and
$$ S_3 \leq 2\tau + \rho/2 + \rho^*/2$$
and also
$$ S_3 \leq \max( 2\tau + 3\rho/2, \tau+1+\rho/2+\rho^*/2).$$
From the latter bound and Lemma \ref{gm-3}, one has
$$ S_3 \leq \max( 2\tau+3\rho/2, \tau+\rho+3-2\sigma, \tau+2\rho+3/2-\sigma, 9\tau/8+29\rho/16 + 3/2-\sigma).$$
Inserting this and the $S_2$ bound (with $k=4$) into the bound for $\rho$ and simplifying (using $\tau=6/5$), we eventually obtain
the desired bound $\rho \leq 18/5-4\sigma$.
\end{proof}
