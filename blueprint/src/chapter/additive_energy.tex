\chapter{Large value additive energy}\label{energy-chapter}

\section{Additive energy}

\begin{definition}[Additive energy]\label{energy-def}  Let $W$ be a finite set of real numbers.  The \emph{additive energy} $E_1(W)$ of such a set is defined to be the number of quadruples $(t_1,t_2,t_3,t_4) \in W$ such that
$$
|t_1 + t_2 - t_3 - t_4| \leq 1.$$
\end{definition}

We remark that in additive combinatorics, the variant $E_0(W)$ of the additive energy is often studied, in which $t_1+t_2-t_3-t_4$ is not merely required to be $1$-bounded, but in fact vanishes exactly.  However, this version of additive energy is less relevant for analytic number theory applications.

\begin{lemma}[Basic properties of additive energy]\label{add-energy}\uses{energy-def}
\begin{itemize}
\item[(i)] If $W$ is a finite set of reals, then
$$ E_1(W) \asymp \int_\R |\# \{ (t_1,t_2) \in W: |t_1+t_2 - x| \leq 1\} |^2\ dx.$$
More generally, for any $r>0$ we have
$$ E_1(W) \asymp r^{O(1)} \int_\R |\# \{ (t_1,t_2) \in W: |t_1+t_2 - x| \leq r\} |^2\ dx.$$
\item[(ii)] If $W$ is a finite set of reals, then
$$ E_1(W) \asymp \int_{-1}^1 |\sum_{t \in W} e(t\theta)|^4\ d\theta.$$
\item[(iii)] If $W_1,\dots,W_k$ are finite sets of reals, then
$$ E_1(W_1 \cup \dots \cup W_k)^{1/4} \ll E_1(W_1)^{1/4} + \dots + E_1(W_k)^{1/4}.$$
\item[(iv)]  If $W$ is $1$-separated and contained in an interval of length $T \geq 1$, then
$$ (\# W)^2, (\# W)^4/T \ll E_1(W) \ll (\# W)^3.$$
\item[(v)]  If $W$ is contained in an interval $I$, which is in turn split into $K$ equally sized subintervals $J_1,\dots,J_K$, then
$$ E_1(W)^{1/3} \ll \sum_{k=1}^K E_1(W \cap J_k)^{1/3}.$$
\end{itemize}
\end{lemma}

Note that the lower bound of $(\# W)^4 / T$ would be expected to be attained if the set $W$ is distributed ``randomly'' and is reasonably large (of size $\gg \sqrt{T}$).  So  getting upper bounds of the additive energy of similar strength to this lower bound can be viewed as a statement of ``pseudorandomness'' (or ``Gowers uniformity'') of this set.

\begin{proof} For (i), we just prove the first estimate, as the second follows from the first by several applications of the triangle inequality.  The right-hand side can be expanded as
    $$ \sum_{t_1,t_2,t_3,t_4 \in W} |\{ x: |t_1+t_2-x|, |t_3+t_4-x| \leq 1 \}|.$$
Every quadruple contributing to $E_1(W)$ then contributes $\gg 1$ to the right-hand side, giving the upper bound.  To get the matching lower bound, note that
$$ \sum_{t_1,t_2,t_3,t_4 \in W} |\{ x: |t_1+t_2-x|, |t_3+t_4-x| \leq 1/2 \} \leq E_1(W)$$
and hence
$$ E_1(W) \gg \int_\R |\# \{ (t_1,t_2) \in W: |t_1+t_2 - x| \leq 1/2\} |^2\ dx.$$
The upper bound then follows from the triangle inequality.

For (ii), we can upper bound the indicator function of $[-1,1]$ by the Fourier transform of a non-negative bump function $\varphi$, so that the right-hand side is bounded by
$$ \sum_{t_1,t_2,t_3,t_4 \in W} \varphi(t_1+t_2-t_3-t_4)$$
which is then bounded by $O(E_1(W))$ by choosing the support of $\varphi$ appropriately.  The lower bound is established similarly (using the arguments in (i) to adjust the error tolerance $1$ in the constraint $ |t_1+t_2 - x| \leq 1$ as necessary.)

For (iii), first observe we may remove duplicates and assume that the $W_i$ are disjoint, then we can use (ii) and the triangle inequality.

For (iv), the first lower bound comes from considering the diagonal case $t_1=t_3, t_2 = t_4$ and the upper bound comes from observing that once $t_1,t_2,t_3$ are fixed, there are only $O(1)$ choices for $t_4$ thanks to the $1$-separated hypothesis.  Finally, observe that
$$ \int_\R |\# \{ (t_1,t_2) \in W: |t_1+t_2 - x| \leq 1\} |\ dx = (2 \# W)^2$$
hence by Cauchy--Schwarz
$$ \int_\R |\# \{ (t_1,t_2) \in W: |t_1+t_2 - x| \leq 1\} |^2\ dx \gg (\# W)^2/T$$
and the claim follows from (i).

For (v), write $a_k := E_1(W \cap J_k)^{1/4}$.  Each tuple $(t_1,t_2,t_3,t_4)$ that contributes to $E_1(W)$ is associated to a tuple $J_{k_1}, J_{k_2}, J_{k_3}, J_{k_4}$ of intervals with $k_1+k_2-k_3-k_4=O(1)$.  By modifying the proof of (ii), the total contribution of such a tuple of intervals is
$$ \ll \int_\R \prod_{j=1}^4 |\sum_{t \in W \cap J_{k_j}} e(t\theta)|\ d\theta$$
which by Cauchy--Schwarz is bounded by
$$ \ll a_{k_1} a_{k_2} a_{k_3} a_{k_4}.$$
Thus we see that
$$ E_1(W) \ll \sum_{m=O(1)} a * a * \tilde a * \tilde a(m)$$
where $\tilde a_k := a_{-k}$ and $*$ denotes convolution on the integers.  By Young's inequality we then have
$$ E_1(W) \ll \|a\|_{\ell^{4/3}}^4$$
and the claim follows.

We remark that (v) can also be proven using \cite[Lemma 4.8, (4.2)]{cladek-tao}.
\end{proof}

We will also study the following related quantity.  Given a set $W$ and a scale $N>1$, let $S(N,W)$ denote the \emph{double zeta sum}
\begin{equation}\label{snw-def}
     S(N,W) := \sum_{t,t' \in W} \left|\sum_{n \in [N,2N]} n^{-i(t-t')} \right|^2.
\end{equation}
We caution that this normalization differs from the one in \cite{ivic}, where $n^{-1/2-i(t-t')}$ is used in place of $n^{-i(t-t')}$.  This sum may also be rearranged as
\begin{equation}\label{snw}
 S(N,W) = \sum_{n,m \in [N,2N]} |R_W(n/m)|^2
\end{equation}
where $R_W$ is the exponential sum
$$ R_W(x) := \sum_{t \in W} x^{it}.$$
From the first formula it is clear that $S(N,W)$ is monotone non-decreasing in $W$, and from the second formula one has the triangle inequality
\begin{equation}\label{S-triangle}
S\left(N, \bigcup_{i=1}^k W_i \right)^{1/2} \leq \sum_{i=1}^k S(N,W_i)^{1/2}
\end{equation}
when the $W_i$ are disjoint, and hence also when they are not assumed to be disjoint, thanks to the monotonicity.

The following Cauchy--Schwarz inequality is also useful:

\begin{lemma}[Cauchy--Schwarz and double $\zeta$-sums]\label{cauchy-schwarz}\uses{energy-def} \cite[Lemma 3.4]{bourgain_dirichlet_2000} If $W,W'$ are finite sets of reals, $N>1$, and $a_n$ is a $1$-bounded sequence for $n \in [N,2N]$, then
\begin{equation}\label{ttww}
     \sum_{t \in W, t' \in W'} \left|\sum_{n \in [N,2N]} a_n n^{-it} \right|^2 \leq S(N,W)^{1/2} S(N,W')^{1/2}.
\end{equation}
In particular
$$ \sum_{t \in W} \left|\sum_{n \in [N,2N]} a_n n^{-it} \right|^2 \leq S(N,W).$$
\end{lemma}

\begin{proof}  The left-hand side of \eqref{ttww} can be rewritten as
    $$ \sum_{n,m \in [N,2N]} a_n \overline{a_m} \overline{R_W}(n/m) R_{W'}(n/m).$$
The claim is now immediate from \eqref{snw} and the Cauchy--Schwarz inequality.
\end{proof}


To relate $S(N,W)$ to $E_1(W)$, we first observe the following lemma, implicit in \cite{heath_brown_consecutive_II} and made more explicit in \cite[Lemma 11.4]{guth-maynard}.

\begin{lemma}[Energy controlled by third moment]\label{energy-third}\uses{energy-def} Suppose that $(N,T,V,(a_n)_{n \in [N,2N]},J,W)$ is a large value pattern with $T \geq 1$ and $1 \leq N \ll T^{O(1)}$.  Then
$$ V^2 E_1(W) \ll T^{o(1)} \sum_{n,m \in [N,2N]} |R_W(n/m)|^3 + T^{-50}.$$
\end{lemma}

\begin{proof} By hypothesis, we have
$$ V^2 E_1(W) \leq \sum_{t_1,t_2,t_3,t_4 \in W: |t_1+t_2-t_3-t_4| \leq 1} \left|\sum_{n \in [N,2N]} a_n n^{-it_4} \right|^2.$$
By standard Fourier arguments (see \cite[Lemma 11.3]{guth-maynard}), we can bound
$$ \left|\sum_{n \in [N,2N]} a_n n^{-it_4}\right| \ll T^{o(1)} \int_{t: |t-t_4| \leq T^{o(1)}} \left|\sum_{n \in [N,2N]} a_n n^{-it}\right|\ dt + T^{-100}.$$
Since each $t_1,t_2,t_3$ generates at most $O(1)$ choices for $t_4$, we conclude that
$$ V^2 E_1(W) \ll T^{o(1)} \sum_{t_1,t_2,t_3 \in W} \int_{s: |s| \leq T^{o(1)}}  \left|\sum_{n \in [N,2N]} a_n n^{-i(t_1+t_2-t_3+s)} \right|^2\ ds + T^{-50},$$
The right-hand side can be rewritten as
$$ T^{o(1)} \sum_{n,m \in [N,2N]} a_n \overline{a_m} (n/m)^{-is} \overline{R_W}(n/m)^2 R_W(n/m) + T^{-50},$$
and the claim then follows from the triangle inequality.
\end{proof}

Thus, $S(N,W)$ involves a second moment of $R_W$, while the energy $E_1(W)$ is related to the third moment.  Using the trivial bound $|R_W(x)| \leq |W|$ we can then obtain the trivial bound
\begin{equation}\label{energy-triv}
V^2 E_1(W) \ll T^{o(1)} |W| S(N,W) + T^{-50}
\end{equation}
It is then natural to introduce the fourth moment
$$ S_4(N,W) := \sum_{n,m \in [N,2N]} |R_W(n/m)|^4$$
since from H\"older's inequality one now has
\begin{equation}\label{v1w}
    V^2 E_1(W) \ll T^{o(1)} S(N,W)^{1/2} S_4(N,W)^{1/2} + T^{-50}
\end{equation}
(cf. \cite[Lemma 3]{heath_brown_consecutive_II}).  The quantity $S_4(N,W)$ can also be expressed as
$$ S_4(N,W) = \sum_{t_1,t_2,t_3,t_4 \in W} \left|\sum_{n \in [N,2N]} n^{-i(t_1+t_2-t_3-t_4)}\right|^2.$$

One can bound this quantity by an $S(N,W)$ type expression:

\begin{lemma}\label{wtu}\uses{energy-def} If $W \subset [-T,T]$ is $1$-separated and $1 \leq N \ll T^{O(1)}$, then one has
$$ S_4(N,W) \ll T^{o(1)} u^2 S(N,U) + T^{-100}$$
for some $1 \leq u \ll |W|$ and $1$-separated subset $U$ of $[-2T,2T]$ with
\begin{equation}\label{v1}
 u |U| \ll |W|^2
\end{equation}
and
\begin{equation}\label{v2}
     u^2 |U| \ll E_1(W).
\end{equation}
\end{lemma}

This result appears implicitly in \cite[p. 229]{heath_brown_consecutive_II}, and is made more explicit in the proof of \cite[Lemma 11.6]{guth-maynard}.

\begin{proof} One can bound
    $$ S_4(N,W) \ll T^{o(1)} \sum_{t_1,t_2,t_3,t_4 \in W} \int_{t = t_1+t_2-t_3-t_4+O(T^{o(1)})} \left|\sum_{n \in [N,2N]} n^{-it}\right|^2\ dt + T^{-100},$$
    and hence
    $$ S_4(N,W) \ll T^{o(1)} \sum_{t_1,t_2 \in [-2N,2N] \cap \Z} f(t_1) f(t_2) \int_{t = t_1-t_2+O(T^{o(1)})} \left|\sum_{n \in [N,2N]} n^{-it}\right|^2\ dt + T^{-100}$$
where $f$ is the counting function
$$ f(t) := |\{ (t_1,t_2) \in W: |t-t_1-t_2| \leq 1 \}|.$$
Note that $f$ is integer valued and bounded above by $|W|$. By dyadic decomposition, one can then find $1 \leq u \ll |W|$ and a subset $U$ of $[-2N,2N] \cap \Z$ such that $f(t) \asymp u$ for $t \in U$, and
$$ S_4(N,W) \ll T^{o(1)} \sum_{t_1,t_2 \in U} u^2 \int_{t = t_1-t_2+O(T^{o(1)})} \left|\sum_{n \in [N,2N]} n^{-it} \right|^2\ dt + T^{-100}$$
which we can rearrange as
$$ S_4(N,W) \ll T^{o(1)} u^2 \int_{s = O(T^{o(1)})} \sum_{n,m \in [N,2M]} (n/m)^{is} |R_U(n/m)|^2\ ds + T^{-100}$$
and hence by the triangle inequality
$$ S_4(N,W) \ll T^{o(1)} v^2 S(N,V) + T^{-100}.$$
Also, by double counting one easily verifies the claims \eqref{v1}, \eqref{v2}.  The claim follows.
\end{proof}


\section{Large value additive energy region}

Because the cardinality $|W|$ and additive energy $E_1(W)$ of a set $W$ are correlated with each other, as well as with the double zeta sum $S(N,W)$, we will not be able to consider them separately, and instead we will need to consider the possible joint exponents for these two quantities.  We formalize this via the following set:

\begin{definition}[Large value energy region]\label{lv-edef} The \emph{large value energy region} $\Energy \subset \R^5$ is defined to be the set of all fixed tuples $(\sigma,\tau,\rho,\rho^*,s)$ with $1/2 \leq \sigma \leq 1$, $\tau, \rho, \rho' \geq 0$, such that there exists a large value pattern $(N,T,V,(a_n)_{n \in [N,2N]},J,W)$ with $N>1$ unbounded, $V = N^{\sigma+o(1)}$, $T = N^{\tau+o(1)}$, $V = N^{\sigma+o(1)}$, $|W| = N^{\rho+o(1)}$, $E_1(W) = N^{\rho^*+o(1)}$ and $S(N,W) = N^{s+o(1)}$.

We define the \emph{zeta large value energy region} $\Energy_\zeta \subset \R^5$ similarly, but where now $(N,T,V,(a_n)_{n \in [N,2N]},J,W)$ is required to be a zeta large value pattern.
\end{definition}

Clearly we have

\begin{lemma}[Trivial containment]\label{triv-contain}\uses{lv-edef} We have $\Energy_\zeta \subset \Energy$.
\end{lemma}

These regions are related to $\LV(\sigma,\tau)$ and $\LV_{\zeta}(\sigma,\tau)$ as follows:

\begin{lemma}\label{energy-region-lv}\uses{lv-def, lvz-def, energy-def} For any fixed $1/2 \leq \sigma \leq 1, \tau \geq 0$, we have
$$ \LV(\sigma,\tau) = \sup \{ \rho: (\sigma,\tau,\rho,\rho^*,s) \in \Energy\}$$
and
$$ \LV_\zeta(\sigma,\tau) = \sup \{ \rho: (\sigma,\tau,\rho,\rho^*,s) \in \Energy_\zeta\}$$
In particular, we have $\rho \leq \LV(\sigma,\tau)$ for all $(\sigma,\tau,\rho,\rho^*,s) \in \Energy$, and $\rho \leq \LV_\zeta(\sigma,\tau)$ for all $(\sigma,\tau,\rho,\rho^*,s) \in \Energy_\zeta$.
\end{lemma}

\begin{proof} Clear from definition.
\end{proof}

Inspired by this, we can define

\begin{definition}\label{lvze-def}\uses{energy-def}  For any fixed $1/2 \leq \sigma \leq 1, \tau \geq 0$, we define
$$ \LV^*(\sigma,\tau) := \sup \{ \rho^*: (\sigma,\tau,\rho,\rho^*,s) \in \Energy\}$$
and
$$ \LV^*_\zeta(\sigma,\tau) := \sup \{ \rho^*: (\sigma,\tau,\rho,\rho^*,s) \in \Energy_\zeta\}.$$
\end{definition}

Thus these exponents are upper bounds for the additive energy of large values of Dirichlet polynomials which may or may not be of zeta function type.

As usual, we have an equivalent non-asymptotic definition of the large value energy region:

\begin{lemma}[Non-asymptotic form of large value energy region]\label{lve-asymp}\uses{lv-edef} Let $1/2 \leq \sigma \leq 1$, $\tau \geq 0$,  $\rho, \rho^* \geq 0$, and $s \in \R$ be fixed.  Then the following are equivalent:
    \begin{itemize}
    \item[(i)] $(\sigma,\tau,\rho,\rho^*,s) \in \Energy$.
    \item[(ii)] For every $\eps>0$ and $C > 0$, there exists a large value pattern $(N,T,V,(a_n)_{n \in [N,2N]},J,W)$ with $N \geq C$, $N^{\tau-\delta} \leq T \leq N^{\tau+\delta}$, $N^{\sigma-\delta} \leq V \leq N^{\sigma+\delta}$,
    $N^{\rho-\eps} \leq |W| \leq N^{\rho+\eps}$,
    $N^{\rho^*-\eps} \leq E_1(W) \leq N^{\rho^*+\eps}$, and
    $N^{s-\eps} \leq S(N, W) \leq N^{s+\eps}$.
    \end{itemize}
    Similarly with $\Energy$ replaced by $\Energy_\zeta$, and with $(N,T,V,(a_n)_{n \in [N,2N]},J,W)$ required to be a zeta large value pattern.
\end{lemma}

This lemma is proven by a routine expansion of the definitions, and is omitted.

\begin{lemma}[Basic properties]\label{lve-basic}\uses{lv-edef}\
    \begin{itemize}
        \item[(i)] (Monotonicity in $\sigma$) If $(\sigma,\tau,\rho,\rho^*,s) \in \Energy$, then
        $(\sigma',\tau',\rho,\rho^*,s) \in \Energy$ for all $1/2 \leq \sigma' \leq \sigma$ and $\tau' \geq \tau$.
        \item[(ii)] (Subdivision) If $(\sigma,\tau,\rho,\rho^*,s) \in \Energy$ and $0 \leq \tau' \leq \tau$, then amongst all tuples $(\sigma,\tau', \rho', (\rho')^*,s') \in \Energy$ with $\rho' \leq \rho$, $(\rho')^* \leq \rho^*$, and $s' \leq s$, there exists a tuple with
        $$\rho \leq \rho' + \tau - \tau';$$
        there exists a tuple with
        $$\rho^* \leq \rho' + 3\min(\rho-\rho',\tau-\tau');$$
        and there exists a tuple with
        $$s \leq s' + 2\min(\rho-\rho',\tau-\tau').$$
        (But it may not be the same tuple that satisfies all three properties.)
        \item[(iii)]  (Trivial bounds) If $(\sigma,\tau,\rho,\rho^*,s) \in \Energy$, one has
        $$ 2\rho, 4\rho-\tau \leq \rho^* \leq 3 \rho.$$
    \end{itemize}
\end{lemma}

\begin{proof}  The claim (i) is trivial, so we turn to (ii).  By definition, there exists a large value pattern $(N,T,V,(a_n)_{n \in [N,2N]},J,W)$ with $N > 1$ unbounded, $T = N^{\tau+o(1)}$, $V = N^{\sigma+o(1)}$, $|W| = N^{\rho+o(1)}$, $E_1(W) = N^{\rho^*+o(1)}$, and $S(N,W) = N^{s+o(1)}$.  We now partition $J$ into $N^{\tau-\tau'+o(1)}$ subintervals $I$ of length $N^{\tau'+o(1)}$, and subdivide $W$ into $W_I$ accordingly.  By dyadically pigeonholing, we can then subdivide this collection $I$ of intervals into $N^{o(1)}$ subcollections, where on each subcollection there exists fixed $\rho', (\rho')^*, s'$ such that $|W_I| = N^{\rho'+o(1)}$, $E_1(W_I) = N^{(\rho')^* + o(1)}$, and $S(N,W_I) = N^{s'+o(1)}$.  Since $W_I \subset W$, this forces $\rho' \leq \rho$, $(\rho')^* \leq \rho^*$, and $s' \leq s$.  From Definition \ref{lv-edef} we see that $(\sigma,\tau', \rho', (\rho')^*,s') \in \Energy$.

By the pigeonhole principle, one of these subcollections must contribute at least $N^{-o(1)}$ of the cardinality of $W$.  Since there are at most $N^{\tau-\tau'+o(1)}$ intervals in this collection, we must have $\rho \leq \rho' + \tau - \tau'$ in this case.

By Lemma \ref{add-energy}(iii), we also know that a (possibly different subcollection) must contribute at least $N^{-o(1)}$ of the additive energy of $W$.  The number of intervals in this subcollection is at most $\min( N^{\tau-\tau'+o(1)}, N^{\rho-\rho'+o(1)})$.  Applying Lemma \ref{add-energy}(iii) again, we conclude $\rho^* \leq (\rho^*)' + 3\min(\rho-\rho',\tau-\tau')$.

Finally, from \eqref{S-triangle}, we know that a (possibly yet another subcollection) must contribute at least $N^{-o(1)}$ of the double zeta sum $S(N,W)$. The number of intervals in this subcollection is at most $\min( N^{\tau-\tau'+o(1)}, N^{\rho-\rho'+o(1)})$.  Applying Lemma \ref{S-triangle}(iii) again, we conclude $s \leq s' + 2\min(\rho-\rho',\tau-\tau')$.
\end{proof}


\begin{lemma}[Raising to a power]\label{power-energy}  If $(\sigma,\tau,\rho,\rho^*,s) \in \Energy$, and $k \geq 1$, then amongst all tuples $(\sigma,\tau/k,\rho',(\rho')^*,s') \in \Energy$ with $\rho' \leq \rho/k$, $(\rho')^* \leq \rho^*/k$, and $s' \leq s/k$, there exists a tuple with $\rho' = \rho/k$, there exists a tuple with $(\rho')^* = \rho^*/k$, and there exists a tuple with $s' = s/k$. (These may be three different tuples.)
\end{lemma}

\begin{proof} By definition, there exists a large value pattern $(N,T,V,(a_n)_{n \in [N,2N]},J,W)$ with $N > 1$ unbounded, $T = N^{\tau+o(1)}$, $V = N^{\sigma+o(1)}$, $|W| = N^{\rho+o(1)}$, $E_1(W) = N^{\rho^*+o(1)}$, and $S(N,W) = N^{s+o(1)}$.  Observe that
    $$ \left( \sum_{n \in [N,2N]} a_n n^{-it} \right)^k = \sum_{n \in [N^k, 2^k N^k]} b_n n^{-it}$$
for some coefficients $b_n = O(n^{o(1)})$.  In particular, partitioning $[N^k, 2^k N^k]$ into $O(1)$ sub-intervals $[N',2N']$ with $N' = N^{k+o(1)}$, we can partition $W$ into $O(1)$ subcollections $W_{N'}$, such that
$$ \left| \sum_{n \in [N',2N']} b_n n^{-it} \right| \gg V^k = (N')^{\sigma+o(1)}$$
for all $t \in W_{N'}$.  Again by the pigeonhole principle, one of the $W_{N'}$ must have cardinality $N^{\rho+o(1)}$, one must have energy $N^{\rho^*+o(1)}$, and one must have double zeta sum $N^{s+o(1)}$ (but these may be different $W_{N'}$).  Each of these $W_{N'}$ then give the different conclusions to the lemma.
\end{proof}

Morally speaking, one should be able to obtain equality in all three conclusions of Lemma \ref{power-energy} simultaneously, i.e. that $(\sigma,\tau,\rho,\rho^*,s) \in \Energy$ essentially implies $(\sigma,\tau/k,\rho/k,\rho^*/k,s/k) \in \Energy$. This is because in practice one frequently controls $\Energy$ by computing a containment region $\Energy_1$ that possesses precisely the required monotonicity property. Specifically, we have

\begin{lemma}[Monotonicity criterion]\label{lver_e_mono_crit}
Let $\Energy_1$ be the intersection of sets $E_i$, each of the form
\begin{align*}
\{(\sigma, \tau, \rho, \rho^*, s) \in \R^5: \rho \le f_1(\rho^*, s), \rho^* \le f_2(\rho, s), s \le f_3(\rho, \rho^*)\}
\end{align*}
for some monotonically increasing functions $f_1, f_2, f_3$ (that possibly also depend on $\sigma$ and $\tau$).

Suppose amongst all tuples $(\sigma, \tau, \rho', (\rho^*)', s') \in \Energy_1$ with $\rho' \le \rho$, $(\rho^*)' \le \rho^*$ and $s' \le s$, there exists a tuple with $\rho' = \rho$, a tuple with $(\rho^*)' = \rho^*$ and a tuple with $s' = s$ (not necessarily the same tuple each time). Then, $(\sigma, \tau, \rho, \rho^*, s) \in \Energy_1$.
\end{lemma}
\begin{proof}
Suppose that $(\sigma, \tau, \rho, (\rho^*)', s') \in \Energy_1$ for some $(\rho^*)'\le \rho^*$ and $s \le s'$ so that also $(\sigma, \tau, \rho, (\rho^*)', s') \in E_i$. Then by definition $\rho \le f_1((\rho^*)', s')$. Since $f_1$ is monotonically increasing (with respect to both $\rho^*$ and $s$), one has $\rho \le f_1(\rho^*, s)$. Similarly, $(\sigma, \tau, \rho', \rho^*, s') \in \Energy_1$ implies $\rho^* \le f_2(\rho, s)$ and $(\sigma, \tau, \rho', (\rho^*)', s) \in \Energy_1$ implies $s \le f_3(\rho, \rho^*)$, which together imply $(\sigma, \tau, \rho, \rho^*, s) \in E_i$ by definition. Hence $(\sigma, \tau, \rho, \rho^*, s) \in \Energy_1$ since $\Energy_1$ is the intersection of sets $E_i$. 
\end{proof}

\begin{lemma}[Raising to a power, alternative formulation]
Let $k$ be a positive integer, $\mathcal{E}_1 \subseteq \R^{5}$ be a set satisfying the monotonicity criterion of Lemma \ref{lver_e_mono_crit} and
\[
\mathcal{E}_k := \{(\sigma, \tau, \rho, \rho^*, s) \in \R^5: (\sigma, \tau/k, \rho/k, \rho^*/k, s/k) \in \mathcal{E}_1\}.
\]
If $\mathcal{E}\subseteq \mathcal{E}_1$ then $\mathcal{E} \subseteq \mathcal{E}_k$.
\end{lemma}
\begin{proof}
Suppose that $(\sigma, \tau, \rho, \rho^*, s) \in \mathcal{E}\subseteq \mathcal{E}_1$. By Lemma \ref{power-energy} and Lemma \ref{lver_e_mono_crit}, $(\sigma, \tau/k, \rho/k, \rho^*/k, s/k) \in \mathcal{E}_1$, hence by definition $(\sigma, \tau, \rho, \rho^*, s) \in \mathcal{E}_k$.
\end{proof}
\section{Known relations for the large value energy region}

\begin{theorem}[Reflection principle]\label{reflect}\cite[\S 11.5]{ivic}\uses{lv-edef} If $(\sigma,\tau,\rho,\rho^*,s) \in \Energy$ with $\sigma \geq 3/4$ and $\tau>1$, then for any integer $k \geq 1$, either $\rho \leq 2-2\sigma$, or there exists $0 < \alpha \leq k(\tau-1)$ and $(\sigma, \tau/\alpha, \rho/\alpha, \rho^*/\alpha, s'/\alpha) \in \Energy$ such that
$$ \rho \leq \min( 2-2\sigma, k(3-4\sigma)/2 + s' - 1).$$
\end{theorem}

\begin{proof} By definition, there exists a large value pattern $(N,T,V,(a_n)_{n \in [N,2N]},J,W)$ with $N>1$ unbounded, $T = N^{\tau+o(1)}$, $|W| = N^{\rho+o(1)}$, $E_1(W) = N^{\rho^*+o(1)}$ and $S(N,W) = N^{s+o(1)}$.  By \cite[(11.58)]{ivic}, one has
$$ |W|^2 \ll T^{o(1)} \left( |W| N^{2-2\sigma} + N^{1-2\sigma} |W|^2 + N^{(3-4\sigma)/2} \int_{v = O(T^{o(1)})} \sum_{t,t' \in W} \left|\sum_{n \leq 4T/N} n^{-1/2+it-it'+iv}\right|\ dv \right).$$
Since $\sigma>1/2$, the $N^{1-2\sigma} |W|^2$ term can be dropped.  Applying H\"older's inequality and dyadic pigeonholing as in \cite[(11.59)]{ivic}, we conclude that
$$ |W| \ll T^{o(1)} \left(N^{2-2\sigma} + N^{k(3-4\sigma)/2}  \left(\sum_{t,t' \in W} \left|\sum_{n \in [N',2N']} b_n n^{-1/2+it-it'+iv}\right|^2\right)^{1/2}\right)$$
for some $v = O(T^{o(1)})$ and coefficients $b_n = O(T^{o(1)})$, and some $N' \ll (4T/N)^k$.  After passing to a subsequence if necessary, we may assume that $N' = N^{\alpha+o(1)}$ for some $0 \leq \alpha \leq k(\tau-1)$.  If $\alpha=0$ then the second term here is negligible compared to the first and we obtain $\rho \leq 2-2\sigma$, so suppose that $\alpha > 0$.  Using \cite[Lemma 11.1]{ivic} to eliminate the $b_n n^{-1/2+iv}$ coefficients, we conclude that
$$ |W| \ll T^{o(1)} (N^{2-2\sigma} + N^{k(3-4\sigma)/2-1} S(N', W)).$$
By construction, we have $S(N',W) = (N')^{s'/\alpha+o(1)} = N^{s'+o(1)}$ for some tuple $(\sigma, \tau/\alpha, \rho/\alpha, \rho^*/\alpha, s'/\alpha) \in \Energy$.  The claim follows.
\end{proof}

Heuristically one expects $s \leq \max( \rho+1, 2\rho)+1$ (see \cite[(11.63)]{ivic}).  There is one easy case in which this is true:

\begin{lemma}\label{easy-double-zeta-bound}\uses{lv-edef}  If $(\sigma,\tau,\rho,\rho^*,s) \in \Energy$ with $\tau < 1$, then $s \leq \max(\rho+1, 2\rho)+1$.
\end{lemma}

\literature
\code{add_lver_ivic_1985()}

\begin{proof} By definition, there exists a large value pattern $(N,T,V,(a_n)_{n \in [N,2N]},J,W)$ with $N>1$ unbounded, $T = N^{\tau+o(1)}$, $|W| = N^{\rho+o(1)}$, $E_1(W) = N^{\rho^*+o(1)}$ and $S(N,W) = N^{s+o(1)}$.  By the discussion after \cite[(11.63)]{ivic}, we have
    $$ N^{-1} S(N,W) \ll T^\eps ( |W| N + |W|^2 )$$
for any fixed $\eps>0$, which gives the claim.    .
\end{proof}

Another bound is

\begin{lemma}\label{double-zeta_from_exp_pair}\cite[Lemma 11.2]{ivic}\uses{lv-edef}  If $(k,\ell)$ is an exponent pair with $k>0$, and $(\sigma,\tau,\rho,\rho^*,s) \in \Energy$, then
$$ s \leq \max\left( \rho+1, 5\rho/3 + \tau/3, \frac{2+3k+4\ell}{1+2k+2\ell} \rho + \frac{k+\ell}{1+2k+2\ell} \tau\right) + 1.$$
\end{lemma}

\begin{proof} By definition, there exists a large value pattern $(N,T,V,(a_n)_{n \in [N,2N]},J,W)$ with $N>1$ unbounded, $T = N^{\tau+o(1)}$, $|W| = N^{\rho+o(1)}$, $E_1(W) = N^{\rho^*+o(1)}$ and $S(N,W) = N^{s+o(1)}$.   From \cite[Lemma 11.2]{ivic} we have
$$ N^{-1} S(N,W) \ll |W| N + |W|^{5/3} T^{1/3+\eps} + |W|^{\frac{2+3k+4\ell}{1+2k+2\ell}} T^{\frac{k+\ell+\eps}{1+2k+2\ell}}$$
for any fixed $\eps>0$, which gives the claim.
\end{proof}

\python{additive_energy}
\code{ep_to_lver(eph)}

Finally, we have the useful

\begin{lemma}[Heath-Brown bound on double sums]\label{hb-double}\uses{lv-edef}  If $(\sigma,\tau,\rho,\rho^*,s) \in \Energy$, then
    $$ s \leq \max( \rho+1, 2\rho, 5\rho/4+\tau/2) + 1.$$
\end{lemma}

Note that if $\tau \leq 3/2$, the $5\rho/4+\tau/2$ term is bounded by the convex combination $(3/4)(\rho+1)+(1/4)(2\rho)$ and may therefore be omitted.

\literature
\code{add_lver_heath_brown_1979()}

\begin{proof} By definition, there exists a large value pattern $(N,T,V,(a_n)_{n \in [N,2N]},J,W)$ with $N>1$ unbounded, $T = N^{\tau+o(1)}$, $|W| = N^{\rho+o(1)}$, $E_1(W) = N^{\rho^*+o(1)}$ and $S(N,W) = N^{s+o(1)}$.   From \cite[Theorem 1]{heathbrown_large_1979} or \cite[Lemma 11.5]{ivic}, one has
$$ N^{-1} S(N,W) \ll T^\eps ( |W| N + |W|^2 + |W|^{5/4} T^{1/2} ),$$
giving the claim.
\end{proof}

Lemma \ref{wtu} can be formulated in terms of the large value energy region as follows.

\begin{lemma}\label{wtu-alt}\uses{lv-edef}  If $(\sigma,\tau,\rho,\rho^*,s) \in \Energy$, then there exists $(\sigma,\tau,\rho',(\rho')^*,s') \in \Energy$ and $0 \leq \kappa \leq \rho$ such that
$$ \kappa + \rho' \leq 2 \rho$$
$$ 2\kappa + \rho' \leq \rho^*$$
and
$$ \rho^* + 2\sigma \leq \kappa + (s+s')/2.$$
\end{lemma}


\begin{proof}\uses{wtu}  By definition, there exists a large value pattern $$(N,T,V,(a_n)_{n \in [N,2N]},J,W)$$ with $N \geq 1$ unbounded, $T = N^{\tau+o(1)}$, $V = N^{\sigma+o(1)}$, $|W| = N^{\rho+o(1)}$, $E_1(W) = N^{\rho^*+o(1)}$, and $S(N,W) = N^{s+o(1)}$. From \eqref{v1w} we have
$$
    V^2 E_1(W) \ll T^{o(1)} S(N,W)^{1/2} S_4(N,W)^{1/2} + T^{-50}.
$$
By Lemma \ref{wtu}, there exists $1 \leq u \ll |W|$ and a $1$-separated subset $U$ of $[-2T,2T]$ such that
 such that
$$
    V^2 E_1(W) \ll T^{o(1)} u S(N,W)^{1/2} S(N,U)^{1/2} + T^{-50}
$$
with \eqref{v1}, \eqref{v2} holding.  Since $W$ is non-empty, $E_1(W) \geq 1$ and $V \geq N^{1/2} \geq 1$, so the $T^{-50}$ error here may be discarded.  Passing to a subsequence, we may assume that $u = N^{\kappa+o(1)}$ for some $0 \leq \kappa \leq \rho$, and that $|U| = N^{\rho'+o(1)}$ for some $\rho' \geq 0$.  Then we have $S_2(N,U) = s'$ for some
$(\sigma,\tau,\rho',(\rho')^*,s') \in \Energy$, and the claim follows.
\end{proof}

These bounds on the double zeta sums can be used to control additive energies:

\begin{theorem}[Heath-Brown relation]\label{hbt}\cite[(33)]{heathbrown_zero_1979}\uses{lv-edef} If $(\sigma,\tau,\rho,\rho^*,s) \in \Energy$, then one has
$$ \rho^* \leq 1-2\sigma + \max(\rho+1, 2\rho, 5\rho/4+\tau/2)/2 + \max(\rho^*+1, 4\rho, 3\rho^*/4+\rho+\tau/2)/2.$$
\end{theorem}

\literature
\code{add_lver_heath_brown_1979b1()}

\begin{proof}\uses{wtu-alt} By Lemma \ref{wtu-alt} we have
$$ \rho^* + 2\sigma \leq \kappa + ( \max( \rho+1, 2\rho, 5\rho/4+\tau/2)+ \max( \rho'+1, 2\rho', 5\rho'/4+\tau/2) )/2 + 1$$
for some $0 \leq \kappa \leq \rho$ with
$$ \kappa + \rho' \leq 2\rho$$
$$ 2\kappa + \rho' \leq \rho^*$$
In particular,
$$ 2\kappa + 5\rho'/4 \leq 3\rho^*/4 + \rho$$
and the claim follows after moving the $\kappa$ inside the second maximum and performing some algebra.
\end{proof}

\begin{corollary}[Simplified Heath-Brown relation]\label{hb-energy-simp}  If $(\sigma,\tau,\rho,\rho^*,s) \in \Energy$ and $\tau \leq 3/2$, then
$$ \rho^* \leq \max(3 \rho + 1-2\sigma, \rho +4-4\sigma, 5\rho/2 + (3-4\sigma)/2).$$
\end{corollary}

\literature
\code{add_lver_heath_brown_1979b2()}

This result essentially appears as \cite[Lemma 3]{heathbrown_zero_1979}.

\begin{proof}\uses{hbt} Apply the previous result.  For $\tau \leq 3/2$ we observe that $5\rho/4+\tau/2$ is less than $5\rho/4 + 3/4$, which is a convex combination of $\rho+1$ and $2\rho$.  Similarly $3\rho^*/4+\rho+\tau/2$ is less than $3\rho^*/4+\rho+3/4$, which is a convex combination of $\rho^*+1$ and $4\rho$. We conclude that
$$ \rho^* \leq  1-2\sigma + \max(\rho+1, 2\rho)/2 + \max(\rho^*+1, 4\rho)/2.$$
Thus $\rho^*$ is less than one of
$$ 1-2\sigma + (\rho+\rho^*+2)/2, 1-2\sigma + (5\rho+1)/2, 1-2\sigma + (2\rho+\rho^*+1)/2, 1-2\sigma + (6\rho)/2;$$
solving for $\rho^*$, we conclude
$$ \rho^* \leq \max( 4-4\sigma + \rho, (3-4\sigma)/2 + 5\rho/2, 3-4\sigma + 2\rho, 1-2\sigma + 3\rho).$$
But since $\sigma \geq 1/2$, $3-4\sigma + 2\rho$ is less than $5/2-3\sigma + 2\rho$, which is the mean of $4-4\sigma+\rho$ and $1-2\sigma+3\rho$. Thus
$$ \rho^* \leq \max( 4-4\sigma + \rho, (3-4\sigma)/2 + 5\rho/2, 1-2\sigma + 3\rho),$$
which gives the claim.
\end{proof}


\begin{lemma}[Second Heath-Brown relation]\label{hbt-2}\uses{lv-edef} If $(\sigma,\tau,\rho,\rho^*,s) \in \Energy$ then
$$ \rho \leq \max( 2-2\sigma, \rho^*/4 + \max(\tau/4 + k(3-4\sigma)/4, k\tau/4 + k(1-2\sigma)/2))$$
for any positive integer $k$.
\end{lemma}

\literature
\code{add_lver_heath_brown_1979c(K)}

\begin{proof} By definition, there exists a large value pattern $(N,T,V,(a_n)_{n \in [N,2N]},J,W)$ with $N>1$ unbounded, $T = N^{\tau+o(1)}$, $|W| = N^{\rho+o(1)}$, $E_1(W) = N^{\rho^*+o(1)}$ and $S(N,W) = N^{s+o(1)}$.   From \cite[Lemma 4]{heathbrown_zero_1979}, we have
$$ |W| \ll T^\eps \left( N^{2-2\sigma} + E_1(W)^{1/4} (T^{1/4} N^{k(3-4\sigma)/4} + T^{k/4} N^{k(1-2\sigma)/2})\right)$$
for any fixed $\eps>0$, giving the claim.
\end{proof}

\begin{lemma}[Guth-Maynard relation]\label{gm-1}\uses{lv-edef}  If $(\sigma,\tau,\rho,\rho^*,s) \in \Energy$ then
$$ \rho \leq \max(2-2\sigma, 1-2\sigma + \max(S_1, S_2, S_3)/3)$$
where $S_1, S_2, S_3$ are real numbers with
$$ S_1 \leq -10,$$
$$ S_2 \leq \max(2+2\rho, \tau+1+(2-1/k) \rho, 2 + 2\rho + (\tau/2 - 3\rho/4)/k )$$
for any positive integer $k$ and
$$ S_3 \leq 2\tau + \rho/2 + \rho^*/2$$
and also
$$ S_3 \leq \max( 2\tau + 3\rho/2, \tau+1+\rho/2+\rho^*/2).$$
\end{lemma}

\literature
\code{add_lver_guth_maynard_2024a()}

\begin{proof} By definition, there exists a large value pattern $(N,T,V,(a_n)_{n \in [N,2N]},J,W)$ with $N>1$ unbounded, $T = N^{\tau+o(1)}$, $|W| = N^{\rho+o(1)}$, $E_1(W) = N^{\rho^*+o(1)}$ and $S(N,W) = N^{s+o(1)}$.  By \cite[Proposition 4.6, (5.5)]{guth-maynard}, one may bound
$$ |W| \ll N^{2-2\sigma} + N^{1-2\sigma} (S_1 + S_2 + S_3)^{1/3}$$
for three expressions $S_1, S_2, S_3$ defined after \cite[(5.5)]{guth-maynard}.  From \cite[Proposition 5.1]{guth-maynard} we have
$$ S_1 \ll T^{-10}.$$
From \cite[Proposition 6.1]{guth-maynard} we have
$$ S_2 \ll T^{o(1)} ( N^2 |W|^2 + T N |W|^{2-1/k} + N^2 |W|^2 (\frac{T^{1/2}}{|W|^{3/4}})^{1/k} ).$$
From \cite[Proposition 8.1]{guth-maynard} we have
$$ S_3 \ll T^{2+o(1)} |W|^{1/2} E_1(W)^{1/2}$$
while from \cite[Proposition 10.1]{guth-maynard} we have
$$ S_3 \ll T^{2+o(1)} |W|^{3/2} + T^{1+o(1)} N |W|^{1/2} E_1(W)^{1/2}.$$
Combining all these bounds, we obtain the claim.
\end{proof}

\begin{lemma}[Second Guth-Maynard relation]\cite[Lemma 1.7]{guth-maynard}  If $(\sigma,\tau,\rho,\rho^*,s) \in \Energy$ then
$$ \rho^* \leq \rho + s - 2\sigma.$$
In particular, from Lemma \ref{hb-double} we see for $\tau \leq 3/2$ that
$$ \rho^* \leq \max(3\rho+1-2\sigma, 2\rho+2-2\sigma).$$
\end{lemma}

\literature
\code{add_lver_guth_maynard_2024b()}

\begin{proof} By definition, we can find a large value pattern $(N,T,V,(a_n)_{n \in [N,2N]},J,W)$ with $N>1$ unbounded, $T = N^{\tau+o(1)}$, $V = N^{\sigma+o(1)}$, $|W| = N^{\rho+o(1)}$, $E_1(W) = N^{\rho^*+o(1)}$, and $S(N,W) = N^{s+o(1)}$. From \eqref{energy-triv} one has
$$    V^2 E_1(W) \ll T^{o(1)} |W| S(N,W) + T^{-50}.$$
Since $W$ is non-empty, $E_1(W) \geq 1$, and $V \gg 1$, so the $T^{-50}$ error can be discarded.  The claim then follows.
\end{proof}


\begin{lemma}[Third Guth-Maynard relation]\label{gm-3}\uses{lv-edef}  If $(\sigma,\tau,\rho,\rho^*, s) \in \Energy$ and $1 \leq \tau \leq 4/3$, then
    $$ \rho^* \leq \max(\rho+4-4\sigma, 21\rho/8+\tau/4+1-2\sigma, 3\rho+1-2\sigma).$$
\end{lemma}

\literature
\code{add_lver_guth_maynard_2024c()}

\begin{proof} By definition, there exists a large value pattern $(N,T,V,(a_n)_{n \in [N,2N]},J,W)$ with $N>1$ unbounded, $T = N^{\tau+o(1)}$, $|W| = N^{\rho+o(1)}$, $E_1(W) = N^{\rho^*+o(1)}$ and $S(N,W) = N^{s+o(1)}$.  Applying \cite[Proposition 11.1]{guth-maynard}, we conclude that
    $$ E_1(W) \ll T^{o(1)} ( |W| N^{4-4\sigma} + |W|^{21/8} T^{1/4} N^{1-2\sigma} + |W|^3 N^{1-2\sigma} ),$$
giving the claim.
\end{proof}

We can put this all together to prove the Guth--Maynard large values theorem.

\begin{theorem}[Guth--Maynard large values theorem, again]\label{guth-maynard-lvt}\cite[Theorem~1.1]{guth-maynard} One has
    $$ \LV(\sigma,\tau) \leq \max(2-2\sigma, 18/5 - 4 \sigma, \tau + 12/5 - 4\sigma).$$
\end{theorem}

\literature
\code{add_guth_maynard_large_values_estimate()}

\begin{proof}\uses{l2-mvt,huxley-lv, gm-1, gm-3, montgomery-subdivide} For $\sigma \leq 7/10$ this follows from Lemma \ref{l2-mvt}, and for $\sigma \geq 8/10$ it follows from Lemma \ref{huxley-lv}.  Thus we may assume that $7/10 \leq \sigma \leq 8/10$.  By subdivision (Lemma \ref{montgomery-subdivide}) it then suffices to treat the case $\tau = 6/5$, that is to say to show that
    $$ \rho \leq 18/5-4\sigma$$
whenever $(\sigma,\tau,\rho,\rho^*,s) \in \Energy$ with $\tau=6/5$ and $7/10 \leq \sigma \leq 8/10$.

Applying Lemma \ref{gm-1} and discarding the very negative $S_1$ term, we have
$$ \rho \leq \max(2-2\sigma, 1-2\sigma + \max(S_2, S_3)/3)$$
where $S_2, S_3$ are real numbers with
$$ S_2 \leq \max(2+2\rho, \tau+1+(2-1/k) \rho, 2 + 2\rho + (\tau/2 - 3\rho/4)/k )$$
for any positive integer $k$ and
$$ S_3 \leq 2\tau + \rho/2 + \rho^*/2$$
and also
$$ S_3 \leq \max( 2\tau + 3\rho/2, \tau+1+\rho/2+\rho^*/2).$$
From the latter bound and Lemma \ref{gm-3}, one has
$$ S_3 \leq \max( 2\tau+3\rho/2, \tau+\rho+3-2\sigma, \tau+2\rho+3/2-\sigma, 9\tau/8+29\rho/16 + 3/2-\sigma).$$
Inserting this and the $S_2$ bound (with $k=4$) into the bound for $\rho$ and simplifying (using $\tau=6/5$), we eventually obtain
the desired bound $\rho \leq 18/5-4\sigma$.
\end{proof}

Now we turn to another application of double zeta sums to large value theorems.

    \begin{theorem}[Bourgain large values theorem]\label{bourgain-lvt}\uses{lv-def} \cite{bourgain_large_2000} Let $1/2 < \sigma < 1$ and $\tau > 0$, and let $\rho := \LV(\sigma,\tau)$.  Let $\alpha_1, \alpha_2 \geq 0$ be real numbers.  Then either
    \begin{equation}\label{rho1}
     \rho \leq \max( \alpha_2 + 2 - 2 \sigma, -\alpha_2 + 2\tau+4-8\sigma, -2\alpha_1 + \tau + 12 - 16 \sigma)
    \end{equation}
    or else there exists $s \geq 0$ such that
    \begin{equation}\label{rs}
    \begin{split}
         &\frac{1}{2}\max(\rho+2, 2\rho+1, 5\rho/4 + \tau/2 + 1) + \frac{1}{2}\max(s+2, 2s+1, 5s/4 + \tau/2 + 1) \geq \\
            &\qquad\max( -2\alpha_1 + 2\sigma + s + \rho, -\alpha_1 - \alpha_2/2 + 2\sigma + s/2 + 3\rho/2).
    \end{split}
    \end{equation}
    \end{theorem}

    \begin{proof}  By Definition \ref{lv-def}, we can find a large value pattern $(N,T,V,(a_n)_{n \in [N,2N]},J,R)$ with $N>1$ unbounded, $N \geq 1$, $T = N^{\tau+o(1)}$, $|R| = N^{\rho+o(1)}$, $V = N^{\sigma+o(1)}$; we use $R$ here instead of $W$ to be consistent with the notation from \cite{bourgain_large_2000}.  Now set $\delta_1 := N^{-\alpha_1}$, $\delta_2 := N^{-\alpha_2}$.  From \cite[(4.41), (4.42)]{bourgain_large_2000}, one has the inequality
        $$ |R| \leq |R^{(1)}| + |R^{(2)}|$$
        for certain sets $R^{(1)}$ and $R^{(2)}$ with the former set obeying the bound
    $$ |R^{(1)}| \ll \delta_2^{-1} N^2 V^{-2} + \delta_2 T^2 N^4 V^{-8} + \delta_1^2 T N^{12} V^{-16}.$$
    Hence, we either have
    $$ |R| \ll \delta_2^{-1} N^2 V^{-2} + \delta_2 T^2 N^4 V^{-8} + \delta_1^2 T N^{12} V^{-16}$$
    which implies \eqref{rho1}, or else
    \begin{equation}\label{rr}
        |R| \ll |R^{(2)}|.
    \end{equation}
    Henceforth we assume that \eqref{rr} holds. From \cite[(4.53), (4.54)]{bourgain_large_2000} we may upper bound
    \begin{equation}\label{Exp-1}
         T^{-\eps} \delta' (\delta'')^2 V^2 |S| |R^{(2)}| + T^{-\eps} \delta_1 V^2 |S|^{1/2} \sum_\alpha |R_\alpha|^{3/2}
    \end{equation}
    by
    \begin{equation}\label{Exp-2}
        \ll T^\eps S(N,R^{(2)})^{1/2} S(N,S)^{1/2}
    \end{equation}
    for arbitrarily small fixed $\eps$, some $\delta',\delta''>0$ with $\delta' > T^{-\eps} (\delta_1/\delta'')^2$ (see \cite[(4.37)]{bourgain_large_2000}), some set $S$ (which will be non-empty by \cite[(4.47)]{bourgain_large_2000}), and some sets $R_\alpha$ defined in \cite[(4.39)]{bourgain_large_2000}, where the double zeta sums $S(N,W)$ are defined in \eqref{snw}.  Applying Lemma \ref{hb-double}, the latter expression is bounded by
    $$ \ll T^\eps (|R|N^2 + |R|^2 N + |R|^{5/4} T^{1/2} N)^{1/2} (|S| N^2 + |S|^2 N + |S|^{5/4} T^{1/2} N)^{1/2}.$$
    Meanwhile, from \cite[(4.57)]{bourgain_large_2000}, the expression \eqref{Exp-1} is bounded from below by
    $$ \gg T^{-2\eps} (\delta_1^2 V^2 |S| |R| + \delta_1 \delta_2^{1/2} V^2 |S|^{1/2} |R|^{3/2}).$$
    After passing to a subsequence, we can ensure that $|S| = N^{s+o(1)}$ for some $s > 0$.
    Combining these bounds and writing all expressions as powers of $N$, we obtain the claim (after sending $\eps \to 0$).
    \end{proof}

    \begin{corollary}[Bourgain large values theorem, simplified version]\label{borg-lv-simp}\uses{lv-def} \cite[Lemma 4.60]{bourgain_large_2000} Let the notation be as above, but additionally assume $\rho \leq \min(1, 4-2\tau)$.  Then
    $$ \rho \leq \max( \alpha_2 + 2 - 2 \sigma, \alpha_1+\alpha_2/2 + 2-2\sigma, -\alpha_2 + 2\tau+4-8\sigma, -2\alpha_1 + \tau + 12 - 16 \sigma, 4\alpha_1 + 2+\max(1,2\tau-2)-4\sigma).$$
    \end{corollary}

    In \cite{bourgain_large_2000} this bound is only established in the case $\tau \leq 3/2$ (in which case the condition on $\rho$ simplifies to $\rho \leq 1$, and the final term $4\alpha_1 + 2+\max(1,2\tau-2)-4\sigma$ simplifies to $4\alpha_1 + 3 -4\sigma$), but the argument extends to the $\tau > 3/2$ case without significant difficulty.

    \begin{proof}\uses{bourgain-lvt}  With $\rho \leq \min(1,4-2\tau)$, $5\rho/4+\tau/2+1$ and $2\rho+1$ are both bounded by $\rho+2$, hence
    $$ \max(\rho+2, 2\rho+1, 5\rho/4 + \tau/2 + 1) = \rho+2.$$
    Furthermore, $5s/4+\tau+1$ is a convex combination of $s+2$ and $2s + 2\tau-2$, hence
    $$\max(s+2, 2s+1, 5s/4 + \tau/2 + 1) \leq \max(s+2, 2s + \max(1,2\tau-2)).$$
    Thus \eqref{rs} simplifies to
    $$(\rho+2)/2 + \max(s+2, 2s+\max(1,2\tau-2))/2 \geq
        \max( -2\alpha_1 + 2\sigma + s + \rho, -\alpha_1 - \alpha_2/2 + 2\sigma + s/2 + 3\rho/2).$$
    Thus either
    $$(\rho+2)/2 + (s+2)/2 \geq -\alpha_1 - \alpha_2/2 + 2\sigma + s/2 + 3\rho/2$$
    or
    $$(\rho+2)/2 + (2s+\max(1,2\tau-2))/2 \geq  -2\alpha_1 + 2\sigma + s + \rho.$$
    In both cases we may eliminate $s$ and solve for $\rho$ to obtain
    $$ \rho \leq \alpha_1 + \alpha_2/2 + 2 - 2 \sigma $$
    or
    $$ \rho \leq 4\alpha_1 + 2 + \max(1,2\tau-2) - 4 \sigma,$$
    giving the claim.
    \end{proof}

    With the aid of computer assistance, one is able to produce an optimized version of the above large values theorem. We have

    \begin{corollary}[Bourgain large values theorem, optimized version]\label{borg-lv-opt}\uses{lv-def} For each row $(\rho_0, \alpha_1, \alpha_2, \mathcal{S})$ of Table \ref{optimised_bourgain_lve}, one has
    \[
    \rho \le \rho_0(\sigma, \tau),\qquad (\sigma, \tau) \in \mathcal{S}.
    \]
    \end{corollary}

    \derived
    \code{prove_bourgain_large_values_theorem()}

    \begin{proof}\uses{bourgain-lvt}
    Follows from substituting the specified values of $\alpha_1$ and $\alpha_2$ and a routine calculation.
    \end{proof}

    \begin{table}[ht]
    \def\arraystretch{1.8}
    \centering
    \caption{Bounds on $\LV(\sigma, \tau)$ for $1/2 \le \sigma \le 1$ and $\tau \ge 1$}
    \begin{tabular}{|c|c|c|c|}
    \hline
    $\rho_0(\sigma, \tau)$ & $\alpha_1$ & $\alpha_2$ & $\mathcal{S}$\\
    \hline
    $\frac{16}{3} - \frac{20}{3}\sigma + \frac{1}{3}\tau$ & $\frac{10}{3} - \frac{14}{3}\sigma + \frac{\tau}{3}$ & $0$ &
    $\begin{aligned}
    -1 + \tau \ge 0,\\
    10 - 14\sigma + \tau \ge 0,\\
    4 + 4\sigma - 5\tau \ge 0,\\
    -11 + 16\sigma - \tau \ge 0.
    \end{aligned}$\\
    \hline
    $5 - 7\sigma + \frac{3}{4}\tau$ & $\frac{7}{2} - \frac{9}{2}\sigma + \frac{\tau}{8}$ & $-1 - \sigma + \frac{5}{4}\tau$ &
    $\begin{aligned}
    8 - 8\sigma - \tau \ge 0,\\
    -16 + 20\sigma + \frac{1}{3}\tau \ge 0,\\
    -6 + 10\sigma - \frac{7}{6}\tau \ge 0,\\
    -4 - 4\sigma + 5\tau \ge 0.
    \end{aligned}$\\
    \hline
    $3 - 5\sigma + \tau$ & $\frac{9}{2} - \frac{11}{2}\sigma$ & $1 - 3\sigma + \tau$ &
    $\begin{aligned}
    -8+8\sigma+\tau &\ge 0,\\
    2-6\sigma+2\tau &\ge 0,\\
    -10+14\sigma-\frac{2}{3}\tau &\ge 0,\\
    6-2\sigma-2\tau &\ge 0.
    \end{aligned}$
    \\
    \hline
    $-4\sigma + 2\tau$ & $0$ & $1 - 3\sigma + \tau$ & $\begin{aligned}
    -6+2\sigma+2\tau &\ge 0,\\
    -12 + 12\sigma + \tau &\ge 0,\\
    1-\sigma &\ge 0.
    \end{aligned}$\\
    \hline
    $8 - 12\sigma + \frac{4}{3}\tau$ & $ 2 - 2\sigma - \frac{1}{6}\tau$ & $6 - 10\sigma + \frac{4}{3}\tau$ & $\begin{aligned}
    15-21\sigma+\tau \ge 0,\\
    12 - 12\sigma - \tau \ge 0,\\
    -\frac{3}{2} + \tau \ge 0,\\
    6-10\sigma+\frac{7}{6}\tau \ge 0.\\
    \end{aligned}$\\
    \hline
    $2 - 2\sigma$ & $0$ & $0$ & $\begin{aligned}
    1-\sigma \ge 0,\\
    -10+14\sigma-\tau \ge 0,\\
    -1+\tau \ge 0,\\
    -2+6\sigma-2\tau \ge 0.
    \end{aligned}$\\
    \hline
    $9 - 12\sigma + \frac{2}{3}\tau$ & $\frac{3}{2} - 2\sigma + \frac{1}{6}\tau$ & $11 - 16\sigma + \tau$ & $\begin{aligned}
    \frac{3}{2} - \tau \ge 0,\\
    -\frac{1}{2} + \sigma \ge 0,\\
    -1 + \tau \ge 0,\\
    11 - 16\sigma + \tau \ge 0,\\
    16 - 20\sigma - \frac{1}{3}\tau \ge 0.
    \end{aligned}$\\
    \hline
    \end{tabular}
    \label{optimised_bourgain_lve}
    \end{table}

    The preprint of Kerr \cite{kerr} contains additional large value theorems:

    \begin{lemma}[Kerr large values theorem]\label{kerr-thm}\uses{lv-def}\
        \begin{itemize}
        \item[(i)]\cite[Theorem 2]{kerr} Let $3/4 < \sigma \leq 1$, $0 \leq \tau \leq 3/2$, and $0 \leq \rho \leq \LV(\sigma,\tau), 1$ be fixed.  Then for any fixed $\alpha \geq 0$, one has
        $$ \rho \leq \max( 2-2\sigma+\alpha, 2\tau+4-8\sigma-\alpha, \tau/3+16/3 -20\sigma/3 + \alpha/3, 2\tau/3+9-12\sigma).$$
        \item[(ii)] \cite[Theorem 3]{kerr} Under the same hypotheses as (i), we have for any fixed integer $k \geq 2$ obeying
        $-\alpha < 4k\sigma -(\tau+3k-1)$ and $-\alpha < 1 + \frac{1}{k-1} - \tau$ that
        $$ \rho \leq \max(2-2\sigma+\alpha, \tau/3+(3k+4)/3-(4k+4)\sigma/3 + \alpha/3).$$
        \item[(iii)] \cite[Theorem 4]{kerr} If $25/32 < \sigma \leq 1$, $\tau \geq 0$, and $0 \leq \rho \leq \LV(\sigma,\tau), 1, 4-2\tau$ are fixed, then for any fixed $\alpha$ with
        $$ 26 - 32 \sigma - \tau < - \alpha < 16\sigma -11-\tau$$
        one has
        $$ \rho \leq \max( 2-2\sigma+\alpha, 2\tau+4-8\sigma-\alpha, -\tau+8-8\sigma+2\alpha, 10-12\sigma+2\alpha/3).$$
        \item[(iv)] \cite[Theorem 5]{kerr} If $1/2 < \sigma \leq 1$, $0 \leq \tau \leq 3/2$, and $0 \leq \rho \leq \LV(\sigma,\tau), 1$ are fixed, then for any fixed $\alpha$ with
        $$ - \alpha < -\tau + 8\sigma - 5$$
        one has
        $$ \rho \leq \max(2-2\sigma+\alpha, 4\tau/3 +23/3 - 12\sigma - 2\alpha/3, 2\tau/3 + 14/3 - 20/3 ).$$
    \end{itemize}
    \end{lemma}
