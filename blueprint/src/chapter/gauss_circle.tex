\chapter{The Gauss circle problem and its generalizations}\label{gauss-circle-chapter}

\unintegrated

For any fixed integer $k \ge 2$ and unbounded $R$, consider the problem of estimating the number of integer lattice points contained in $B_k(R)$, a $k$-dimensional ball of radius $R$:
\[
S_k(R) := \# \mathbb{Z}^k \cap B_k(R) = \# \{x \in \mathbb{Z}^k: |x| \le R\}.
\]
Equivalently, $S_k(R)$ may be written as the partial sum 
\[
S_k(R) = \sum_{n \le R^{2}}r_k(n)
\]
where $r_k(n)$ counts the number of integer solutions to the equation $x_1^2 + \cdots + x_k^2 = n$.

By considering the volume of a $k$-dimensional ball of radius $R$, one has the asymptotic
\[
S_k(R) \sim \operatorname{Vol}(B_k(R)) = \frac{\pi^{k/2}}{\Gamma(k/2 + 1)}R^k.
\]
The generalized Gauss circle problem concerns estimating the error term in this approximation.

\begin{definition}
For fixed integer $k \ge 2$, define $\theta^{\operatorname{Gauss}}_{k}$ as the least (fixed) exponent for which
\[
P_k(R) := S_k(R) - \operatorname{Vol}(B_k(R)) \ll R^{\theta^{\operatorname{Gauss}}_{k} + o(1)}.
\]
\end{definition}

It is conjectured that 

\begin{conjecture}\label{gauss-circle-conj}
One has $\theta^{\operatorname{Gauss}}_{2} = 1/2$ and $\theta^{\operatorname{Gauss}}_{k} = k - 2$ for integer $k \ge 3$.
\end{conjecture}

This conjecture is known to hold for $k \ge 4$. 

\begin{theorem}
For integer $k \ge 4$, one has $\theta^{\operatorname{Gauss}}_{k} = k - 2$.
\end{theorem}

The remaining open cases are $k = 2, 3$. For such cases the following lower-bounds on $\theta^{\operatorname{Gauss}}_{k}$ are known:

\begin{theorem}[Hardy \cite{}]\label{gauss-circle-lower-23}
One has $\theta^{\operatorname{Gauss}}_{2} \ge 1/2$ and $\theta^{\operatorname{Gauss}}_{3} \ge 1$.
\end{theorem}

In light of \Cref{gauss-circle-conj} and \Cref{gauss-circle-lower-23}, in the rest of this chapter we shall focus on upper bounds on $\theta^{\operatorname{Gauss}}_{k}$ for $k = 2, 3$.