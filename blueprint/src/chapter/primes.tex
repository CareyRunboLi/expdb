\chapter{Distribution of primes: short ranges}\label{primes-sec}

Recall that $\Lambda$ is the von Mangoldt function, and that the prime number theorem asserts that
$$ \sum_{n \leq x} \Lambda(n) = x + o(x)$$
for unbounded $x$.  If $p_n$ denotes the $n^{\mathrm{th}}$ prime, the prime number theorem is also equivalent to
$$ p_n = (1+o(1)) n \log n$$
for unbounded $n$.

We now consider local versions of the prime number theorem.

\begin{definition}[Prime number theorem in short interval exponents]\label{pnt-ap}
    \begin{itemize}
        \item[(i)] We let $\PNTALL$ denote the least exponent with the following property: if $\eps > 0$ is fixed, and $x$ is unbounded, then
        $$ \sum_{x \leq n < x+y} \Lambda(n) = y + o(y)$$
        whenever $x^{\PNTALL + \eps} \leq y \leq x^{1-\eps}$.
        \item[(ii)] We let $\PNTAA$ denote the least exponent with the following property: if $\eps > 0$ is fixed, and $X$ is unbounded, then we have
        $$ \int_X^{2X} |\sum_{x \leq n < x+y} \Lambda(n)-y|\ dx = o(Xy)$$
        whenever $X^{\PNTAA + \eps} \leq y \leq X^{1-\eps}$.
        \item[(iii)] We let $\GAPMAX$ denote the least exponent such that, if $p_n$ denotes the $n^{\mathrm{th}}$ prime, that
        $$ p_{n+1} - p_n \ll n^{\GAPMAX+o(1)} = p_n^{\GAPMAX+o(1)}$$
        as $n \to \infty$.
        \item[(iv)] We let $\GAPSQUARE$ denote the least exponent such that
        $$ \sum_{p_n \leq x} (p_{n+1}-p_n)^2 \ll x^{\GAPSQUARE+o(1)}$$
        as $x \to \infty$.
        \item[(v)] We let $\GAPAA$ denote the least exponent such that for every $\eps>0$, the intervals $[n, n^{\GAPAA+\eps}]$ contain a prime for a density $1$ set of natural numbers $n$.
    \end{itemize}
\end{definition}

\begin{lemma}[Trivial bounds]\label{pnt-triv}\uses{pnt-ap}
    We have
    $$ 0 \leq \GAPAA \leq \PNTAA, \GAPMAX \leq \PNTALL \leq 1$$
    and $1 \leq \GAPSQUARE \leq 1+\GAPMAX$.
\end{lemma}

\begin{proof} These are all immediate, after noting from the prime number theorem that $\sum_{p_n \leq x} p_{n+1} - p_n = x^{1+o(1)}$.
\end{proof}

The Cram\'er random model \cite{cramer} predicts

\begin{conjecture}[Prime gap conjecture]\uses{pnt-ap} $\PNTALL = 0$, and hence (by Lemma \ref{pnt-triv}) $\GAPAA = \PNTAA = \GAPMAX=0$ and $\GAPSQUARE=1$.
\end{conjecture}

We note that the results of Maier \cite{maier-primes} show that there is some deviation from the prime number theorem at very small scales (of order $\log^{O(1)} x$), but this does not directly affect the exponents discussed here due to the epsilons in our definitions.

A basic connection with zero density exponents is

\begin{proposition}[Zero density theorems and prime gaps]\label{prime-gap}  Let
\begin{equation}\label{A-def}
    \|\A\|_\infty := \sup_{1/2 \leq \sigma \leq 1} A(\sigma).
\end{equation}
Then
    $$ \PNTALL \leq 1 - \frac{1}{\|\A\|_\infty}$$
    and
    $$ \PNTAA \leq 1 - \frac{2}{\|\A\|_\infty}.$$
\end{proposition}

\begin{proof} See for instance \cite[\S 13.2]{guth-maynard}.
\end{proof}

\begin{corollary}[Ingham-Huxley bound]\uses{pnt-ap}  We have $\PNTALL \leq \frac{7}{12}$ and $\PNTAA \leq \frac{1}{6}$.
\end{corollary}

\begin{proof}\uses{prime-gap, thm:ingham_zero_density2, huxley-bound}  From Theorem \ref{thm:ingham_zero_density2} and Theorem \ref{huxley-bound} one as $\|\A\|_\infty \leq 12/5$, and the claim now follows from Proposition \ref{prime-gap}.
\end{proof}

\begin{corollary}[Ingham-Guth-Maynard bound]\cite{guth-maynard}\uses{pnt-ap}  We have $\PNTALL \leq \frac{17}{30}$ and $\PNTAA \leq \frac{2}{15}$.
\end{corollary}

These are currently the best known upper bounds on $\PNTALL$ and $\PNTAA$.

\begin{proof}\uses{prime-gap, thm:ingham_zero_density2, guth-maynard-density}  From Theorem \ref{thm:ingham_zero_density2} and Theorem \ref{guth-maynard-density} one as $\|\A\|_\infty \leq 30/13$, and the claim now follows from Proposition \ref{prime-gap}.
\end{proof}

\begin{corollary}\uses{pnt-ap}  The density hypothesis implies that $\PNTALL \leq 1/2$ and $\PNTAA = 0$.
\end{corollary}

The current unconditional best bound on $\GAPMAX$ is

\begin{theorem}\cite{li_number_2025}\label{bhp-thm}\uses{pnt-ap} We have $\GAPMAX \leq 13/25 = 0.52$.
\end{theorem}

Historical bounds on $\GAPMAX$ are summarized in the following table:

\begin{table}[ht]
    \caption{Historical upper bounds on $\GAPMAX$.}
    \centering
    \renewcommand{\arraystretch}{1.2}
    \begin{tabular}{|c|c|}
    \hline
    Reference & Upper bound \\
    \hline
    Hoheisel (1930) \cite{hoheisel_1930} & $1 - \frac{1}{33000} = 0.999\dots$ \\
    \hline
    Heilbronn (1933) \cite{heilbronn_1933} & $1 - \frac{1}{250} = 0.996$ \\
    \hline
    Ingham (1937) \cite{ingham_difference_1937} & $\frac{5}{8} = 0.625$ \\
    \hline
    Montgomery (1969) \cite{montgomery_1969} & $\frac{3}{5} = 0.6$ \\
    \hline
    Huxley (1972) \cite{Huxley} & $\frac{7}{12} = 0.5833\dots$ \\
    \hline
    Iwaniec--Jutila (1979)\cite{iwaniec-jutila} & $\frac{13}{23} = 0.5652\dots$ \\
    \hline
    Heath-Brown--Iwaniec (1979) \cite{heathbrown_iwaniec_1979}, Lou--Yao (1993) \cite{lou_yao_number_1993}& $\frac{11}{20} = 0.55$ \\
    \hline
    Pintz (1981) \cite{pintz_1981} & $\frac{17}{31} = 0.5483\dots$ \\
    \hline
    Iwaniec--Pintz (1984) \cite{iwaniec-pintz} & $\frac{23}{42} = 0.5476\dots$\\
    \hline
    Mozzochi (1986) \cite{mozzochi-consecutive} & $\frac{1051}{1920} = 0.5473\dots$ \\
    \hline
    Lou--Yao (1984) \cite{LouYao3564} & $\frac{35}{64} = 0.5469\dots$\\
    \hline
    Lou--Yao (1992) \cite{lou-yao-chebychev} & $\frac{6}{11} = 0.5454\dots$\\
    \hline
    Baker-Harman (1996) \cite{baker-harman} & $\frac{107}{200} = 0.535$\\
    \hline
    Baker--Harman--Pintz (2001) \cite{baker-harman-pintz} & $\frac{21}{40} = 0.525$ \\
    \hline
    R. Li (2025) \cite{li_number_2025} & $\frac{13}{25} = 0.52$\\
    \hline
    \end{tabular}
    \end{table}\label{gapmax-table}

Bounds on $\GAPAA$ are recorded in Table \ref{gapaa-table}.

\begin{table}[ht]
    \caption{Historical upper bounds on $\GAPAA$.}
    \centering
    \renewcommand{\arraystretch}{1.2}
    \begin{tabular}{|c|c|}
    \hline
    Reference & Upper bound \\
    \hline
    Selberg (1943) \cite{selberg_1943} & $\frac{19}{77}=0.2467\dots$\\
    \hline
    Montgomery (1971) \cite{montgomery_topics_1971} & $\frac{1}{5}=0.2$ \\
    \hline
    Huxley (1972) \cite{Huxley} & $\frac{1}{6} = 0.1666\dots$ \\
    \hline
    Harman (1982) \cite{harman_1982} & $\frac{1}{10} = 0.1$ \\
    \hline
    Harman (1983) \cite{harman_1983}, Heath-Brown (1983) \cite{heathbrown_sieve_1984}  & $\frac{1}{12} = 0.0833\dots$ \\
    \hline
    Jia (1995) \cite{jia_goldbach_1994} & $\frac{1}{13} = 0.0769\dots$ \\
    \hline
    Lou--Yao (1985) \cite{lou_yao_1985} & $\frac{17}{227}= 0.0748\dots$ \\
    \hline
    H. Li (1995) \cite{li_goldbach_1995} & $\frac{2}{27} = 0.0740\dots$ \\
    \hline
    Jia (1995) \cite{jia_difference_1995}, Watt (1995) \cite{watt_short_1995} & $\frac{1}{14}=0.0714\dots$ \\
    \hline
    H. Li (1997) \cite{li_short_1997} & $\frac{1}{15}=0.0666\dots$ \\
    \hline
    Baker--Harman--Pintz (1997) \cite{baker-harman-pintz_goldbach_1997} & $\frac{1}{16} = 0.625$ \\
    \hline
    Wong (1996) \cite{wong_thesis_1996}, Jia (1996) \cite{jia_exceptional_1996}, Harman (2007) \cite{harman-book} & $\frac{1}{18} = 0.0555\dots$ \\
    \hline
    Jia (1996) \cite{jia_almost_all} & $\frac{1}{20} = 0.05$ \\
    \hline
    R. Li (2024) \cite{li_primes_almost_2024} & $\frac{2}{43} = 0.0465\dots$ \\
    \hline
    \end{tabular}
    \end{table}\label{gapaa-table}

    Historical bounds on $\GAPSQUARE$ are recorded in Table \ref{gapsquare-table}.

    \begin{table}[ht]
        \caption{Historical upper bounds on $\GAPSQUARE$.}
        \centering
        \renewcommand{\arraystretch}{1.2}
        \begin{tabular}{|c|c|}
        \hline
        Reference & Upper bound \\
        \hline
        Selberg (1943) \cite{selberg_1943} & $1$ (on RH)\\
        \hline
        Heath-Brown (1978) \cite{heath_brown_consecutive_I} & $\frac{4}{3} = 1.3333\dots$\\
        \hline
        Heath-Brown (1979) \cite{heath_brown_consecutive_II} & $\frac{7}{6} = 1.1666\dots$ (on LH) \\
        \hline
        Heath-Brown (1979) \cite{heath_brown_consecutive_II} & $\frac{1413}{1067} = 1.3242\dots$ \\
        \hline
        Heath-Brown (1979) \cite{heath_brown_consecutive_III} & $\frac{23}{18} = 1.2777\dots$ \\
        \hline
        Yu (1996) \cite{yu_differences_1996} & $1$ (on LH) \\
        \hline
        Peck (1996) \cite{peck_on_1996}, Maynard (2012) \cite{maynard_difference_2012} & $\frac{5}{4} = 1.25$ \\
        \hline
        Stadlmann (2022) \cite{stadlmann_mean_2022} & $\frac{123}{100} = 1.23$ \\
        \hline
    \end{tabular}
    \end{table}\label{gapsquare-table}

The following general bound on $\GAPSQUARE$ is known:

\begin{proposition}\label{gapsquare-from-a}\uses{pnt-ap, zero-def, zeroe-def}
    We have
    $$ \GAPSQUARE \leq \max\left( 2-\frac{2}{\|\A\|_\infty}, \sup_{1/2 \leq \sigma \leq 1} \max(\alpha(\sigma), \beta(\sigma))\right)$$
    where
    $$ \alpha(\sigma) := 4\sigma - 2 + 2 \frac{B(\sigma)(1-\sigma)-1}{B(\sigma)-A(\sigma)}$$
    and
    $$ \beta(\sigma) := 4\sigma - 2 + \frac{B(\sigma)(1-\sigma)-1}{A(\sigma)}$$
    where $A(\sigma), B(\sigma)$ are any upper bounds for $\A(\sigma), \A^*(\sigma)$ respectively.
\end{proposition}

\begin{proof} See \cite[Lemma 2]{heath_brown_consecutive_II}. We remark that this lemma allows $\sigma$ to range over $0 \leq \sigma \leq 1$ rather than $1/2 \leq \sigma \leq 1$, but it is easy to see that the contributions of the $0 \leq \sigma < 1/2$ cases are dominated by the $\sigma=1/2$ case.
\end{proof}

\code{compute_gap2()}


This proposition can be used to recover the following bounds on $\GAPSQUARE$:

\begin{corollary}\
    \begin{itemize}
    \item[(i)] Assuming the Riemann hypothesis, $\GAPSQUARE = 1$. (Selberg, 1943 \cite{selberg_1943})
    \item[(ii)] Assuming the Lindelof hypothesis, $\GAPSQUARE \leq 7/6$. (Heath-Brown, 1979 \cite{heath_brown_consecutive_II})
    \item[(iii)] Unconditionally, $\GAPSQUARE \leq 23/18$. (Heath-Brown, 1979 \cite{heath_brown_consecutive_III}).
\end{itemize}
\end{corollary}

\begin{proof}\uses{gapsquare-from-a, lindelof_implies_density, zeroe-lindelof, hb-energy-bound} For (i), we observe that $\|A\|_\infty=2$ and that one can take $A(\sigma)=B(\sigma)=\eps$ for any $\sigma>1/2$ and $\eps>0$, and $A(\sigma)=2$, $B(\sigma)=6$ for $\sigma=1/2$, and then the claim follows from Proposition \ref{gapsquare-from-a}.

For (ii), from Theorem \ref{lindelof_implies_density} we may take $A(\sigma)=2$ for $\sigma \leq 3/4$ and $A(\sigma)=\eps$ for $3/4 < \sigma \leq 1$ and any $\eps>0$, while from Theorem \ref{zeroe-lindelof} one can take $B(\sigma) = 8-4\sigma$ for $\sigma \leq 3/4$ and $B(\sigma)=\eps$ for $3/4 < \sigma \leq 1$.  The claim now follows from Proposition \ref{gapsquare-from-a} and a routine calculation.

Part (iii) follows from applying Proposition \ref{gapsquare-from-a} using the bounds from Theorem \ref{hb-energy-bound}, together and various bounds on $\A(\sigma)$; see \cite[Theorem 12.14]{ivic} for details.
\end{proof}

Two variants of $\GAPSQUARE$ are $\GAPLARGE$ and $\GAPLARGEEQ$, defined respectively as the least exponent for which
$$ \sum_{p_n \leq x: p_{n+1}-p_n \geq x^{1/2+\eps}} (p_{n+1}-p_n) \ll x^{\GAPLARGE+o(1)}$$
(for any fixed $\eps>0$ for unbounded $x \geq 1$) and
$$ \sum_{p_n \leq x: p_{n+1}-p_n \geq x^{1/2}} (p_{n+1}-p_n) \ll x^{\GAPLARGEEQ+o(1)}$$
(for unbounded $x \geq 1$).  The trivial bounds are

\begin{proposition}[Trivial bounds on large gaps]\label{trivial-large-gap}  One has $\GAPLARGE \le \GAPLARGEEQ$. If $\GAPMAX < 1/2$, then $\GAPLARGE = -\infty$.  In general, we have
    $$ \max(1/2,\GAPMAX) \leq \max(1/2, \GAPLARGE)$$
    and $\GAPLARGE \leq 1$.  Also $\GAPLARGE \leq \GAPSQUARE - 1/2$.
\end{proposition}

The proofs are routine and are omitted.  Historical bounds on $\GAPLARGE$ are recorded in Table \ref{gaplarge-table}.

\begin{table}[ht]
    \caption{Historical upper bounds on $\GAPLARGE$ and $\GAPLARGEEQ$.}
    \centering
    \renewcommand{\arraystretch}{1.2}
    \begin{tabular}{|c|c|c|}
    \hline
    Reference & Upper bound on $\GAPLARGE$ & Upper bound on $\GAPLARGEEQ$\\
    \hline
    Selberg (1943) \cite{selberg_1943} & $\frac{1}{2}=0.5$ (on RH) & \\
    \hline
    Wolke (1975) \cite{wolke_1975} & & $\frac{29}{30} = 0.966\dots$ \\
    \hline
    Cook (1979) \cite{cook_1979} & $\frac{85}{98} = 0.8673\dots$ & \\
    \hline
    Huxley (1980) \cite{huxley_large_1980} & $\frac{1759}{2134} = 0.8242\dots$ & \\
    \hline
    Huxley (1980) \cite{huxley_large_1980} & $\frac{3}{4} = 0.75$ (on LH) & \\
    \hline
    Iv\'ic (1979) \cite{ivic_sums_1979} & $\frac{215}{266} = 0.8082\dots$ & \\
    \hline
    Heath-Brown (1979) \cite{heath_brown_consecutive_III} & & $\frac{3}{4} = 0.75$ \\
    \hline
    Heath-Brown (1979) \cite{heath_brown_consecutive_II} & $\frac{5}{8} = 0.625$ & \\
    \hline
    Peck (1998) \cite{peck_differences_1998} & & $\frac{25}{36} = 0.6944\ldots$ \\
    \hline
    Matom\"{a}ki (2007) \cite{matomaki_large_2007} & & $\frac{2}{3} = 0.6666\ldots$ \\
    \hline
    Heath-Brown (2020) \cite{heath-brown_differences_2021} & & $\frac{3}{5} = 0.6$\\
    \hline
    J\"{a}rviniemi (2022) \cite{jarviniemi_large_2022} & & $\frac{57}{100} = 0.57$\\
    \hline
    \end{tabular}
\end{table}\label{gaplarge-table}

For any $0 < \theta < 1$, let $\PNTEXCEP(\theta)$ denote the least exponent $\mu$ such that for all unbounded $X$, one has $\sum_{x \leq n < x + x^\theta} \Lambda(n) = (1+o(1)) x^\theta$ for all $x \in [X,2X]$ outside of an exceptional set of measure $O(X^{\mu+o(1)})$.  Thus for instance $\PNTEXCEP(\theta)=-\infty$ for $\theta > \PNTALL$ (and $\PNTEXCEP(\theta) \geq 0$ for $\theta < \PNTALL$), and $\PNTEXCEP(\theta) < 1$ implies $\theta \geq \PNTAA$. The quantity $\PNTEXCEP(\theta)$ is clearly non-decreasing in $\theta$.

The following bounds are known:

\begin{lemma}[Bounds on $\mu$]\label{baz-bound}\
  \begin{itemize}
  \item[(i)]  \cite[Theorem 2(i)]{bazzanella-perelli}  For sufficiently small $\Delta>0$, we have $\PNTEXCEP(1/6 + \Delta) \leq 1 - c\Delta$ and $\PNTEXCEP(7/12-\Delta) \leq \frac{5}{8} + \frac{7}{4}\Delta + O(\Delta^2)$.
  \item[(ii)] \cite[Theorem 2(ii)]{bazzanella-perelli} Assuming RH, we have $\PNTEXCEP(\theta) \leq 1-\theta$ for $0 < \theta \leq 1/2$.
    \item[(iii)] \cite[Lemma 1]{bazzanella} We have
  $$
  \PNTEXCEP(\theta) \leq
  \begin{cases}
  \frac{3(1-\theta)}{2} & \frac{1}{2} < \theta \leq \frac{11}{21} \\
  \frac{47-42\theta}{35}& \frac{11}{21} < \theta \leq \frac{23}{42} \\
  \frac{36\theta^2-96\theta+55}{39-36\theta} & \frac{23}{42} < \theta \leq \frac{7}{12} \\
  \end{cases}
  $$
  Some further bounds were claimed in the region $1/6 < \theta \leq 1/2$, but unfortunately the arguments provided are incomplete (the claim (13) of that paper is not justified for $\theta \leq 1/2$).
  \item[(iv)] \cite{gafni-tao} For any $0 < \theta < 1$, one has
$$\PNTEXCEP(\theta) \leq \inf_{\eps>0} \sup_{\stackrel{0 \leq \sigma < 1}{\A(\sigma) \geq \frac{1}{1-\theta}-\eps}} \min(\mu_{2,\sigma}(\theta), \mu_{4,\sigma}(\theta))$$
where
$$ \mu_{2,\sigma}(\theta) := (1-\theta)(1-\sigma) \A(\sigma) + 2\sigma - 1$$
and
$$ \mu_{4,\sigma}(\theta) := (1-\theta)(1-\sigma) \A^*(\sigma) + 4\sigma-3.$$
  \end{itemize}
  \item[(v)] \cite[Theorem 2]{heath-brown_differences_2021} $\PNTEXCEP(1/2) \leq 3/5$.
\end{lemma}

\code{prime_excep()}



\section{Extremal values of prime gaps}

Consider now the problem of determining upper bounds on
\begin{equation}\label{eqn:small-prime-gaps}
H_1 := \liminf_{n\to\infty}(p_{n + 1} - p_n)
\end{equation}
as well as lower bounds on
\begin{equation}\label{eqn:large-prime-gaps}
G(X) := \max_{p_{n + 1} \le X}(p_{n + 1} - p_n).
\end{equation}
From the prime number theorem one expects $p_{n + 1} - p_n$ to be of size $\log p_n$ on average, so that
\begin{theorem}[Consequences of the prime number theorem]
One has
\[
\liminf_{n\to\infty}\frac{p_{n + 1} - p_n}{\log p_n} \le 1,\qquad G(X) \ge (1 + o(1)) \log X\quad (X \to \infty).
\]
\end{theorem}

However, $p_{n + 1} - p_n$ can be sometimes be much smaller or much larger than its average size. The following is a classical conjecture regarding small prime gaps.

\begin{conjecture}[Twin prime conjecture]
One has
\[
\liminf_{n\to\infty}(p_{n + 1} - p_n) = 2.
\]
\end{conjecture}

Since all sufficiently large primes are odd, the twin prime conjecture states that prime gaps achieve the smallest possible size, infinitely often. In the other direction, it is conjectured that
\begin{conjecture}[Cram\'er \cite{cramer}]
One has
\[
\limsup_{X \to \infty}\frac{G(X)}{(\log X)^2} = 1.
\]
\end{conjecture}
Note that by \Cref{bhp-thm} it is known that $G(X) \ll X^{0.52}$.

The current best known result concerning \eqref{eqn:small-prime-gaps} is
\begin{theorem}[Polymath 8b \cite{polymath_variants_2014}]
One has
\[
\liminf_{n\to\infty}(p_{n + 1} - p_n) \le 246.
\]
\end{theorem}

Sharper conditional bounds are also known.
\begin{theorem}[Maynard \cite{maynard_small_2015}]
Assuming the Elliott-Halberstam conjecture (EH), one has
\[
\liminf_{n\to\infty}(p_{n + 1} - p_n) \le 12.
\]
\end{theorem}
\begin{theorem}[Polymath 8b \cite{polymath_variants_2014}]
Assuming the Generalized Elliott-Halberstam conjecture (GEH), one has
\[
\liminf_{n\to\infty}(p_{n + 1} - p_n) \le 6.
\]
\end{theorem}
Historical progress towards this problem is recorded in \Cref{small-primegap-table}.

\begin{table}[ht]
    \caption{Historical progression of bounds related to \eqref{eqn:small-prime-gaps}.}
    \centering
    \renewcommand{\arraystretch}{2.2}
    \begin{tabular}{|c|c|c|}
    \hline
    Reference & Unconditional result & Assuming EH \\
    \hline
    Goldston--Pintz--Yıldırım (2009) \cite{goldston_primes_2009} & $\displaystyle\liminf_{n\to\infty}\frac{p_{n + 1} - p_n}{\log p_n} = 0$ & $H_1 \le 16$ \\
    \hline
    Goldston--Pintz--Yıldırım (2010) \cite{goldston_primes_2010} & $\displaystyle\liminf_{n\to\infty}\frac{p_{n + 1} - p_n}{(\log p_n)^{1/2}(\log\log p_n)^2} < \infty$ & \\
    \hline
    Pintz (2013) \cite{Pintz_3747} & $\displaystyle\liminf_{n\to\infty}\frac{p_{n + 1} - p_n}{(\log p_n)^{3/7}(\log\log p_n)^{4/7}} < \infty$ & \\
    \hline
    Zhang (2014) \cite{zhang_bounded_2014} & $H_1 < 7\cdot 10^7$ & \\
    \hline
    Polymath 8a (2014) \cite{castryck_new_2014} & $H_1 \le 4680$ & \\
    \hline
    Maynard (2015) \cite{maynard_small_2015} & $H_1 \le 600$ & $H_1 \le 12$ \\
    \hline
    Polymath 8b (2015) \cite{polymath_variants_2014} & $H_1 \le 246$ & \\
    \hline
    \end{tabular}
\end{table}\label{small-primegap-table}

The current best known lower bound on $G(X)$ is
\begin{theorem}[Ford--Green--Konyagin--Maynard--Tao (2017) \cite{ford_long_2017}]
For unbounded $X$, one has
\[
G(X) \gg \frac{\log X \log \log X \log\log\log \log X}{\log \log \log X}
\]
\end{theorem}
\begin{table}[ht]
    \caption{Historical progression of bounds related to \eqref{eqn:large-prime-gaps}.}
    \centering
    \renewcommand{\arraystretch}{2.2}
    \begin{tabular}{|c|c|c|}
    \hline
    Reference & Lower bound on $G(X)$ (for $X$ sufficiently large) \\
    \hline
    Westzynthius (1931) \cite{westzynthius_uber_1931} & $\displaystyle G(X) \gg \log X \frac{\log\log\log X}{\log\log\log\log X}$\\
    \hline
    Erdős (1935) \cite{erdos_difference_1935} & $\displaystyle G(X) \gg \log X \frac{\log\log X}{(\log\log\log X)^2}$ \\
    \hline
    Rankin (1938) \cite{rankin_difference_1938} & $\displaystyle G(X) > (c_0 + o(1))\log X \frac{\log \log X \log \log \log \log X}{(\log \log \log X)^2}$ with $c_0 = 1/3$ \\
    \hline
    Sch\"{o}nhage (1963) \cite{schonhage_bemerkung_1963} & $c_0 = \dfrac{1}{2}e^\gamma$\\
    \hline
    Rankin (1963) \cite{rankin_difference_1963} & $c_0 = e^\gamma$ \\
    \hline
    Maier--Pomerance (1990) \cite{maier_unusually_1990} & $c_0 = 1.31256 e^\gamma$\\
    \hline
    Pintz (1997) \cite{pintz_very_1997} & $c_0 = 2e^\gamma$\\
    \hline
    Ford--Green--Konyagin--Tao (2016) \cite{ford_large_2016}, Maynard (2016) \cite{maynard_large_2016} & $\displaystyle G(X) \gg f(X)\log X \frac{\log \log X \log \log \log \log X}{(\log \log \log X)^2}$ for some $f(X) \to \infty$\\
    \hline
    Ford--Green--Konyagin--Maynard--Tao (2017) \cite{ford_long_2017} & $\displaystyle G(X) \gg \frac{\log X \log \log X \log\log\log \log X}{\log \log \log X}$ \\
    \hline
\end{tabular}
\end{table}\label{small-primegap-table2}
