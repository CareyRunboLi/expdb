\chapter{Applications to the primes}\label{primes-sec}

Recall that $\Lambda$ is the von Mangoldt function, and that the prime number theorem asserts that
$$ \sum_{n \leq x} \Lambda(n) = x + o(x)$$
for unbounded $x$.  If $p_n$ denotes the $n^{\mathrm{th}}$ prime, the prime number theorem is also equivalent to
$$ p_n = (1+o(1)) n \log n$$
for unbounded $n$.

We now consider local versions of the prime number theorem.

\begin{definition}[Prime number theorem in short interval exponents]\label{pnt-ap}
    \begin{itemize}
        \item[(i)] We let $\PNTALL$ denote the least exponent with the following property: if $\eps > 0$ is fixed, and $x$ is unbounded, then
        $$ \sum_{x \leq n < x+y} \Lambda(n) = y + o(y)$$
        whenever $x^{\PNTALL + \eps} \leq y \leq x^{1-\eps}$.
        \item[(ii)] We let $\PNTAA$ denote the least exponent with the following property: if $\eps > 0$ is fixed, and $X$ is unbounded, then we have
        $$ \int_X^{2X} |\sum_{x \leq n < x+y} \Lambda(n)-y|\ dx = o(Xy)$$
        whenever $X^{\PNTAA + \eps} \leq y \leq X^{1-\eps}$.
        \item[(iii)] We let $\GAPMAX$ denote the least exponent such that, if $p_n$ denotes the $n^{\mathrm{th}}$ prime, that
        $$ p_{n+1} - p_n \ll n^{\GAPMAX+o(1)} = p_n^{\GAPMAX+o(1)}$$
        as $n \to \infty$.
        \item[(iv)] We let $\GAPSQUARE$ denote the least exponent such that
        $$ \sum_{p_n \leq x} (p_{n+1}-p_n)^2 \ll x^{\GAPSQUARE+o(1)}$$
        as $x \to \infty$.
    \end{itemize}
\end{definition}

\begin{lemma}[Trivial bounds]\label{pnt-triv}
    We have
    $$ 0 \leq \PNTAA, \GAPMAX \leq \PNTALL \leq 1$$
    and $1 \leq \GAPSQUARE \leq 1+\GAPMAX$.
\end{lemma}

\begin{proof} These are all immediate, after noting from the prime number theorem that $\sum_{p_n \leq x} p_{n+1} - p_n = x^{1+o(1)}$.
\end{proof}

The Cram\'er random model {\bf give cite} predicts

\begin{conjecture}[Prime gap conjecture] $\PNTALL = 0$, and hence (by Lemma \ref{pnt-triv}) $\PNTAA = \GAPMAX=0$ and $\GAPSQUARE=1$.
\end{conjecture}

We note that the results of Maier {\bf give cite} show that there is some deviation from the prime number theorem at very small scales (of order $\log^{O(1)} x$), but this does not directly affect the exponents discussed here due to the epsilons in our definitions.

A basic connection with zero density exponents is

\begin{proposition}[Zero density theorems and prime gaps]\label{prime-gap}  Let
\begin{equation}\label{A-def}
    \|\A\|_\infty \coloneqq \sup_{1/2 \leq \sigma \leq 1} A(\sigma).
\end{equation}
Then
    $$ \PNTALL \leq 1 - \frac{1}{\|\A\|_\infty}$$
    and
    $$ \PNTAA \leq 1 - \frac{2}{\|\A\|_\infty}.$$
\end{proposition}

\begin{proof} See for instance \cite[\S 13.2]{guth-maynard}.
\end{proof}

\begin{corollary}[Ingham-Huxley bound]  We have $\PNTALL \leq \frac{7}{12}$ and $\PNTAA \leq \frac{1}{6}$.
\end{corollary}

\begin{proof}  From Theorem \ref{thm:ingham_zero_density2} and Theorem \ref{huxley-bound} one as $\|\A\|_\infty \leq 12/5$, and the claim now follows from Proposition \ref{prime-gap}.
\end{proof}

\begin{corollary}[Ingham-Guth-Maynard bound]\cite{guth-maynard}  We have $\PNTALL \leq \frac{17}{30}$ and $\PNTAA \leq \frac{2}{15}$.
\end{corollary}

These are currently the best known upper bounds on $\PNTALL$ and $\PNTAA$.

\begin{proof}  From Theorem \ref{thm:ingham_zero_density2} and Theorem \ref{guth-maynard-density} one as $\|\A\|_\infty \leq 30/13$, and the claim now follows from Proposition \ref{prime-gap}.
\end{proof}

\begin{corollary}  The density hypothesis implies that $\PNTALL \leq 1/2$ and $\PNTAA = 0$.
\end{corollary}

The current unconditional best bound on $\GAPMAX$ is

\begin{theorem}[Baker-Harman-Pintz theorem]\cite{baker-harman-pintz} We have $\GAPMAX \leq 0.525$.
\end{theorem}

Historical bounds on $\GAPMAX$ are summarized in the following table:

\begin{table}[ht]
    \caption{Historical upper bounds on $\GAPMAX$.}
    \centering
    \renewcommand{\arraystretch}{1.2}
    \begin{tabular}{|c|c|}
    \hline
    Reference & Upper bound \\
    \hline
    Hoheisel (1930) \cite{hoheisel_1930} & $1 - \frac{1}{33000}$ \\
    Heilbronn (1933) \cite{heilbronn_1933} & $1 - \frac{1}{250}$ \\
    Ingham (1937) \cite{ingham_difference_1937} & $\frac{5}{8}$ \\
    Montgomery (1969) \cite{montgomery_1969} & $\frac{3}{5}$ \\
    Huxley (1972) \cite{Huxley} & $\frac{7}{12}$ \\
    Iwaniec--Jutila (1979)\cite{iwaniec-jutila} & $\frac{13}{23}$ \\
    Heath-Brown--Iwaniec (1979) \cite{heathbrown_iwaniec_1979} & $\frac{11}{20}$ \\
    Pintz (1981) \cite{pintz_1981} & $\frac{17}{31}$ \\
    Iwaniec--Pintz (1984) \cite{iwaniec-pintz} & $\frac{23}{42}$\\
    Baker--Harman--Pintz (2001) \cite{baker-harman-pintz} & $\frac{21}{40}$ \\
    \hline
    \end{tabular}
    \end{table}\label{gapmax-table}


The following general bound on $\GAPSQUARE$ is known:

\begin{proposition}
    We have
    $$ \GAPSQUARE \leq \max( 2-\frac{2}{\|\A\|_\infty}, \sup_{1/2 \leq \sigma \leq 1} \max(\alpha(\sigma), \beta(\sigma)))$$
    where
    $$ \alpha(\sigma) \coloneqq 4\sigma-2 + 2 \frac{A^*(\sigma)(1-\sigma)-1}{A^*(\sigma)-A(\sigma)}$$
    and
    $$ \beta(\sigma) \coloneqq 4\sigma-2 + \frac{A^*(\sigma)(1-\sigma)-1}{A(\sigma)}.$$
\end{proposition}

\begin{proof} See \cite[Lemma 2]{heath_brown_consecutive_II}. We remark that this lemma allows $\sigma$ to range over $0 \leq \sigma \leq 1$ rather than $1/2 \leq \sigma \leq 1$, but it is easy to see that the contributions of the $0 \leq \sigma < 1/2$ cases are dominated by the $\sigma=1/2$ case.
\end{proof}

This proposition recovers several known bounds (both conditional and unconditional) on $\GAPSQUARE$:

\begin{corollary}\
    \begin{itemize}
    \item[(i)] Assuming the Riemann hypothesis, $\GAPSQUARE = 1$.  {\bf Selberg}
    \item[(ii)] Unconditionally, one has $\GAPSQUARE \leq 4/3$.  \cite{heath_brown_consecutive_I}
    \item[(iii)] Assuming the Lindelof hypothesis, $\GAPSQUARE \leq 7/6$. \cite{heath_brown_consecutive_II}
    \item[(iv)] Unconditionally, $\GAPSQUARE \leq 1 + 2 \times \frac{173}{1067}$. \cite[\S 7]{heath_brown_consecutive_II}
    \item[(v)] Unconditionally, $\GAPSQUARE \leq 23/18$. \cite[Theorem 12.14]{ivic}.
\end{itemize}
\end{corollary}

\begin{proof} {\bf give details}
\end{proof}
