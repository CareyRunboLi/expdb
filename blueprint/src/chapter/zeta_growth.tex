\chapter{Growth exponents for the Riemann zeta function}

\begin{definition}[Growth rate of zeta]\label{zeta-grow-def}  For any fixed $\sigma \in \R$, let $\mu(\sigma)$ denote the least possible (fixed) exponent for which one has the bound
    $$ |\zeta(\sigma+it)| \ll |t|^{\mu(\sigma)+o(1)}$$
    for all unbounded $t$.
\end{definition}

One can check that for each $\sigma$, the set of possible candidates for $\mu(\sigma)$ is closed (by underspill), non-empty, and bounded from below, so that $\mu(\sigma)$ is well-defined as a real number.  An equivalent definition without asymptotic notation, is that $\mu(\sigma)$ is the least real number such that for every $\eps>0$ there exists $C>0$ such that
$$ |\zeta(\sigma+it)| \ll C |t|^{\mu(\sigma)+\eps}$$
for all $t$ with $|t| \geq C$; equivalently, one has
$$ \mu(\sigma) = \limsup_{|t| \to \infty} \frac{\log |\zeta(\sigma+it)|}{\log|t|}.$$

\python{bound_mu}
\code{Bound_mu}

\begin{lemma}[Trivial bound]\label{zeta-grow-triv}\uses{zeta-grow-def}
    One has $\mu(\sigma)=0$ for all $\sigma \geq 1$.
\end{lemma}

\python{bound_mu}
\code{apply_trivial_mu_bound()}

\begin{proof}  Immediate from the absolute convergence of the Dirichlet series for both $\zeta(s)$ and $1/\zeta(s)$; see e.g., \cite[Theorem 1.9]{ivic}.
\end{proof}

\begin{lemma}[Convexity]\label{zeta-convex}\uses{zeta-grow-def}
$\mu$ is convex.
\end{lemma}

\python{bound_mu}
\code{bound_mu_convexity()}

\begin{proof} Immediate from the Phragm\'en--Lindel\"of principle; see e.g., \cite[\S A.8]{ivic}.
\end{proof}

\begin{lemma}[Functional equation]\label{zeta-functional}\uses{zeta-grow-def}
    One has $\mu(1-\sigma) = \mu(\sigma) + \sigma - 1/2$ for all $0 \leq \sigma \leq 1/2$.
\end{lemma}

\python{bound_mu}
\code{apply_functional_equation()}

\begin{proof}  Immediate from the functional equation for $\zeta$ and asymptotics of the Gamma function; see e.g., \cite[(1.23), (1.25)]{ivic}.
\end{proof}

\begin{lemma}[Left of critical strip]\label{zeta-left}\uses{zeta-grow-def}
    One has $\mu(\sigma)=1/2-\sigma$ for $\sigma \leq 0$.
\end{lemma}

\python{bound_mu}
\code{apply_trivial_mu_bound()}

\begin{proof}\uses{zeta-grow-triv, zeta-functional}  Immediate from Lemmas \ref{zeta-grow-triv}, \ref{zeta-functional}.
\end{proof}

\begin{lemma}[Convexity bounds]\label{zeta-convexity}\uses{zeta-grow-def}  One has $\max(0, 1/2-\sigma) \leq \mu(\sigma) \leq (1-\sigma)/2$ for $0 \leq \sigma \leq 1$.
\end{lemma}

\python{bound_mu}
\code{apply_trivial_mu_bound()}

\begin{proof}\uses{zeta-grow-triv, zeta-convex, zeta-left}  Immediate from Lemma \ref{zeta-grow-triv}, Lemma \ref{zeta-left}, and Lemma \ref{zeta-convexity}.
\end{proof}

\section{Connection with exponent pairs and dual exponent pairs}

\begin{lemma}[Connection with dual exponent pairs]\label{mu-beta}\label{zeta-grow-def, beta-def}  For any $1/2 \leq \sigma \leq 1$, one has
    $$ \mu(\sigma) \leq \sup_{0 \leq \alpha \leq 1/2} \beta(\alpha) - \alpha \sigma.$$
\end{lemma}

\begin{proof}\uses{auto}  Let $t$ be unbounded.  From the Riemann--Siegel formula (see \cite[Theorem 4.1]{ivic}) one has
    $$ \zeta(\sigma+it) \ll \left|\sum_{n \leq \sqrt{t/2\pi}} \frac{1}{n^{\sigma+it}}\right| + |t|^{1/2-\sigma} \left|\sum_{n \leq \sqrt{t/2\pi}} \frac{1}{n^{1-\sigma-it}}\right| + O(1).$$
From dyadic decomposition and Definition \ref{beta-def} (and Lemma \ref{auto}) one has for any fixed $\eps>0$ that
$$\sum_{t^\eps \leq n \leq \sqrt{t/2\pi}} \frac{1}{n^{\sigma+it}} \ll
|t|^{\sup_{\eps \leq \alpha \leq 1/2} \beta(\alpha) - \alpha \sigma + o(1)},$$
while from the triangle inequality one has the crude bound
$$\sum_{n < t^\eps} \frac{1}{n^{\sigma+it}} \ll |t|^\eps.$$
Combining the bounds and using underspill, we conclude that
$$\sum_{n \leq \sqrt{t/2\pi}} \frac{1}{n^{\sigma+it}} \ll
|t|^{\sup_{0 \leq \alpha \leq 1/2} \beta(\alpha) - \alpha \sigma + o(1)}.$$
A similar argument gives
$$\sum_{n \leq \sqrt{t/2\pi}} \frac{1}{n^{1-\sigma-it}} \ll
|t|^{\sup_{0 \leq \alpha \leq 1/2} \beta(\alpha) - \alpha (1-\sigma) + o(1)}$$
Since $\sigma \geq 1/2$ and $\alpha \leq 1/2$, one has $(1/2-\sigma) - \alpha(1-\sigma) \leq -\alpha \sigma$, and hence
$$ \zeta(\sigma+it) \ll
|t|^{\sup_{0 \leq \alpha \leq 1/2} \beta(\alpha) - \alpha \sigma + o(1)}$$
giving the claim.
\end{proof}

We remark that this inequality is morally an equality (indeed, it would be if one would restrict the model phases in Definition \ref{beta-def} to purely the logarithmic phase $u \mapsto \log u$).

The following form of Lemma \ref{mu-beta} is convenient for applications:

\begin{corollary}[Exponent pairs and $\mu$]\label{exp-pair-mu}\uses{exp-pair-def, zeta-grow-def} If $(k,\ell)$ is an exponent pair, then
$$ \mu(\ell-k) \leq k.$$
\end{corollary}

\python{bound_mu}
\code{obtain_mu_bound_from_exponent_pair()}

\begin{proof}\uses{mu-beta, beta-duality}  Immediate from Lemma \ref{mu-beta} and Lemma \ref{beta-duality}.  See also \cite[(7.57)]{ivic}.
\end{proof}

\begin{conjecture}[Lindelof hypothesis]\label{LH}\uses{zeta-grow-def} One has $\mu(1/2)=0$.
\end{conjecture}

\python{bound_mu}
\code{bound_mu_Lindelof()}

\begin{proposition}[Conjectured value of $\mu$]\label{mu-conj}\uses{zeta-grow-def, LH}  We have the lower bound
\begin{equation}\label{muh}
    \mu(\sigma) \geq \max\left(0, \frac{1}{2}-\sigma\right)
\end{equation}
for all $\sigma \in \R$, and equality holds everywhere in \eqref{muh} if and only if the Lindel\"of hypothesis holds.
\end{proposition}

We remark that this proposition explains why there are no further lower bounds on $\mu$ in the literature beyond \eqref{muh}; all the remaining known results revolve around upper bounds.

\begin{proof}\uses{zeta-grow-triv, zeta-left, zeta-convexity}  Clearly equality in \eqref{muh} implies the Lindel\"of hypothesis, while from the trivial bounds in Propositions \ref{zeta-grow-triv}, \ref{zeta-left} and convexity (Lemma \ref{zeta-convexity}) one we see that the Lindel\"of hypothesis implies the upper bound
$$     \mu(\sigma) \leq \max\left(0, \frac{1}{2}-\sigma\right)$$
for all $\sigma$.  So it suffices to establish the lower bound unconditionally.  By the functional equation (Proposition \ref{zeta-functional}) it suffices to do this for $\sigma \geq 1/2$; in fact by convexity it suffices to establish the claim when $1/2 < \sigma < 1$.  In this regime, the $L^2$ mean value theorem (see \cite[Theorem 1.11]{ivic}) gives
$$ \int_0^T |\zeta(\sigma+it)|^2\ dt \asymp T$$
for large $T$, giving the claim.
\end{proof}

\section{Known bounds on \texorpdfstring{$\mu$}{mu}}

\begin{table}[ht]
\caption{Historical bounds on $\mu(\sigma)$ for $1/2 \le \sigma \le 1$, and the exponent pair generating them (if applicable). \textbf{Longer term goal: supplement as many of these citations as possible with derivations from other exponents and relations in the database}}
\centering
\renewcommand{\arraystretch}{1.2}
\begin{tabular}{|c|c|c|}
\hline
Reference & Results & Exponent pair \\
\hline
Hardy--Littlewood (1923) \cite{hardy_littlewood_1923} & $\mu(1/2) \le 1/6$ & (1/6, 2/3)\\
\hline
Walfisz (1924) \cite{walfisz_1924} & $\mu(1/2) \le 193/988$ & \\
\hline
Titchmarsh (1932) \cite{titchmarsh_van_1931} & $\mu(1/2) \leq 27/164$ & \\
\hline
Phillips (1933) \cite{phillips_zeta_1933} & $\mu(1/2) \leq 229/1392$ & \\
\hline
Titchmarsh (1942) \cite{titchmarsh_order_1942}  & $\mu(1/2) \leq 19/116$ & \\
\hline
Min (1949) \cite{min_on_1949} & $\mu(1/2) \leq 15/92$ & \\
\hline
Haneke (1962) \cite{haneke_verscharfung_1963} & $\mu(1/2) \leq 6/37$&  \\
\hline
Kolesnik (1973) \cite{kolesnik_1973} & $\mu(1/2) \leq 173/1067$ & \\
\hline
Kolesnik (1982) \cite{kolesnik_order_1982} & $\mu(1/2) \leq 35/216$ & \\
\hline
Kolesnik (1985) \cite{kolesnik_1985} & $\mu(1/2) \leq 139/858$ & \\
\hline
Bombieri--Iwaniec (1985) \cite{bombieri_order_1986} & $\mu(1/2) \leq 9/56$ & $(9/56, 1/2+9/56)$\\
\hline
Watt (1989) \cite{watt_exponential_1989} & $\mu(1/2) \leq 89/560$ & $(89/560, 1/2+89/560)$\\
\hline
Huxley--Kolesnik (1991) \cite{huxley_exponential_1991} & $\mu(1/2) \leq 17/108$ & $(17/108, 1/2+17/108)$\\
\hline
Huxley (1993) \cite{huxley_exponential_1993} & $\mu(1/2) \leq 89/570$ & $(89/570, 1/2+89/570)$\\
\hline
Huxley (1996) \cite{huxley_area_1996} & $\mu(1934/3655) \leq 6299/43860$ & \\
\hline
Sargos (2003) \cite{sargos_analog_2003} & $\mu(49/51) \leq 1/204$, $\mu(361/370) \leq 1/370$&  \\
\hline
Huxley (2005) \cite{huxley_exponential_2005} & $\mu(1/2) \leq 32/205$ & $(32/205, 1/2+32/205)$ \\
\hline
Lelechenko (2014) \cite{Lelechenko_linear_2014} & $\mu(3/5) \leq 1409/12170$, $\mu(4/5) \leq 3/71$& \\
\hline
Bourgain (2017) \cite{bourgain_decoupling_2017} & $\mu(1/2) \leq 13/84$ & $(13/84, 1/2+13/84)$ \\
\hline
Heath-Brown (2017) \cite{heathbrown_new_2017} & $\mu(\sigma) \le \frac{8}{63}\sqrt{15}(1 - \sigma)^{3/2}$ for $1/2 \le \sigma \le 1$& \\
\hline
Heath-Brown (2020) \cite{demeter_small_2020} & $\mu(11/15) \leq 1/15$& \\
\hline
\end{tabular}
\end{table}\label{mu-table}

\literature
\code{add_literature_bounds_mu()}
