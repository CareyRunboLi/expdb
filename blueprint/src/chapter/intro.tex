\chapter{Introduction}

This is the LaTeX ``Blueprint'' form of the \emph{analytic number theory exponent database (ANTEDB)}, which is an ongoing project to record (both in a human-readable and computer-executable formats) the latest known bounds, conjectures, and other relationships concerning several exponents of interest in analytic number theory.  It can be viewed as an expansion of the paper \cite{trudgian-yang}. Currently, the database is recording information on the following exponents:

\begin{itemize}
\item The growth exponent $\mu(\sigma)$ of the zeta function $\zeta(\sigma+it)$.
\item Exponent pairs $(k,\ell)$.
\item The exponential sum function $\beta(\alpha)$ dual to exponent pairs.
\item The moment exponents $M(\sigma,A)$ of the zeta function.
\item Large value exponents $\LV_\zeta(\sigma, \tau)$ of the exponential sums $\sum_{n \in I} n^{it}$.
\item Large value exponents $\LV(\sigma, \tau)$.
\item Large value additive energy exponents $\LV^*(\sigma, \tau)$.
\item Zero density exponents $\A(\sigma)$ for the zeta function.
\item Zero density additive energy exponents $A^*(\sigma)$ for the zeta function.
\item Exponents $\alpha_k$ for the Dirichlet divisor problem.
\item The primitive Pythagorean triple exponent $\Pythag$.
\item Exponents $\PNTALL$, $\PNTAA$ for the prime number theorem in all or almost all short intervals.
\item The maximal prime gap exponent $\GAPMAX$.
\item The prime gap second moment exponent $\GAPSQUARE$.
\end{itemize}

Possible future directions for expansion include
\begin{itemize}
    \item Exponents for double zeta sums
    \item Exponents for $L$-functions (in both $q$ and $T$ aspects)
    \item More exponents relating to prime gaps
\end{itemize}

The database aims to enumerate, as comprehensively as possible, all the various known or conjectured facts about these exponents, including ``trivial'' or ``obsolete'' such facts.  Of particular interest are implications that allow new bounds on exponents to be established from existing bounds on other exponents.

Each of the facts in the database can be supported with a reference, or one or more proofs, or with executable code in Python; ideally one should have all three (and with a preference for proofs that rely as much as possible on other facts in the database).  In the future we could also expand this database to support each fact with formal proofs in proof assistant languages such as Lean.

In order to facilitate the dependency tree of the python code, as well as to assist readers who wish to derive the facts in this database from first principles, the blueprint is arranged in linear order.  Thus, the statement and proof of a proposition in the blueprint is only permitted to use propositions and definitions that are located earlier in the blueprint, although we do allow forward-referencing references in the remarks.  As a consequence, the material relating to a single topic will not necessarily be located in a single chapter, but could be spread out over multiple chapters, depending on how much advanced material is needed to state or prove the required results.  Additionally, a single proposition may occur multiple times in the blueprint, if it has multiple proofs with varying prerequisites.  In the future, we plan to implement a search feature that will allow the reader to locate all propositions of relevance to a given topic (e.g., all propositions whose statement involves the concept of an exponent pair).

