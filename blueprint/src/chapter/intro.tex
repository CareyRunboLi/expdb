\chapter{Introduction}\label{intro-chapter}

This is the LaTeX ``Blueprint'' form of the \emph{analytic number theory exponent database (ANTEDB)}, which is an ongoing project to record (both in a human-readable and computer-executable formats) the latest known bounds, conjectures, and other relationships concerning several exponents of interest in analytic number theory.  It can be viewed as an expansion of the paper \cite{trudgian-yang}. Currently, the database is recording information on the following exponents:

\begin{itemize}
\item Exponent pairs $(k,\ell)$.
\item The exponential sum function $\beta(\alpha)$ dual to exponent pairs.
\item The growth exponent $\mu(\sigma)$ of the zeta function $\zeta(\sigma+it)$.
\item The moment exponents $M(\sigma,A)$ of the zeta function.
\item Large value exponents $\LV(\sigma, \tau)$ for Dirichlet polynomials $\sum_{n \in [N,2N]} a_n n^{-it}$.
\item Large value exponents $\LV_\zeta(\sigma, \tau)$ of the zeta polynomials $\sum_{n \in I} n^{-it}$.
\item Large value additive energy exponents $\LV^*(\sigma, \tau)$, $\LV^*_\zeta(\sigma, \tau)$ for Dirichlet and zeta polynomials.
\item Zero density exponents $\A(\sigma)$ for the zeta function.
\item Zero density additive energy exponents $\A^*(\sigma)$ for the zeta function.
\item The regions $\Energy$, $\Energy_\zeta$ of exponent tuples $(\sigma,\tau,\rho,\rho^*,s)$ recording possible large values, large value additive energy, and double zeta sums for Dirichlet and zeta polynomials.
\item Exponents $\alpha_k$ for the Dirichlet divisor problem.
\item The primitive Pythagorean triple exponent $\Pythag$.
\item Exponents $\PNTALL$, $\PNTAA$ for the prime number theorem in all or almost all short intervals.
\item The maximal prime gap exponent $\GAPMAX$.
\item The prime gap second moment exponent $\GAPSQUARE$.
\item Results on the de Bruijn-Newman constant $\Lambda$.
\end{itemize}

Possible future directions for expansion include
\begin{itemize}
    \item Exponents for $L$-functions (in both $q$ and $T$ aspects).
    \item More exponents relating to prime gaps.
    \item Exponents relating to sieve theory.
    \item Integration with the \href{https://tmeemt.github.io/Chest/}{TME-EMT project}.
    \item Log-free estimates, or estimates with explicit constants.
    \item Lean certification of some of the calculations in the database.
    \item Zero-free regions for the Riemann zeta function and L-functions.
    \item Upper and lower bounds on gaps between zeroes of zeta or L-functions (assuming RH if necessary), for instance incorporating results obtained via the Montgomery-Odlyzko method.
    \item Brun-Titchmarsh type theorems.
    \item Character sum bounds (such as the Polya-Vinogradov and Burgess bounds) and the least quadratic residue problem.
    \item Small gaps (or narrow k-tuples) between primes, including the work of Zhang, Maynard, and Polymath.
    \item Error terms in the prime number theorem, and in the prime number theorem in arithmetic progressions.  (This topic is currently claimed.)
    \item Level of distribution of the primes and other multiplicative functions, possibly with restrictions on the moduli.
    \item Error terms in the Titchmarsh divisor problem.
    \item The Gauss circle problem and its generalizations.
    \item The proportion of zeroes on the critical line, and estimates on mollifiers for the zeta function.
    \item Goldbach and Waring type problems. (This topic is currently claimed.)
    \item Vinogradov mean value type theorems.
\end{itemize}

The database aims to enumerate, as comprehensively as possible, all the various known or conjectured facts about these exponents, including ``trivial'' or ``obsolete'' such facts.  Of particular interest are implications that allow new bounds on exponents to be established from existing bounds on other exponents.

Each of the facts in the database can be supported with a reference, or one or more proofs, or with executable code in Python; ideally one should have all three (and with a preference for proofs that rely as much as possible on other facts in the database).  In the future we could also expand this database to support as many of these facts as possible with formal derivations in proof assistant languages such as Lean.

In order to facilitate the dependency tree of the python code, as well as to assist readers who wish to derive the facts in this database from first principles, the blueprint is arranged in linear order.  Thus, the statement and proof of a proposition in the blueprint is only permitted to use propositions and definitions that are located earlier in the blueprint, although we do allow forward-referencing references in the remarks.  As a consequence, the material relating to a single topic will not necessarily be located in a single chapter, but could be spread out over multiple chapters, depending on how much advanced material is needed to state or prove the required results.  Additionally, a single proposition may occur multiple times in the blueprint, if it has multiple proofs with varying prerequisites.  In the future, one could hope to implement a search feature that will allow the reader to locate all propositions of relevance to a given topic (e.g., all propositions whose statement involves the concept of an exponent pair).

This is intended to be a living database, and we hope to gain community support for updating it.  As such, corrections, suggestions, and new contributions are very welcome, either via email to one of us (\href{mailto:tao@math.ucla.edu}{Terence Tao}, \href{mailto:timothy.trudgian@unsw.edu.au}{Timothy Trudgian}, or \href{mailto:andrew.yang1@unsw.edu.au}{Andrew Yang}), or by a direct pull request to the \href{https://github.com/teorth/expdb}{Github repository}.  Instructions for contributing can be found \href{https://github.com/teorth/expdb/blob/main/CONTRIBUTING.md}{here}.

A paper describing the ANTEDB, and the new bounds that were already obtained as a result of compiling the database, can be found at \cite{tao-trudgian-yang}.
