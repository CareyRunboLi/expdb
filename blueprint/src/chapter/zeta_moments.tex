\chapter{Moment growth for the zeta function}

\begin{definition}[Zeta moment exponents]\label{zeta-moment-def}  For fixed $\sigma \in \R$ and $A \geq 0$, we define $M(\sigma,A)$ to be the least (fixed) exponent for which the bound
$$ \int_T^{2T} |\zeta(\sigma+it)|^A\ dt \ll T^{M(\sigma,A)+o(1)}$$
holds for all unbounded $T > 1$.
\end{definition}

Again, it is not difficult to show that $M(\sigma,A)$ is a well-defined (fixed) real number.  A non-asymptotic definition is that it is the least constant such that for every $\eps>0$ there exists $C>0$ such that
$$ \int_T^{2T} |\zeta(\sigma+it)|^A\ dt \leq C T^{M(\sigma,A)+\eps}$$
holds for all $T \geq C$.

\begin{lemma}[Basic properties of $M(\sigma,A)$]\label{zeta-moment-basic}\uses{zeta-moment-def}\
\begin{itemize}
\item[(i)] $M(\sigma,A)$ is convex in $\sigma$.
\item[(ii)] For any $\sigma$, $a (M(\sigma,1/a)-1)$ is convex non-increasing in $a$.
\item[(iii)] $M(\sigma,A)=0$ for all $A \geq 0$ and $\sigma \geq 1$.
\item[(iv)] $M(1/2,A)=0$ for all $0 \leq A \leq 4$.
\item[(v)] $M(\sigma,A) \geq 0$ for all $1/2 \leq \sigma \leq 1$ and $A \geq 0$.
\item[(vi)] $M(\sigma,0) = 1$ for all $\sigma$.
\item[(vii)] $M(1-\sigma,A) = M(1-\sigma,A) + (1/2-\sigma) A$ for all $\sigma \in \R$ and $A \geq 0$.
\item[(viii)] For any $\sigma$, $a(M(\sigma,1/a)-1)$ converges to $\mu(\sigma)$ as $a \to 0$.  In particular, $M(\sigma,A) \leq A \mu(\sigma) + 1$ for all $\sigma \geq 0$ and $A \geq 0$.
\end{itemize}
\end{lemma}

\begin{proof} The claim (i) follows from the Phragmen-Lindelof principle.  The claim (ii) follows from H\"older.  The claim (iii) follows from standard upper and lower bounds on $\zeta(\sigma+it)$ for $\sigma \geq 1$.  For the claim (iv), we have standard moment estimates
$$ \int_T^{2T} |\zeta(\frac{1}{2}+it)|^2\ dt = T^{1+o(1)}$$
and
$$ \int_T^{2T} |\zeta(\frac{1}{2}+it)|^4\ dt = T^{1+o(1)}$$
for any unbounded $T>1$, and the claim follows from H\"older's inequality.  The claim (v) follows from (i)-(iv), and (vi) is trivial. The claim (vii) follows easily from the functional equation.

For (viii), the bound $M(\sigma,A) \leq A \mu(\sigma) + 1$ is trivial, which implies that
$$ \lim_{a \to 0} a(M(\sigma,1/a)-1) \leq \mu(\sigma).$$
Suppose for contradiction that
$$ \lim_{a \to 0} a(M(\sigma,1/a)-1) < \mu(\sigma),$$
thus there is $\delta>0$ such that
$$M(\sigma,A) \leq A (\mu(\sigma)-\delta) + 1$$
for all $A\geq 0$.  By convexity, this gives
$$M(\sigma+\eps,A) \leq A (\mu(\sigma)-\delta/2) + 1$$
for all sufficiently small $\eps$, and then by the Cauchy integral formula and H\"older's inequality we can conclude that
$$ |\zeta(\sigma+\eps/2 +it)| \ll |t|^{\mu(\sigma)-\delta/2 + O(1/A) + o(1)} $$
for unbounded $|t|$, leading to
$$ \mu(\sigma+\eps/2) \leq \mu(\sigma)-\delta/2 + O(1/A).$$
Sending $A$ to infinity and $\eps$ to zero, we obtain a contradiction.
\end{proof}

\begin{corollary}\label{moment_from_lindelof}\uses{zeta-moment-def, zeta-moment-basic} If the Lindelof hypothesis holds, then $M(\sigma,A) = \max(1/2-\sigma,0) A$ for all $\sigma \in \R$ and $A \geq 0$.
\end{corollary}

Note from Lemma \ref{zeta-moment-basic} that we always have the lower bound $M(\sigma,A) \geq \max(1/2-\sigma,0) A$.  Thus there are not expected to be any further lower bound results for $M(\sigma,A)$, and we focus now on upper bounds.  From Lemma \ref{zeta-moment-basic} we may restrict attention to the region $1/2 \leq \sigma \leq 1$ and $A \geq 4$.

We can relate $M(\sigma,A)$ to $\LV_\zeta(\sigma,\tau)$:

\begin{lemma}\label{mad}\uses{zeta-moment-def, lvz-def}  If $1/2 \leq \sigma_0 \leq 1$ and $A \geq 1$, then
\begin{equation}\label{M-form}
 M(\sigma_0,A) = \sup_{\tau \geq 2; \sigma \geq 1/2} (A(\sigma-\sigma_0) + \LV_\zeta(\sigma,\tau))/\tau.
\end{equation}
In particular, one has
$$ \LV_\zeta(\sigma,\tau) \leq \tau M(\sigma_0,A) - A(\sigma-\sigma_0)$$
whenever $\sigma \geq 1/2$ and $\tau \geq 2$.
\end{lemma}

\begin{proof}  We first show the lower bound, or equivalently that
$$ A(\sigma-\sigma_0) + \LV_\zeta(\sigma,\tau) \leq \tau M(\sigma_0,A) - A(\sigma-\sigma_0)$$
whenever $\tau \geq 2$ and $\sigma \geq 1/2$.  Accordingly, let $N$ be unbounded, $T = N^{\tau+o(1)}$, $I \subset [N,2N]$, and $t_1,\dots,t_R$ be a $1$-separated subset of $[T,2T]$ such that
$$ |\sum_{n \in I} n^{-it_r}| \gg N^{\sigma+o(1)}.$$
By standard Fourier analysis, this gives
$$ \int_{T/2}^{3T} |\zeta(\sigma_0+it)|\ \frac{dt}{1+|t-t_r|} \gg N^{\sigma - \sigma_0 + o(1)}$$
and hence by H\"older
$$ \int_{T/2}^{3T} |\zeta(\sigma_0+it)|^A\ \frac{dt}{1+|t-t_r|} \gg N^{A(\sigma - \sigma_0) + o(1)}$$
so on summing in $r$
$$ \int_{T/2}^{3T} |\zeta(\sigma_0+it)|^A\ dt \gg R N^{A(\sigma - \sigma_0) + o(1)}.$$
By Definition \ref{zeta-moment-def}, the left-hand side is $\ll T^{M(\sigma_0,A)+o(1)}$.  Since $T = N^{\alpha+o(1)}$, we obtain
$$ R \ll N^{\tau M(\sigma_0,A) - A(\sigma-\sigma_0)},$$
giving the claim.

For the converse bound, let $M$ be the right-hand side of \eqref{M-form}.  From Lemma \ref{lvz-2} we have $M \geq 1$. By \cite[\S 8.1]{ivic} it will suffice to show that for any $V>0$ and any $1$-separated $W \subset [T,2T]$ with
$$ |\zeta(\sigma_0+it)| \geq V$$
for all $t \in W$, one has
$$ |W| \ll T^{M+o(1)} V^{-A}.$$
The claim is clear if $V \geq T^C$ or $V \leq T^C$ for some sufficiently large $C$, so we may assume that $V = T^{O(1)}$.  We also clearly can assume $|W|\geq 1$.  Using the Riemann--Siegel formula \cite[Theorem 4.1]{ivic} and dyadic decomposition, we have either
$$ |\sum_{n \in I} \frac{1}{n^{\sigma_0+it}} \gg T^{-o(1)} V$$
or
$$ T^{1/2-\sigma_0} |\sum_{n \in I} \frac{1}{n^{1-\sigma_0-it}}| \gg T^{-o(1)} V$$
for some $I \subset [N,2N]$ and $1 \leq N \ll T^{1/2}$, and all $t \in W$.  In either case, we can perform summation by parts and conclude that
$$ |\sum_{n \in I'} n^{-it}|\gg T^{-o(1)} V N^{\sigma_0}$$
or
$$ |\sum_{n \in I'} n^{-it}|\gg T^{-o(1)} V N^{1-\sigma_0} T^{\sigma_0-1/2}$$
for some $I'$ in $[N,2N]$ and all $t \in W$.  As $\sigma_0 \geq 1/2$, the letter hypothesis is stronger than the former, so we may assume the former.  If $N = T^{o(1)}$ then this would imply that $V \ll T^{o(1)}$, and we would be done from the trivial bound $R \ll T$ since $M \geq 1$.
Hence, after passing to a subsequence, we can assume that $N = T^{1/\tau+o(1)}$ for some $2 < \tau < \infty$.  We can also assume that $V = N^{\sigma-\sigma_0+o(1)}$ for some $\sigma \in \R$. If $\sigma \leq \sigma_0$ then $V \ll T^{o(1)}$ and we are done as before, so we may assume $\sigma > \sigma_0$; in particular, $\sigma \geq 1/2$.  From Lemma \ref{lvz-asymp} we have
$$ |W| \ll N^{\LV_\zeta(\sigma,\tau)+o(1)}$$
and hence by definition of $M$
$$ |W| \ll N^{M \tau - A (\sigma-\sigma_0)+o(1)} = T^{M+o(1)} V^{-A}$$
as required.
\end{proof}

\begin{corollary}[Fourth moment bound]\label{lvz-4}\uses{lvz-def} One has $\LV_\zeta(\sigma,\tau) \leq \tau - 4 (\sigma-1/2)$ for all $1/2 \leq \sigma \leq 1$ and $\tau \geq 2$.
\end{corollary}

\begin{proof}\uses{mad, zeta-moment-basic} Apply Lemma \ref{mad} with $\sigma_0 = 1/2$ and $A=4$, using Lemma \ref{zeta-moment-basic}(iv).
\end{proof}

We have an important twelfth moment estimate of Heath-Brown:

\begin{theorem}[Heath-Brown twelfth moment estimate]\label{hb-12}\cite{heathbrown_twelfth_1978}\uses{zeta-moment-def, lvz-def, mad} $M(1/2,12) \leq 2$.  Equivalently (by Lemma \ref{mad}), one has $\LV_\zeta(\sigma,\tau) \leq 2\tau - 12 (\sigma-1/2)$ for all $\tau \geq 2$ and $1/2 \leq \sigma \leq 1$.
\end{theorem}

\begin{proof}\uses{zeta-from-exp, vdc-class, lvz-infty}  From Lemma \ref{zeta-from-exp} with the exponent pair $(1/2,1/2)$ from Lemma \ref{vdc-class} we have
$$ \LV_\zeta(\sigma,\tau) \leq \min( \tau - 6(\sigma-1/2), 2\tau - 12 (\sigma-1/2) ).$$
If $2\tau - 12 (\sigma-1/2)  \geq 0$, the claim is immediate; if instead $2\tau - 12 (\sigma-1/2) < 0$, use Lemma \ref{lvz-infty}.
\end{proof}

We also have a variant bound, which is slightly better when $\tau$ is close to $6(\sigma-1/2)$:

\begin{theorem}[Auxiliary Heath-Brown estimate]\label{hb-12-aux}\uses{lvz-def}  For $\tau \geq 2$ and $1/2 \leq \sigma \leq 1$, one has
$$ \LV_\zeta(\sigma,\tau) \leq \min( \tau-6(\sigma-1/2), 5\tau-32(\sigma-1/2)).$$
\end{theorem}

\begin{proof} Let $(N,T,V,(a_n)_{n \in [N,2N]},J,W)$ be a zeta large value pattern with $N$, $V = N^{\sigma+o(1)}$, $T = N^{\tau+o(1)}$ and $W = N^{\LV_\zeta(\sigma,\tau)+o(1)}$.
Our task is to show that
$$ |W| \ll T^{o(1)} ( T (N^{-1/2} V)^{-6} + T^5 (N^{-1/2} V)^{-32}).$$
Write $a(n) = 1_I(n)$. By a Fourier analytic expansion we can bound
$$ N^{-1/2} |\sum_{n \in I} n^{-it}| \ll T^{o(1)} \int_{T/4}^{3T} |\zeta(1/2+it_1)| \frac{dt_1}{1+|t_1-t|} + N^{-\eps}$$
for some fixed $\eps>0$ and all $t \in W$, hence
$$ \int_{T/4}^{3T} |\zeta(1/2+it_1)| \frac{dt}{1+|t_1-t|} \gg T^{-o(1)} N^{-1/2} V.$$
In particular, we can truncate to large values of $\zeta(1/2+it_1)$, in the sense that
$$ \int_{T/4}^{3T} |\zeta(1/2+it_1)| 1_{|\zeta(1/2+it_1)| \geq T^{-o(1)} N^{-1/2} V} \frac{dt_1}{1+|t_1-t|} \gg T^{-o(1)} N^{-1/2} V.$$
Summing in $t$ and using the $1$-separation to bound the sum of $1/(1+|t_1-t|)$ by $T^{o(1)}$, we conclude that
$$ \int_{T/4}^{3T} |\zeta(1/2+it_1)| 1_{|\zeta(1/2+it_1)| \geq T^{-o(1)} N^{-1/2} V} \ dt_1 \gg T^{-o(1)} R N^{-1/2} V.$$
Hence by dyadic pigeonholing we have
$$ V' \int_{T/4}^{3T} |\zeta(1/2+it_1)| 1_{|\zeta(1/2+it_1)| \asymp V'} \ dt \gg T^{-o(1)} R N^{-1/2} V$$
for some $V' \geq T^{-o(1)} N^{-1/2} V$, and thus
$$ |\zeta(1/2+it')| \asymp V'$$
for all $t'$ in some $1$-separated subset $W'$ of $[T/4, 3T]$ with
$$ |W'| \gg T^{-o(1)} |W| N^{-1/2} V / V'.$$
Applying \cite[Theorem 2]{heathbrown_twelfth_1978} (treating different cases using the bounds \cite[(7), (8), (9)]{heathbrown_twelfth_1978}), we have the bound
$$ |W'| \ll T^{o(1)} ( T (V')^{-6} + T^5 (V')^{-32})$$
and thus
$$ |W| \ll T^{o(1)} ( T (N^{-1/2} V)^{-1} (V')^{-5} + T^5 (N^{-1/2} V)^{-1} (V')^{-31})$$
and the claim now follows from the lower bound on $V'$.
\end{proof}

\begin{lemma}\label{ivic-moment}\cite[Theorem 8.4]{ivic}\uses{zeta-moment-def}  We have $M(\sigma,A) = 1$ when $A$ is equal to
\begin{align*}
    \frac{4}{3-4\sigma} & \hbox{ for } 1/2 < \sigma \leq 5/8; \\
    \frac{10}{5-6\sigma} & \hbox{ for } 5/8 < \sigma \leq 35/54; \\
    \frac{19}{6-6\sigma} & \hbox{ for } 35/54 < \sigma \leq 41/60; \\
    \frac{2112}{859-948\sigma} & \hbox{ for } 41/60 < \sigma \leq 3/4;\\
    \frac{12408}{4537-4890\sigma} & \hbox{ for } 3/4 \leq \sigma \leq 5/6; \\
    \frac{4324}{1031-1044\sigma} & \hbox{ for } 5/6 \leq \sigma \leq 7/8; \\
    \frac{98}{31-32\sigma} & \hbox{ for } 7/8 \leq \sigma \leq 0.91591\dots; \\
    \frac{24\sigma-9}{(4\sigma-1)(1-\sigma)} & \hbox{ for } 0.91591\dots \leq \sigma < 1.
\end{align*}
Additionally, for $\sigma=2/3$ one can take $A = 9.6187\dots$, for $\sigma = 7/10$ one can take $A=11$, and for $\sigma=5/7$ one can take $A=12$.
\end{lemma}

\begin{proof}\uses{zeta-from-exp, ivic-lvt-82, mad} This is a computation using Lemma \ref{zeta-from-exp}, Theorem \ref{ivic-lvt-82}, and Lemma \ref{mad}; see \cite{ivic} for details.
\end{proof}

{\bf TODO: report the improvements from \cite{trudgian-yang}.}

\section{Large values of \texorpdfstring{$\zeta$}{zeta} moments}

It is also of interest to control large values of the moments.  For fixed $1/2 \leq \sigma \leq 1$, $A \geq 0$, and $h \geq 0$, let $M(\sigma,A,h)$ be the least (fixed) exponent for which the bound
$$ \int_{0 \leq t \leq T: |\zeta(\sigma+it)| \geq T^h} |\zeta(\sigma+it)|^A\ dt \ll T^{M(\sigma,A,h)+o(1)}.$$
A modification of the proof of the upper bound in Lemma \ref{mad} reveals that
\begin{equation}\label{msah}
    M(\sigma_0,A,h) \leq \sup_{\tau \geq 2; \sigma \geq 1/2, h\tau} (A(\sigma-\sigma_0) + \LV_\zeta(\sigma,\tau))/\tau.
\end{equation}
That is to say, any bound of the form
$$ \LV_\zeta(\sigma,\tau) \leq M \tau - A (\sigma - \sigma_0)$$
whenever $\tau \geq 2$ and $\sigma \geq 1/2, h\tau$, gives rise to a bound
$$ M(\sigma_0,A,h) \leq M.$$

A typical result concerning $M(\sigma_0,A,h)$ is as follows.

\begin{lemma}\label{ivic-6-large}\cite[(8.56)]{ivic} $M(1/2, 6, 11/72) \leq 1$.
\end{lemma}

\begin{proof}\uses{zeta-from-exp} Applying Lemma \ref{zeta-from-exp} with the exponent pair $(4/18, 11/18) = BABA(1/6, 2/3)$ we have
$$  \LV_\zeta(\sigma,\tau) \leq \min( \tau - 6(\sigma-1/2), 15\tau/4 - 24 (\sigma-1/2) ).$$
In particular, we have
$$ \LV_\zeta(\sigma,\tau) \leq \tau - 6(\sigma-1/2)$$
if $\sigma - 1/2 \geq 11 \tau/72$, giving the claim.
\end{proof}

\begin{lemma}\label{bourgain-remark-1}\cite[Proposition 2]{bourgain_remarks_1995}\uses{lv-def} Suppose that $M(1/2,A,h) \leq 1$ for some $A \geq 4$ and $h \geq 0$.  Then one has
    $$ \LV(\sigma,\tau) \leq \max( \alpha + 2 - 2 \sigma, -\alpha + \tau + A/2 - 2A (\sigma-1/2))$$
whenever $1/2 \leq \sigma \leq 1$, $\tau > 0$, and $0 \leq \alpha \leq 1-\sigma$ is such that
$$ \sigma - \frac{1}{2} > \frac{\tau h}{2} + \frac{1}{4}.$$
\end{lemma}

\begin{lemma}\label{bourgain-remark-2}\cite[Proposition 5]{bourgain_remarks_1995}\uses{lv-def, zeta-grow-def}  Suppose that $M(1/2,6,h) \leq 1$ for some $h \geq 0$.  Then for any $1/2 \leq \alpha < \sigma < 1$, one has
$$ A(\sigma) \leq \max( \frac{\mu(\alpha)}{\sigma-\alpha}, \frac{3}{8\sigma-5}, \frac{6h}{4\sigma-3}).$$
\end{lemma}

It is remarked in \cite{bourgain_remarks_1995} that this proposition could lead to some improvements in current zero density estimate bounds.
