\chapter{Zero free region for the zeta function}\label{zerofree-chapter}

\unintegrated 

A zero of the Riemann zeta function is a complex number $\rho = \beta + i\gamma$ for which $\zeta(\rho) = 0$. The zeta function has a infinite number of zeros of the form $\rho = -2n$ for integer $n \ge 1$; these are known as trivial zeros and are well understood. There are also an infinite number zeros inside the ``critical strip" $0 < \Re z < 1$, called non-trivial zeros. The locations of the non-trivial zeros have deep consequences for many fields of mathematics. A well-known conjecture regarding the non-trivial zeros is the Riemann hypothesis. 

\begin{conjecture}[Riemann hypothesis]\label{rh}
If $\rho$ is a non-trivial zero of the Riemann zeta function, then $\Re \rho = 1/2$.
\end{conjecture}

This conjecture remains far out of reach. Instead, currently what are known are zero free regions.  

\begin{definition}[Zero free region]\label{zeta-zero-free-def}
A zero-free region of the Riemann zeta function is a set $D \subset \mathbb{C}$ for which $\zeta(s) \ne 0$ for all $s \in D$. 
\end{definition}

\begin{lemma}[Basic properties of zero free regions]\label{zero-free-basic-lem}
The following properties hold:
\begin{enumerate}
    \item[(i)] (Symmetry about the real axis) If $\zeta(\sigma + it) \ne 0$ then $\zeta(\sigma - it) \ne 0$.
    \item[(ii)] (Symmetry about the critical line $\Re s = 1/2$) For $0 \le \sigma \le 1$, if $\zeta(\sigma + it) \ne 0$ then $\zeta(1 - \sigma + it) \ne 0$.
    \item[(iii)] (Non vanishing for $\Re s > 1$) If $\Re s > 1$ then $\zeta(s) \ne 0$.
\end{enumerate}
\end{lemma}
\begin{proof}
Claim (i) follows directly from the property $\overline{\zeta(s)} = \zeta(\overline{s})$. Claim (ii) follows from the functional equation 
\[
\zeta(s) = 2^s \pi^{s - 1}\sin(\pi s/2) \Gamma(1 -s)\zeta(1-s)
\]
and claim (iii) follows from the Euler product formula
\[
\zeta(s) = \prod_{p}(1 - p^{-s})^{-1},\qquad (\Re s > 1).
\]
\end{proof}

In light of \Cref{zero-free-basic-lem}, for the rest of the chapter we will focus on the quadrant 
\[
D\subseteq \{z \in \mathbb{C}: \Re z > 1/2, \Im z > 0\}.
\]
The first zero free region for non-trivial zeros proved was on $D = \{z \in \mathbb{C}: \Re z = 1\}$, and was used to prove the prime number theorem $\pi(x) \sim x/\log x$ as $x \to \infty$.

\begin{theorem}[Non-vanishing on the 1-line]
One has $\zeta(1 + it) \ne 0$ for any real $t$.
\end{theorem}

\section{Relation to upper bound on zeta in the critical strip}

Using estimates of $\zeta(\sigma + it)$ close to the line $\sigma = 1$, one can extend the zero free region slightly inside the critical strip. 

\begin{lemma}[Relation to growth exponents of zeta]\label{mu_to_zero_free}
Suppose $f(t) > 0$ and $0 < g(t) \le 1$ are real-valued functions for $t \ge 0$, with $f(t)$ non-decreasing and tending to infinity with $t$, and $g(t)$ non-increasing. Suppose further that $f(t)/g(t) = o(\exp(f(t)))$. If 
\[
\zeta(\sigma + it) \ll \exp(f(t))\qquad (1 - g(t) \le \sigma \le 2, t \ge 0)
\]
then $\zeta(\sigma + it) \ne 0$ for 
\[
\sigma \ge 1 - A\frac{g(2t + 1)}{f(2t + 1)}
\]
where $A > 0$ is an absolute constant. 
\end{lemma}
\begin{proof}
See \cite[Theorem 3.10]{titchmarsh_theory_1986}.
\end{proof}

The following zero free region of classical type was proved independently by de la Vall\'ee Poussin and Hadamard.

\begin{theorem}[Classical zero free region]\label{zfr-classical}
One has $\zeta(\sigma + it) \ne 0$ if 
\[
\sigma \ge 1 - \frac{A}{\log t}.
\]
for an absolute constant $A > 0$ and $t$ sufficiently large.
\end{theorem}
\begin{proof}
Apply \Cref{mu_to_zero_free} with $g(t) = 1/2$, $f(t) = \log (t + 2)$ and the convexity bound $\mu(\sigma) \le (1-\sigma)/2$.
\end{proof}

This classical result has been improved in a number of works, most of which make crucial use of non-trivial estimates of certain types of exponential sums.

\begin{theorem}[Littlewood zero free region]\label{zfr-littlewood}
One has $\zeta(\sigma + it) \ne 0$ if 
\[
\sigma \ge 1 - \frac{A \log\log t}{\log t}
\]
for an absolute constant $A > 0$ and $t$ sufficiently large.
\end{theorem}
\begin{proof}
Follows from the zeta bound corresponding to
\[
\mu(1 - \frac{k}{2^k - 2}) \le \frac{1}{2^k - 2}
\]
for integer $k\ge 3$, which is generated by the van der Corput exponent pair $A^{k - 2}B(0, 1) = (\frac{1}{2^k - 2}, 1 - \frac{k - 1}{2^k - 2})$. However, one needs to make explicit the $o(1)$ term in the bound $\zeta(\sigma + it) \ll t^{\mu(\sigma) + o(1)}$. In particular, by \cite[Theorem 5.14]{titchmarsh_theory_1986}, one has 
\[
\zeta(1 - \frac{k}{2^k - 2} + it) \ll t^{1/(2^k - 2)}\log t.
\]
Taking 
\[
k = \left\lfloor \frac{1}{\log 2}\log\left(\frac{\log t}{\log\log t}\right)\right\rfloor
\]
and using the Phragm\'en Lindel\"of principle, one has 
\[
\zeta(\sigma + it) \ll (\log t)^5,\qquad (\sigma \ge 1 - \frac{(\log\log t)^2}{\log t}),
\]
so we may take $f(t) = 5\log\log t$ and $g(t) = (\log\log t)^2/\log t$ in \Cref{mu_to_zero_free}.
\end{proof}

\begin{theorem}[Chudakov zero free region]\label{zfr-chudakov}
One has $\zeta(\sigma + it) \ne 0$ if 
\[
\sigma \ge 1 - \frac{1}{(\log t)^{3/4 + o(1)}}
\]
for $t$ sufficiently large.
\end{theorem}

\begin{theorem}[Korobov-Vinogradov zero free region]\label{zfr-vk}
One has $\zeta(\sigma + it) \ne 0$ if 
\[
\sigma \ge 1 - \frac{A}{(\log t)^{2/3}(\log\log t)^{1/3}}
\]
for an absolute constant $A > 0$ and $t$ sufficiently large.
\end{theorem}

\section{Relation to the error term in the prime number theorem}

\begin{lemma}[Relation to prime number theorem error term]\label{zero_free_to_pnt}
Suppose $\zeta(\sigma + it) \ne 0$ for $\sigma \ge 1 - \eta(t)$ where $\eta(t)$ is a positive and decreasing function. Then
\[
\sum_{n \le x}\Lambda(n) - x \ll x \exp\left(-A \omega(x) \right),\qquad (x \to \infty)
\]
for an absolute constant $A > 0$, where 
\[
\omega(x) := \inf_{t \ge 1}(\eta (t) \log x + \log t).
\]
\end{lemma}
\begin{proof}
See e.g. \cite{ingham_distribution_1990}.
\end{proof}

Applying \Cref{zero_free_to_pnt}, one obtains the error term estimates in the prime number theorem given in Table \ref{zero-free-pnt-table}.   

\begin{table}[ht]
    \def\arraystretch{2.5}
    \centering
    \caption{Error bounds on the prime number theorem. Here $A$ represents an absolute, positive constant, which may be different at each occurrence.}
    \begin{tabular}{|c|c|c|}
    \hline
    Bound on $(\psi(x) - x)/x$ & Associated zero-free region & Reference \\
    \hline
    $\exp(-A(\log x)^{1/2})$ & $\sigma \ge 1 - \dfrac{A}{\log t}$ & \Cref{zfr-classical} \\
    \hline 
    $\exp(-A(\log x \log\log x)^{1/2})$ & $\sigma \ge 1 - \dfrac{A\log\log t}{\log t}$ & \Cref{zfr-littlewood} \\
    \hline 
    $\exp\left(-A\dfrac{(\log x)^{3/5}}{(\log\log x)^{1/5}}\right)$ & $\sigma \ge 1 - \dfrac{A}{(\log t)^{2/3}(\log\log t)^{1/3}}$ & \Cref{zfr-vk} \\
    \hline 
    \end{tabular}
\label{zero-free-pnt-table}
\end{table}