\chapter{Zero free region for the zeta function}\label{zerofree-chapter}

\unintegrated 

A zero of the Riemann zeta function is a complex number $\rho = \beta + i\gamma$ for which $\zeta(\rho) = 0$. The zeta function has a infinite number of zeros of the form $\rho = -2n$ for integer $n \ge 1$; these are known as trivial zeros and are well understood. There are also an infinite number zeros inside the ``critical strip" $0 < \Re z < 1$, called non-trivial zeros. The locations of the non-trivial zeros have deep consequences for many fields of mathematics. A well-known conjecture regarding the non-trivial zeros is the Riemann hypothesis. 

\begin{conjecture}[Riemann hypothesis]\label{rh}
If $\rho$ is a non-trivial zero of the Riemann zeta function, then $\Re \rho = 1/2$.
\end{conjecture}

This conjecture remains far out of reach. Instead, currently what are known are zero free regions.  

\begin{definition}[Zero free region]\label{zeta-zero-free-def}
A zero-free region of the Riemann zeta function is a set $D \subset \mathbb{C}$ for which $\zeta(s) \ne 0$ for all $s \in D$. 
\end{definition}

\begin{lemma}[Basic properties of zero free regions]\label{zero-free-basic-lem}
The following properties hold:
\begin{enumerate}
    \item[(i)] (Symmetry about the real axis) If $\zeta(\sigma + it) \ne 0$ then $\zeta(\sigma - it) \ne 0$.
    \item[(ii)] (Symmetry about the critical line $\Re s = 1/2$) For $0 \le \sigma \le 1$, if $\zeta(\sigma + it) \ne 0$ then $\zeta(1 - \sigma + it) \ne 0$.
    \item[(iii)] (Non vanishing for $\Re s > 1$) If $\Re s > 1$ then $\zeta(s) \ne 0$.
\end{enumerate}
\end{lemma}
\begin{proof}
Claim (i) follows directly from the property $\overline{\zeta(s)} = \zeta(\overline{s})$. Claim (ii) follows from the functional equation 
\[
\zeta(s) = 2^s \pi^{s - 1}\sin(\pi s/2) \Gamma(1 -s)\zeta(1-s)
\]
and claim (iii) follows from the Euler product formula
\[
\zeta(s) = \prod_{p}(1 - p^{-s})^{-1},\qquad (\Re s > 1).
\]
\end{proof}

In light of \Cref{zero-free-basic-lem}, for the rest of the chapter we will focus on the quadrant 
\[
D\subseteq \{z \in \mathbb{C}: \Re z > 1/2, \Im z > 0\}.
\]
The first zero free region for non-trivial zeros proved was on $D = \{z \in \mathbb{C}: \Re z = 1\}$, and was used to prove the prime number theorem $\pi(x) \sim x/\log x$ as $x \to \infty$.

\begin{theorem}[Non-vanishing on the 1-line]
One has $\zeta(1 + it) \ne 0$ for any real $t$.
\end{theorem}

With enough work one can extend the zero free region slightly inside the critical strip. The following zero free region of classical type was proved independently by de la Vall\'ee Poussin and Hadamard.

\begin{theorem}[Classical zero free region]
One has $\zeta(\sigma + it) \ne 0$ if 
\[
\sigma \ge 1 - \frac{A}{\log t}.
\]
for an absolute constant $A > 0$ and $t$ sufficiently large.
\end{theorem}

This classical result has been improved in a number of works, most of which make crucial use of non-trivial estimates of certain types of exponential sums.

\begin{theorem}[Littlewood zero free region]
One has $\zeta(\sigma + it) \ne 0$ if 
\[
\sigma \ge 1 - \frac{A \log\log t}{\log t}
\]
for an absolute constant $A > 0$ and $t$ sufficiently large.
\end{theorem}

\begin{theorem}[Chudakov zero free region]
One has $\zeta(\sigma + it) \ne 0$ if 
\[
\sigma \ge 1 - \frac{1}{(\log t)^{3/4 + o(1)}}
\]
for $t$ sufficiently large.
\end{theorem}

\begin{theorem}[Korobov-Vinogradov zero free region]
One has $\zeta(\sigma + it) \ne 0$ if 
\[
\sigma \ge 1 - \frac{A}{(\log t)^{2/3}(\log\log t)^{1/3}}
\]
for an absolute constant $A > 0$ and $t$ sufficiently large.
\end{theorem}
